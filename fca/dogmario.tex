\documentclass{article}

\usepackage[margin=1in]{geometry}
\usepackage{amssymb}
\usepackage{mathtools}
\usepackage{ulem}

% TODO: Define argmax

\title{Dogmario FCA}
\author{ale-cci}

\begin{document}

\section{Teorema (poli di $\sum$ e stabilit\'a)}
\begin{enumerate}
    \item $\sum$ \'e \textit{stabile} se e solo se tutti i poli hanno parte reale non positiva e gli eventuali pli puramente immaginari sono semplici
    \item $\sum$ \'e \textit{asintoticamente stabile} se e solo se tutti i suoi poli hanno parte reale negativa
    \item $\sum$ \'e \textit{semplicemente stabile} se e solo se tutti i poli hanno partre reale non positiva e quelli puramente immaginari (che devono esistere) sono semplici
    \item $\sum$ \'e \textit{instabile} se e solo se esiste almeno un polo a parte reale positiva o un polo puramente immaginario con molteplicit\'a maggiore di uno
\end{enumerate}
\subsection{Teorema (Stabilit\'a BIBO)}
Un sistema $\sum$ \'e BIBO se e solo se $\int_0^{+\infty} | g(\tau) | d\tau < + \infty$


\subsection{Stabilit\'a asintotica $\Leftrightarrow$ Stabilit\'a BIBO}
$\sum$ \'e BIBO stabile se e solo se $\sum$ \'e asintoticamente stabile

\subsection{Polinomio di Hurwitz}
Un polinomio $a(s)$ \'e detto di Hurwitz se tutte le sue radici hanno parte reale negativa

\textbf{propriet\'a}: Il polinomio $a(s)$ \'e hurwitziano se e solo se tutti i suoi coefficenti sono positivi

\subsection{Criterio di Routh}
Il polinomio $a(s)$ \'e hurwitziano se e solo se l'associata tabella di Routh pu\'o essere completata (con l'algoritmo base) e presenta nella prima colonna solo permanenze.

(TODO:Aggiungere come calcolare il resto dei coefficenti)

I \textbf{Casi singolari} nella costruzione della Tablella di Routh avvengono quando il primo o tutti gli elementi di una riga sono nulli

\subsection{Metodo di Benidir-Picinbono}
Ogni riga, non nulla che nizia con $p$ zeri viene sommata con la riga da questa ottenuta moltiplicandola er ${(-1)}^p$ e traslandola verso sinistra di $p$ posizioni. (slide 23 lezione 8)

\[ a(s) - s^3 + 3s -2 = 0 \]
\begin{center}
    \begin{tabular} {c|c c c}
        3 & 1 & 3 & 0\\
        \hline
        $2^i$ & 0 & -2 & 0\\
        $2^{ii}$ & 2 & 0 & 0\\
        \hline
        2 & 2 & -2 & 0\\
        1 & 4 & 0\\
        0 & -2
\end{tabular}
\end{center}

\subsection{Prosecuzione tabella nel caso di riga tutta nulla}

\begin{enumerate}
    \item Derivare polinomio ausiliario
    \item i coefficenti ottenuti dalla derivata sostituiscono gli zeri della riga nulla.
    \item Proseguire la tabella nel modo usuale
\end{enumerate}

\begin{center}
    \begin{tabular}{c|c c c c c c}
        2i & $\gamma_{n-2i,1}$ & $\gamma_{n-2i,2}$ & $\gamma_{n-2i,3}$  & $\cdots$ & $\gamma_{n-2i,i+1}$  & 0 \\
        2i -1 & 0 & 0 & 0 & $\cdots$ & 0 & 0
    \end{tabular}
\end{center}

Polinomio ausiliario:
$ \beta(s) = \gamma_{n-2i,1} s^{2i} + \gamma_{n-2i,1} s^{2i-2} +\gamma_{n-2i,3} s^{2i - 4}  + \cdots \gamma_{n-2,1} s^{2} +  \cdots \gamma_{n-2i,1} $

Equazione ausiliaria: $ \beta(s) = 0$

\underline{Esempio}:

\begin{minipage}{0.45\textwidth}
    \[ a(s) = s^6 + s^5 - 2s^4 -3s^3 - 7s^2 -4s -4 \]
    \begin{center}
        \begin{tabular}{c|c c c c c}
            6 & 1 & -2 & -7 & -4 & 0\\
            5 & 1 & -3 & -4 & 0\\
            4 & 1 & -3 & -4 & 0\\
            \hline
            3 & 2 & -3 & 0 & 0\\
            2 & -3 & -8 & 0\\
            1 & -25 & 0\\
            0 & -8
        \end{tabular}
    \end{center}
\end{minipage}
\begin{minipage}{0.5\textwidth}
    polinomio ausiliario $\beta(s) = s^4 -3s^2 -4$

    $D\beta(s) = 4s^3 - 6s$
\end{minipage}

% teorema di analisi armonica
\subsection{Teorema di analisi armonica}
Dato $\sum$ sistema asintoticamente stabile cond f.d.t. $G(s) \in \mathbb{Q}$. La risposta forzata di $\sum$ ad un segnale armonico \'e un segnale armonico con stessa frequenza dell'ingresso.
\[ F(w) = G(jw) \]

(Dimostrazione a Lezione 9 slide 6)

Rappresentazioni grafiche della funzione di risposta armonica sono i Diagrammi di Bode, diagrammi di Nyquist.
% sistema dinamico lineare come filtro
% Rappresentazioni standard di G e G
% Diagrammi di bode elementari
% Diagrammi di bode asintotici

\subsection{Parametri caratteristici della risposta armonica}
\begin{enumerate}
    \item Pulsazione di risonanza $w_r \coloneqq \arg\max_{w\in\mathbb{R+}}|G(jw)|$
    \item Picco di risonanza $M_R \coloneqq \frac{|G(jw_R)|}{|G(j0)|}$ oppure $M_R \coloneqq |G(jw_R)|$
    \item Larghezza di banda $B_w \coloneqq w_{t2} - w_{t1} \ge 0$
\end{enumerate}

\section{Diagramma di Nyquist}
Curva tracciata sul piano complesso dal vettore $G(j\omega)$ per $\omega$ che varia da 0 a $+\infty$

\section{Sistemi a fase minima}
Nella risposta armonica l'andamento delle fasi \`e strettamente legato a quello delle ampiezze

\section{Approssimate di Pad\'e}
Approssimazione del ritardo finito con una funzione razionale


$T_{ry}(s) = \frac{G(s)}{1 + L(s)}$

Guadagno d'anello $L(s) \coloneqq G(s)H(s)$

Un \textbf{sistema retroazionato} \'e \textbf{ben connesso} se $\lim\limits_{|s| \to +\infty} 1 + L(s) \neq 0$

\section{Teorema dell'indice logaritmico}
Se $\Gamma$ \'e una curva su $\mathbb{C}$ e $\mathcal{D}$ la regione contenuta al suo interno, Data $F(s)$ una funzione analitica  su $\Gamma \cup \mathcal{D}$ ad eccezione di un numero finito di poli in $\mathcal{D}$, e senza zeri su $\Gamma$, allora vale la relazione:
\[\frac{1}{2\pi} \Delta \arg F(s) = n_z - n_p \]

Dove:
\begin{itemize}
    \item $\Delta\arg F(s)$ \`e la variazione dell'argomento di $F(s)$ lungo $\Gamma$ per un giro completo antiorario
    \item $n_z$ e $n_p$ sono rispettivamente il numero di zeri e poli di $F(s)$ su $\mathcal{D}$ (contati con molteplicit\'a)
\end{itemize}

\subsection{Corollario\label{corollario_logaritmo}}
Con stesse ipotesi si ha che $\psi = n_z - n_p$ = numero di giri in senso antiorario di $\Gamma$

\subsection{Applicato a Nyquist}
\`E possibile applicare questo con $\Gamma$ = Contorno di Nyquist se
\begin{itemize}
    \item $1+L(s)$ \'e analitica sul contorno ed analitica su $\mathbb{C}^+$ ad eccezione di un numero finito di poli
    \item $1+L(s)$ non deve avere zeri sul contorno $\Rightarrow L(s) \neq -1 \qquad \forall s \in \Gamma$
\end{itemize}

Chiamiamo \textbf{Diagramma Polare completo} la curva chiusa immagine di $L(s)$ su $\Gamma$

\section{Criterio di Nyquist}
Un sistema in retroazione \'e asintoticamente stabile se e solo se il d.p.c non tocchi il punto critico -1, ma lo circondi tante volte in senso antiorario quanti sono i poli del guadagno di anello con parte reale positiva.

\subsection{Caso particolare}
Nel caso in cui non abbia poli a parte reale positiva, il d.p.c non deve ne toccare ne circondare il punto -1.

\section{Margine di fase ed Ampiezza}
\textbf{Margine d'ampiezza}: ${\displaystyle M_A \coloneqq \frac{1}{|L(j\omega_p)|}}$ dove $\omega_p \ni \arg L(j\omega_p) = -\pi$\\
$\omega_p \equiv$ pulsazione di fase $\pi$\\
\textbf{Margine di fase}:  ${\displaystyle M_F \coloneqq \pi - |\varphi_c|}$ dove $\varphi_c = \arg L(j\omega_c)$ e $\omega_c \ni |L(j\omega_c)| = 1$\\
$\varphi_c \in (-\pi, 0)$, $\omega_c \equiv$ pulsazione critica

I margini di stabilit\'a sono `norme'' che misurano la distanza del punto dritico $-1$ dal diagramma polare di $L(j\omega)$

\subsection{Propriet\'a}
Sia $M>1$, se $L(j\omega) \notin \left[-M,-\frac{1}{M}\right]$ alora vale la disequazione $|1 + \gamma L(j\omega)| > 0 \quad \forall \gamma \in \left[\frac{1}{M}, M\right] \quad \forall \omega \ge 0$

\section{Luogo delle radici}
Luogo delle radici (diretto) \`e il luogo geometrico descritto dalle radici dell'equazione $1+K_1G_1(s) = 0$ al variare di $K_1$ da 0 $+\infty$

Luogo delle radici (inverso) stessa roba ma con $K_1$ che varia da 0 a $-\infty$

\subsection{Propiet\'a del luogo}
\begin{itemize}
    \item Il luogo ha tanti rami quanti sono i poli di $G_1(s)$
    \item Ogni ramo parte da un polo di $G_1(s)$ e termina in uno zero di $G_1(s)$ o in un punto all'infinito
    \item I rami si intersecano in corrispondenza di radici multiple
    \item Il luogo \'e simmetrico rispetto all'asse reale
    \item Nel luogo delle radici diretto un punto dell'asse reale fa parte del luogo se si lascia alla sua destra un numero totale \textit{dispari} di zeri e poli di $G_1(s)$
\end{itemize}


\section{Angoli di partenza e di arrivo (Slide 10 Lezine 13)}
nel luogo delle radici diretto $K_1 > 0$ l'angolo di partenza $\varphi_0$ da un polo $p_i$ semplice \`e dato dalla relazione:
\[ \varphi_0 = \pi + \sum\limits_{j=1}^m \arg(p_i - z_j) - \sum\limits_{j\neq i} (p_i - p_j) \]
l'angolo di arrivo sullo zero $z_i$ semplice \`e dato da
\[ \varphi_a = \pi + \sum\limits_{j=1}^n \arg(z_i - p_j) - \sum\limits_{j\neq i} \arg(z_i - z_j)\]
Se il luogo delle radici \`e inverso, sis sostituisce nelle relazioni 0 a $\pi$.

\section{Teorema del bericentro del luogo delle radici}
Se il guadagno di anello ha grado relative $\rho \ge 2$ allora vale la relazione:
\[ \sum\limits_{i=1}^n p_{Ci} = \sum\limits_{i=1}^n p_i \qquad \forall K_1 \in \mathbb{R} \quad \text{e} \quad \forall z_i \in \mathbb{C}, i=1,\ldots,m\]

(Dimostrazione slide 24 lezione 13)

\textbf{Grado di stabilit\'a} di $\sum$: $ G_s \coloneqq -\max\{\Re p_1, \Re p_2, \ldots, \Re p_n\} $, ovvero la distanza minima dei poli dall'asse immaginario
\newpage
\section{Dubbi domande perplessit\'a ed incertezze (Domande per Felice)}
\begin{enumerate}
    \item Se non si riesce a completare la tabella di routh normalmente il sistema pu\'o essere stabile?

        Bisogna vedere le permanenze e le variazioni di segno, non c'entra come se una riga risulta di tutti 0
    \item Cos'\`e un sistema strettamente proprio?

        Da vedere roba del primo parziale
    \item \sout{Come si trova l'ascissa dell'asintoto  $G(jw) = \frac{10}{jw(1 + j2w)}$? }

        Formula dogmatica a in slide 6 lezione 10
    \item \sout{Ci sono esercizi con Approssimate di Pad\'e?}

        Esercitazione 10 es 1
    \item{\sout{In~\ref{corollario_logaritmo} $\psi$ \'e il numero di giri intorno all'origine, quando viene applicato a Nyquist diventa il numero di giri intorno a $-1$, why?}}

        $\phi$ \'e il numero di giri intorno all'origine della funzione $1+L(s)$, segue per traslazione segue che \'e il numero di giri di $L(s)$ intorno a $0$
    \item Da dove salta fuori $1 + L(s)$?

        Denominatore del guadagno ad anello?

    \item \sout{In esercitazione 5 dopo esercizio 1, da dove salta fuori il sistema con $2K+8>0$?}

        Mancava una parte dagli appunti

    \item Da Esercitazione 6, perch\`e la fase di $G_1(s)$ va a $\frac{3}{2} \pi$ e non a $\frac{3}{4} \pi$?

    \item Da dove salta fuori il vettore rosso $G_2(s)$ Esercitazione 6 (foto 135823)
    \item Perch\`e si studia la stabilit\`a chiudendo il diagramma di Nyquist (A partire da esercitazione 7)
    \item Foto 135837 Mossa
    \item Come si fa in caso di doppio polo nell origine ($\rho \to 0$)?
\end{enumerate}

\end{document}
