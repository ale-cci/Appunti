\documentclass{article}

\usepackage[margin=1in]{geometry}
\usepackage{amssymb}
\usepackage{mathtools}

% TODO: Define argmax

\title {Dogmario FCA}
\author{ale-cci}

\begin{document}

\subsection{Teorema (Stabilit\'a BIBO)}
Un sistema $\sum$ \'e BIBO se e solo se $\int_0^{+\infty} | g(\tau) | d\tau < + \infty$

\subsection{Stabilit\'a asintotica $\Leftrightarrow$ Stabilit\'a BIBO}
$\sum$ \'e BIBO stabile se e solo se $\sum$ \'e asintoticamente stabile

\subsection{Polinomio di Hurwitz}
Un polinomio $a(s)$ \'e detto di Hurwitz se tutte le sue radici hanno parte reale negativa

\textbf{propriet\'a}: Il polinomio $a(s)$ \'e hurwitziano se e solo se tutti i suoi coefficenti sono positivi

\subsection{Criterio di Routh}
Il polinomio $a(s)$ \'e hurwitziano se e solo se l'associata tabella di Routh pu\'o essere completata (con l'algoritmo base) e presenta nella prima colonna solo permanenze.

I \textbf{Casi singolari} nella costruzione della Tablella di Routh avvengono quando il primo o tutti gli elementi di una riga sono nulli

\subsection{Metodo di Benidir-Picinbono}
Ogni riga, non nulla che nizia con $p$ zeri viene sommata con la riga da questa ottenuta moltiplicandola er $(-1)^p$ e traslandola verso sinistra di $p$ posizioni. (slide 23 lezione 8)

\[ a(s) - s^3 + 3s -2 = 0 \]
\begin{center}
    \begin{tabular} {c|c c c}
        3 & 1 & 3 & 0\\
        \hline
        $2^i$ & 0 & -2 & 0\\
        $2^{ii}$ & 2 & 0 & 0\\
        \hline
        2 & 2 & -2 & 0\\
        1 & 4 & 0\\
        0 & -2
\end{tabular}
\end{center}

\subsection{Prosecuzione tabella nel caso di riga tutta nulla}

\begin{enumerate}
    \item Derivare polinomio ausiliario
    \item i coefficenti ottenuti dalla derivata sostituiscono gli zeri della riga nulla.
    \item Proseguire la tabella nel modo usuale
\end{enumerate}

\begin{center}
\begin{tabular}{c|c c c c c c}
    2i & $\gamma_{n-2i,1}$ & $\gamma_{n-2i,2}$ & $\gamma_{n-2i,3}$  & $\cdots$ & $\gamma_{n-2i,i+1}$  & 0 \\
    2i -1 & 0 & 0 & 0 & $\cdots$ & 0 & 0
\end{tabular}
\end{center}

Polinomio ausiliario:
$ \beta(s) = \gamma_{n-2i,1} s^{2i} + \gamma_{n-2i,1} s^{2i-2} +\gamma_{n-2i,3} s^{2i - 4}  + \cdots \gamma_{n-2,1} s^{2} +  \cdots \gamma_{n-2i,1} $

Equazione ausiliaria: $ \beta(s) = 0$

\underline{Esempio}:

\begin{minipage}{0.45\textwidth}
    \[ a(s) = s^6 + s^5 - 2s^4 -3s^3 - 7s^2 -4s -4 \]
    \begin{center}
        \begin{tabular}{c|c c c c c}
            6 & 1 & -2 & -7 & -4 & 0\\
            5 & 1 & -3 & -4 & 0\\
            4 & 1 & -3 & -4 & 0\\
            \hline
            3 & 2 & -3 & 0 & 0\\
            2 & -3 & -8 & 0\\
            1 & -25 & 0\\
            0 & -8
        \end{tabular}
    \end{center}
\end{minipage}
\begin{minipage}{0.5\textwidth}
    polinomio ausiliario $\beta(s) = s^4 -3s^2 -4$

    $D\beta(s) = 4s^3 - 6s$
\end{minipage}

% teorema di analisi armonica
\subsection{Teorema di analisi armonica}
Dato $\sum$ sistema asintoticamente stabile cond f.d.t. $G(s) \in \mathbb{Q}$. La risposta forzata di $\sum$ ad un segnale armonico \'e un segnale armonico con stessa frequenza dell'ingresso.
\[ F(w) = G(jw) \]

(Dimostrazione a Lezione 9 slide 6)

Rappresentazioni grafiche della funzione di risposta armonica sono i Diagrammi di Bode, diagrammi di Nyquist.
% sistema dinamico lineare come filtro
% Rappresentazioni standard di G e G
% Diagrammi di bode elementari
% Diagrammi di bode asintotici

\subsection{Parametri caratteristici della risposta armonica}
\begin{enumerate}
    \item Pulsazione di risonanza $w_r \coloneqq $
    \item Picco di risonanza $M_R$
    \item Larghezza di banda
\end{enumerate}


\end{document}
