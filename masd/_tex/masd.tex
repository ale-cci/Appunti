\documentclass{article}
\usepackage[margin=1in]{geometry}

\title{Modelli ed Algoritmi per il supporto alle decisioni}
\author{ale-cci}

\begin{document}
\maketitle{}
\section{Riunione in Generale}
\subsection{TODO}
\begin{itemize}
    \item \textbf{Dati}:
        Grandezze di un sistema reale, il cui valore é fissato a priori. (input del programma)
    \item \textbf{Varabili Decisionali}:
        Grandezze del sistema su cui abbiamo controllo diretto.
        A diversi valori delle variabili decisionali corrispondono diverse realizzazioni del sistema
    \item \textbf{Vincoli}:
        Condizioni che imposte sui valori che possono assumere le variabili decisionali.
        L'insieme dei vincoli forma l'\textit{insieme delle soluzioni possibili} del sistema reale,
        detta anche \textit{Regione ammissibile del problema}
    \item \textbf{Obbettivo}:
        Criterio di confronto applicato alle varie soluzioni del sistema reale.
\end{itemize}

\subsection{Procedura di soluzione}

\subsubsection{}
Una volta definito un modello, si passa attraverso un'algoritmo per calcolare i miglori valori da assegnare alle variabili decisionali, in modo da ottenere la miglior realizzazione possibile del sistema.

\subsubsection{Validazione del modello}
Ri-verificare la validitá del modello. In caso modificare il modello adattandolo al sistema reale, e ri-applicare l'algoritmo.

- costruire modello di un sistema reale:
    - identificare dati, variabili decisionali, vincoli del sistema reale e definire un obbiettivo
- processare il modello attraverso un algoritmo, in grado di determinare i valori delle variabili
-  validazione del modello: guardare soluzione ottenuta e determinare se accettabile
    - se non accettabile si aggiorna il modello e si torna al passo 2
- realizzare il sistema reale

\subsection{Tipi di modello}
\begin{itemize}
    \item Modelli a scala:
        riprodotto su scala ridotta (es. plastico)
    \item Modelli matematici:
        Ciascuna componente viene tradotta in oggetti matematici
    \item Modelli di simulazione:
        Componenti vengono rappresentate attraverso componenti di tipo informatico

        Hanno vantaggi e svantaggi circa complementari al modello matematico.
        Hanno maggiore capacitá di rappresentazione.

        computer utilizzato sia in fase di rappresentazione che in fase di risoluzione.
\end{itemize}
Ci concentriamo su modelli di tipo matematico, hanno il vantaggio di poter utilizzare tutti gli strumenti della matematica. Hanno lo svantaggio di avere una capacitá espressiva limitata.
Non tutti i sistemi reali sono rappresentabili in maniera ottimale attraverso sistemi matematici, risulterebbero troppo complessi

Utilizzo dei computer durante la fase di risoluzione.
\subsection{Tipi di algoritmi}
\begin{itemize}
    \item Algoritmi di tipo costruttivo:
        costruiti in modo progressivo, partono da una soluzione vuota, e progressivamente vengono aggiunti pezzi per raggiungere una soluzione completa
        -  Senza revisione di decisioni passate:
            Una volta aggiunti pezzi non vengono piu tolti
        - revisioni decisioni passate:
            pezzi aggiunti possono essere sostituiti

    \item Raffinamento Locale:
        partire da una soluzinoe possibile che rispetta tutti i vincoli, e si procede a raffinare progressivamente la soluzione.
        Richiede di definire un concetto di \textit{Vicinanza alla soluzione}
    \item Enumerazione:
        enumera le possibili soluzioni possibili per andare a cercare la soluzione migliore
        - eumerazione completa
            Prendere individualmente ogni soluzione, e valutarla esplicitamente e prendere la migliore tra esse
        - eumerazione implicita
            Enumerare impl. initeri sottinsiemi attraverso restrizione di sottoproblemi
            Valutazione di interi gruppi di soluzioni attraverso soluzioni di sottoproblemi
\end{itemize}

Analizzare problema reale, individuare le 4 componenti e Costruzione modello matematico

\subsection{Es. 1}
Effettuare investimento di quantitá $x$. Fonte di ricavo $\sqrt{x} + \frac{1}{2}x$.
\\
Problema: quantitá di denaro da investire per massimizzare $x$.

Dati: formula di ricavo : $\sqrt{x} + \frac{1}{2}x$
\\
Variabili decisionali: $x$
\\
Vincoli: no vincoli
\\
Obbiettivo: massimizzare il guadagno

\subsection{Es. 2 - Trasporto sacchi}
Trasportare numero fissato di carichi, in modo da minimizzare il costo di trasporto.

\begin{enumerate}
    \item Grandezze che non sono sotto il nostro controllo:
        \\
        Quantitativi minimi di ogni sacco,
        \\
        Sacchi trasportabili da ogni automezzo e costo relativo dell'automezzo
    \item Variabili decisionali:
        \\
        Numero di automezzi di ciascun tipo da utilizzare
    \item Vincoli:
        \\
        Variabili decisionali non negative
        \\
        Quantitativo minimo di sacchi da portare
    \item Obbiettivo:
        \\
        Costo complessivo del trasporto minimo
\end{enumerate}
\subsubsection{Trasformazione dei dati in concetto matematico}
Dati sono gia rappresentati
Approfondimento: Guarda slide

\section{Lezione 2 - Introduzione ai grafi}
Forma canonica
Cammino semplice: nessun arco é percorso piu di una volta
Cammino elementare: nessun nodo viene toccato piu di una volta
Circuito hemiltoniano: toccano tutti i nodi del grafo

Individuazione componenti connesse: BFS
matching: sottinsieme di archi non adiacenti
Riconoscimento grafi bipartiti 1:19:21, esempio: 1:21:49


\end{document}
