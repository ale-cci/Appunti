\documentclass[../template]{subfiles}

\begin{document}
\section{Branch and bound}
Il generico problema associato a questo algoritmo è determinare il massimo valore che può assumere una funzione $f(x)$.
\subsubsection{Upper Bound}
Considerato un insieme $T \subseteq S$, un' upper bound per un insieme $T$ è un valore $U(T)$ tale che $U(T) \ge f(x) \forall x \in
T$.

Il valore di $U(T)$ è calcolato tramite una procedura che deve avere tempo di esecuzione minore possibile, e precisione
più elevata possibile.

Indicato con $\alpha(f, T) = \max_{x\in T} f(x)$ il valore ottimo della funzione $f$ su $T$,
è definito rilassamento:
\[
    \alpha(f', T') = \max_{x \in T'} f'(x)
\]
Dove $T \subseteq T'$ e $f'(x) \ge f(x) \quad \forall x \in T$.

Il rilassamento è un upper bound del sottoinsieme $T$, dato che $\alpha(f', T') \ge \alpha(f, T)$:
chiamata $\bar{x}\in T$ una soluzione ottima del problema su $T$ , e $x' \in T'$ una soluzione ottima del rilassamento;
Siccome $T \subseteq T'$, allora $\bar{x} \in T'$ e $f'(\bar{x}) \ge f(\bar{x})$, da cui $\alpha(f', T') \ge \alpha(f,
T)$.

Per trovare un' upper bound di un insieme, basta risolvere quindi il problema di rilassamento.
\subsubsection{Lower Bound}
Indicato con $L_B$, rappresenta la limitazione inferiore della funzione $f$ sull'intero insieme $S$:
\[
    L_B \le f(x) \quad \forall x \in S
\]
Per ottenere una precisione più elevata possibile, dato un insieme di punti $y_1\dots y_h \in S$
\[
    L_B = \max \big\{f(y_i) : i=1\dots h\big\} \le f(x)
\]
I punti $y_1\dots y_h$ sono spesso individuati attraverso un euristica
\footnote{Rapido algoritmo che restituisce delle buone soluzioni ammissibili ma non necessariamente ottime}.
%TODO: missing second point before min.29
\subsection{Branching}
L'operazione consiste nel sostituire l'insieme $T \subseteq S$ con una sua partizione $T_1 \dots T_m$.
$T_1\dots T_m$ formano una partizione se e solo se:
\[
    T = \bigcup^m_{i=1}T_i \qquad \bigwedge \qquad T_i \cap T_j = \emptyset \quad \forall i \neq j
\]
La partizione è rappresentata con un albero.
\subsubsection{Eliminazione di sottoinsiemi}
Un insieme $T_i$ viene eliminato se $U(T_i) \le L_B$. In tale insieme non può essere contenuto il massimo di $f$.
\subsection{Algoritmo branch and bound}
\begin{enumerate}
    \item
        Posto $C = \{S\}$ il set di sottoinsiemi analizzabili, e $Q = \emptyset$ il set di insiemi eliminati,
        si calcoli $U(S)$ e $L_B$ in caso si disponga di un euristica, altrimenti $L_B=-\infty$.

    \item\label{branch_loop}
        Selezionato $T \in C$, ad esempio il sottoinsieme con upper bound maggiore: $U(T) = \max_{Q \in C} U(C)$
    \item
        Si sostituisca $T$ in $C$ con la sua partizione in $k$ sottoinsiemi:
        $C = C \cup \{T_1 \dots T_k\} - \{T\}$
    \item
        Calcolare per ciascuno dei sottoinsiemi il valore $U(T_i) \quad \forall i \in 1\dots k$
    \item
        Aggiornare il valore $L_B$ con il massimo valore di $f$ osservato durante l'esecuzione dell'algoritmo.
    \item
        Spostare da $C$ a $Q$ tutti i sottoinsiemi per cui $U(T') \le L_B$:
        $C = C - \{T': U(T') \le L_B\}$ , $Q = Q \cup \{T': U(T') \le L_B\}$
    \item Se $C = \emptyset$, allora $L_B$ coincide con il valore ottimo. Altrimenti ripartire dal passo
        \ref{branch_loop}
\end{enumerate}
Esistono problemi la cui regione ammissibile è vuota, per questo il loro lower bound risulta $-\infty$.
\subsubsection{Dimostrazione che $L_B$ è soluzione ottima}
Durante tutto l'algoritmo $C + Q$ forma una partizione di $S$, quindi al termine dell'algoritmo esiste sicuramente $T
\in Q$ tale che $\bar{x} \in T$, ma dato che $T$ è stato cancellato, allora
\[
    f(\bar{x}) \le U(T) \le L_B \le f(\bar{x})
\]
da cui $L_B = f(\bar{x})$.

Per trovare il minimo di una funzione, o consideriamo $-f(x)$, o invertiamo il ruolo di $L_B$ ed $U_B$

%euristiche 26
\subsection{Knapsack Problem}
È dato uno zaino con capacità $b$ (intero positivo) ed un numero di oggetti $n$.
A ciascun oggetto $i$ è associato un peso $p_i$ ed un valore $v_i$.
\\
Il problema consiste trovare un gruppo di oggetti il cui peso sia inferiore alla capacità $b$, e valore cumulativo sia massimo.

Applichiamo un modello matematico per studiare questo modello.
Indichiamo con $x_i = 1$ gli oggetti che sono presi nello zaino, e con $x_i = 0$ gli oggetti non presi.
In tal caso $\sum^n_i p_i x_i \le b$ e $\sum_i^n v_i x_i$ è massima.
\subsubsection{Calcolo dell'upper bound}
Aggiungendo l'ipotesi che gli oggetti possano essere inseriti parzialmente: $x \in [0,1]$
\begin{enumerate}
    \item Riordiniamo gli oggetti in ordine non crescente rispetto ai rapporti peso/valore.
    \item Calcoliamo i valori $b - p_1$, $b - p_1 - p_2$, \ldots
        Calcoliamo il massimo valore di $r$ per cui $b - \sum^{r+1}_{i=1} p_i > 0$ risulta vera.
        \\
        Nel caso in cui non si arriva ad un valore negativo, allora la soluzione ottima è mettere tutti gli oggetti nello zaino.
    \item \label{enum:calc_ub} La soluzione ottima del rilassamento lineare è
        \begin{align*}
            &x_1\dots x_r = 1 \quad , \quad x_{r+2} \dots x_n = 0 \\
            &x_{r+1} = \frac{b - \sum^r_{j=1} p_j}{p_r +1}
        \end{align*}
        con valore ottimo $\sum^r v_j + v_{r+1}\frac{b - \sum^r p_j}{p_{r+1}}$.
    \item $x_1 \dots x_r$ è una soluzione ammissibile al problema originario, quindi utilizziamo quella per calcolare il lower bound.
        Per upper bound, prendiamo il valore ottimo calcolato al punto \ref{enum:calc_ub} arrotondato per difetto.
\end{enumerate}

\subsubsection{Operazione di branching}
Indichiamo con $S(I_0, I_1)$ tutti i sottoinsiemi che non contengono elementi in $I_0$ ma contengono tutti gli elementi appartenenti ad $I_1$.
Supponendo di non essere nel caso banale in cui non tutto gli oggetti sono contenuti nello zaino,
utilizziamo la regola di branching $S(\{r + 1\}, \emptyset)$ tutti i sottoinsiemi in cui non è presente l'oggetto $r +1$ e $S(\emptyset, \{r + 1\})$ tutti i sottoinsiemi in cui è presente $r + 1$.

Per calcolare l'upper bound di questi insiemi al punto \ref{enum:calc_ub} basta forzare nella formula tutti gli elementi in $I_1$ e non prendere in considerazione gli elementi in $I_0$.

\subsubsection{Procedura di calcolo}
\begin{enumerate}
    \item Caso base: se $b - \sum_{i \in I_1} p_i < 0$ allora il nodo non ha soluzioni ammissibili. In tal caso poniamo $U(S(I_0, I_1)) = -\infty$.
    \item Altrimenti trovo il massimo valore di $r$ per cui $b - \sum_{i \in I_1} p_i -\sum^{r}_{i \notin I_1} p_i \ge 0$.
    \item Se esiste tale valore di $r$ allora l'upper bound di tale insieme è
        \[
            \sum_{i \in I_1} v_i + \sum_{h=1}^r v_{i_h} + \Big\lfloor v_{i_{r+1}} \frac{b - \sum_{i\in I_1} p_i - \sum^r_{h=1} p_{i_h}}{p_{i_{r+1}}} \Big\rfloor
        \]
        Con soluzione ammissibile $N = I_1 \cup \{i_1\dots i_r\}$
    \item Se si sottraggono i pesi di tutti gli oggetti senza mai arrivare ad un valore negativo,
        l'upper bound sarà semplicemente:
        \[
            \sum_{i \in I_1} v_i + \sum_{h=1}^k v_{i_h}
        \]
        Con soluzione ammissibile $N = I_1 \cup I_f \in S$
\end{enumerate}
\subsubsection{Complessità}
Il caso peggiore si ha quando non si riesce a cancellare alcun nodo. In tal caso il numero di  nodi generati è dell'ordine $2^n$.
Quindi non è migliore rispetto ad un algoritmo di enumerazione, ma mediamente il tempo di esecuzione è minore.
Questo è valido per tutti gli algoritmi di branch and bound.
\end{document}
