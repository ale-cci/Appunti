\documentclass[../template]{subfiles}

\begin{document}
\section{Approssimazione TSP Metrico}
Introduciamo un caso particolare di TSP in cui esistono restrizioni sui valori che possono assumere le distanze lungo gli archi del grafo:
\begin{itemize}
    \item $d_{ij} = d_{ji}$ il grafo è bidirezionato
    \item $d_{ij} \ge 0$ tutte le distanze sono non negative
    \item le distanze soddisfano tutte la disuguaglianza triangolare: $d_{ij} + d_{jk} \ge d_{ik}$
\end{itemize}

\subsection{Algoritmo Double Spanning Tree}
\begin{enumerate}
    \item Dato il grafo $G = (V, A)$ si determini un albero di supporto a costo minimo $T = (V, A_T)$ di tale grafo.
    \item Si duplichi ogni arco $A_T$ assegnando ad ogni arco la stessa distanza dell'arco originale.
        Sul grafo risultante si determini un ciclo euleriano (vedi pg..)
    \item Dalla sequenza dei nodi del ciclo, si rimuovono tutte le ripetizioni di nodi a parte quella dell'ultimo nodo. Ottenendo un circuito hemiltoniano.
\end{enumerate}
\subsubsection{Note}
L'algoritmo DST richiede un tempo di esecuzione polinomiale rispetto alla dimensione dell'istanza.

Un grafo ammette un ciclo euleriano se e solo se tutti i suoi nodi hanno grado pari, o due nodi dispari con stesso grado.
Nel nostro caso, avendo raddoppiato tutti gli archi dell'albero di supporto ottimo, la condizione è sicuramente soddisfatta.

\subsection{Calcolo di un ciclo euleriano}
\begin{enumerate}
    \item Dato un albero $T = (V, A_T)$ fissato un nodo radice $v'$, si ponga:
        $S = \emptyset$,
        $i_1 = v'$,
        $k = 2$,
        $w = v'$
    \item Se $w$ ha nodi figli in $V - S$ allora si selezioni il suo nodo figlio $z \in V - S$, ponendo $w = z$, $i_k = z$, $k = k+1$
        e si ripeta il passo 2.
        \\
        Se $w$ non ha figli in $V - S$ e $w \neq v'$ si risalga al nodo padre $y$ di $w$ e si ponga
            $S = S \cup {w}$, $w = y$, $i_k = y$, $k=k+1$ e si ripeta il passo 2
        \\
        Altrimenti si termina l'esecuzione
\end{enumerate}
\subsubsection{Osservazione}
Per il TSP  metrico, 'algoritmo DST è un algoritmo di 1-approssimazione.
\subsubsection{Dimostrazione}
...


\end{document}
