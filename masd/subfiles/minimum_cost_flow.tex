\documentclass[../template]{subfiles}

\begin{document}
\section{Flusso a costo minimo}
\subsection{Grafo di flusso}
Dato un grafo orientato e connesso $G = (V, A)$, viene indicato con $c_{ij}$ il costo di trasporto su un arco, $d_{ij}$ la
capacità massima di trasporto e con $b_{i}$ la quantità di prodotto che ogni nodo è in grado di produrre.

I nodi si possono distinguere in tre gruppi in base al valore di $b_i$
\begin{enumerate}
    \item Nodi sorgente $b_i > 0$, viene realizzato il prodotto che circola internamente alla rete
    \item Nodi di transito: $b_i = 0$, dove il prodotto rimane invariato e si limita a transitare.
    \item Nodi destinazione: $b_i < 0$, viene consumato il prodotto della rete
\end{enumerate}
La proprietà $\sum_i^n b_i = 0$, quando non risulta valida, è forzabile aggiungendo un fittizio
con $b_{n+1} = -\sum_i^n b_i$, collegato a tutti i nodi sorgente, attraverso archi di costo 0 e capacità $+\infty$.

\subsubsection{Soluzione del problema con archi di capacità illimitata}
Il problema di flusso a costo minimo consiste nel far giungere il prodotto dai nodi sorgente ai nodi destinazione
minimizzando il costo di trasporto.

Una base di un grafo di flusso è un suo albero di supporto.
Ad ogni base è associata una soluzione di base, ottenuta ponendo a 0 il flusso \footnote{quantità di prodotto inviata} di ogni arco non contenuto in essa.

\begin{figure}[h]
    \centering
    \begin{tikzpicture}[rotate=90, EdgeStyle/.style={->}]
        \Vertices[unit=1.5]{circle}{1, 2, 3, 4, 5}
        \Edge[label=$5$, color=gray](1)(2)
        \Edge[label=$-2$, color=gray](1)(3)
        \Edge[label=$2$, color=red](1)(5)
        \Edge[label=$-4$, color=red](2)(3)
        \Edge[label=$0$, color=red](3)(4)
        \Edge[label=$6$, color=gray](4)(2)
        \Edge[label=$3$, color=red](4)(5)
        \Edge[label=$4$, color=gray](5)(3)
        ;
        \node[draw] at ([shift={(0:.7cm)}]1)   {$b=2$};
        \node[draw] at ([shift={(0:.7cm)}]2)   {$b=5$};
        \node[draw] at ([shift={(180:.7cm)}]3) {$b=1$};
        \node[draw] at ([shift={(180:.7cm)}]4) {$b=-4$};
        \node[draw] at ([shift={(0:.7cm)}]5)   {$b=-4$};
    \end{tikzpicture}
    \caption{$b_i$ sono indicati vicino al nodo}
    \label{img:infty_flux}
\end{figure}
Prendiamo come esempio la base $B_0 = \{(1, 5), (2, 3), (3, 4), (4, 5)\}$ del grafo $G = (V, A)$
rappresentata in figura \ref{img:infty_flux}.
Ponendo nullo il flusso degli archi che non appartenenti a $B_0$, otteniamo che i flussi della soluzione di base sono:

\begin{table}[h]
    \centering
    \setlength{\tabcolsep}{2em}
    \begin{tabular}{cc}
        (1, 5) $\to$ 2 & (2, 3) $\to$ 5\\
        (3, 4) $\to$ 6 & (4, 5) $\to$ 2
    \end{tabular}
\end{table}
Se tutti i flussi degli archi in base è non negativo, allora la soluzione si dice ammissibile.
Inoltre se i flussi sono tutti strettamente maggiori di 0, allora si parla di soluzione non degenere.

Il costo totale del trasporto si calcola sommando per ogni arco della soluzione il prodotto tra quantità di flusso e
costo di trasporto.
Nel caso di esempio:
\[
    c_{15} x_{15} + c_{23} x_{23} + c_{34} x_{34} + c_{45} x_{45} =
    2 \cdot 2 - 4 \cdot 5 + 0 \cdot 6 + 3 \cdot 2 = 10
\]
Ovviamente questa rimane solo una soluzione ammissibile del problema, per avere conferma che sia anche ottimale, occorre
confrontarla con altre soluzioni ammissibili.

\subsection{Coefficienti di costo ridotto}
I coefficienti a costo ridotto sono valori numerici associati agli archi che non fanno parte della
base, misurano la variazione del costo di trasporto al crescere dell'unità del valore del flusso
associato tale arco.

\subsubsection{Condizione di ottimalità}
Se i coefficienti di costo ridotto di tutti gli archi fuori base sono non negativi, questo indica che
la crescita del flusso su qualsiasi arco fuori base comporta una crescita del costo di trasporto o
nessuna variazione.
\\
In tal caso possiamo concludere che la soluzione di base attuale è ottima.
Inoltre se tutti i coefficienti a costo ridotto sono strettamente positivi, la soluzione è unica.

\subsubsection{Calcolo dei coefficienti a costo ridotto}
Per calcolare il coefficiente relativo ad un arco fuori base, si aggiunge tale arco alla base attuale,
considerare l'unico ciclo che si forma con tale aggiunta.
Percorrendo il ciclo che si forma nel verso indicato dall'arco, sommo tutti i costi relativi agli archi
percorsi in senso concorde e sottraggo tutti i costi relativi agli archi percorsi in senso opposto.

\subsubsection{Condizione di illimitatezza}
Insieme alla condizione di ottimalità, esiste una seconda condizione d'arresto per l'algoritmo,
che si verifica, quando, l'aggiunta di un arco (alla base) con coefficiente di costo ridotto negativo
forma un ciclo orientato.
\\
La motivazione è facilmente intuibile, siccome il costo di trasporto diminuisce indefinitivamente.

\subsubsection{Dimostrazione}
Dato un flusso ammissibile $\{\bar{x}_{ij}\}$ corrispondente ad una certa base e con valore dell'obbiettivo
(costo di trasporto) $C$.
\\
Sia $\bar{c}_{rs} < 0$ il coefficiente di costo ridotto dell'arco fuori base $(r, s)$.
\\
Sia $r \to s \to l_1 \to \dots l_t \to r$ il ciclo orientato creato aggiungendo alla base l'arco $(r, s)$

Per ogni $\Delta \ge 0$ il seguente aggiornamento del flusso lungo gli archi del ciclo orientato dà origine
ad un flusso ancora ammissibile, con obbiettivo di corrispondenza:
\[
    C + \Delta c_{rs} \to -\infty \quad \text{per}\,\, \Delta \to +\infty
\]
Il che mostra come l'obbiettivo diverga a $-\infty$ sulla regione ammissibile.
\subsection{Cambio di Base}
Nel caso in cui la base scelta non rispetti ne la condizione di ottimalità ne la condizione di illimitatezza
esiste un algoritmo per cambiare la base attuale in una ammissibile.

Scelgo l'arco fuori base con coefficiente di costo ridotto negativo \footnote{Ne esiste almeno uno altrimenti
sarebbe stata soddisfatta la condizione di ottimalità} attraverso un approccio greedy, prendendo quello con
coefficiente di costo ridotto minore, e chiamo tale coefficiente $\bar{c}$.
Aggiungendo tale arco alla base, formando un ciclo.
Tra tutti gli archi percorsi in verso opposto, rimuovo quello con quantità di prodotto su arco minore, e chiamo
tale quantità $\Delta$. Successivamente, assegno all'arco appena aggiunto una quantità di prodotto inviata
pari a $\Delta$ ed aggiorno il flusso degli archi rimanenti dell'ex-ciclo.

Il nuovo costo è il costo precedente,
Chiamato $T$ il costo precedente, avremo che il nuovo costo sarà dato dalla formula $T + \bar{c} \cdot \Delta$.

\subsection{Algoritmo del simplesso su rete} % Rewrite this subsection better
Dopo aver trovato una base, ripete iterativamente
\begin{enumerate}
    \item Condizione di illimitatezza
    \item condizione di ottimalità
    \item cambio di base
\end{enumerate}
Si può notare che nel caso degenere può cambiare la base, tenendo costante il trasporto e la soluzione di base.
\subsubsection{Problema di 1\textsuperscript{a} fase}
%58
Se il valore ottimo, risultato dalla 1\textsuperscript{a} fase, è maggiore di 0, allora il problema originario ha regione
ammissibile vuota (non ha soluzione).
Se il valore ottimo è uguale a 0, il problema originario ha regione ammissibile non vuota (ha soluzione).
Questo è dovuto al fatto che tutti gli archi di collegamento a $q$, hanno costo 1, mentre quelli del
grafo originario hanno costo nullo. Un valore ottimo nullo, indica che esiste una soluzione all'interno del
grafo originario.

Inoltre, in tal caso esiste un albero di supporto ottimo che contiene solo uno dei nuovi archi (incidenti su $q$).
Eliminando tale arco si ottiene una base ammissibile per il problema originario.
% \lstinputlisting{algorithms/unlimited_symplex.py}

\subsection{Algoritmo del simplesso con capacità limitate}
\begin{lstlisting}
def symplex(edges):
    base_solution = find_base_solution(edges)

    while not optimal_solution(base_solution):
        base_solution = change_base(base_solution)

    return base_solution
\end{lstlisting}
Gli archi sono suddivisi in tre tipi: gli archi in base $B$, archi fuori base a valore nullo $N_0$ ed archi fuori base saturi
(con flusso pari al proprio limite superiore) $N_1$.
\\
Una soluzione di base è definita ammissibile se tutti gli archi in base hanno flusso compreso tra 0 e la propria capacità massima.
Inoltre la soluzione non è degenere se tutti gli archi hanno flusso strettamente compreso tra 0 e la capacità dell'arco.
\\
Partendo da una tripla $(B, N_0, N_1)$, per trovare la soluzione di base associata seguo i passaggi:
\begin{enumerate}
    \item Inizializzazione
        \begin{enumerate}
            \item Pongo a 0 il flusso lungo tutti gli archi in $N_0$
            \item Pongo pongo saturi tutti gli archi $\in N_1$
        \end{enumerate}
    \item Determinare la quantità di prodotto inviata
        \begin{enumerate}
            \item Partendo dai nodi foglia dell'albero di supporto, invio la massima quantità di prodotto inviabile
            \item DFS per calcolare il prodotto inviato su ogni arco
        \end{enumerate}
    \item Verifico che la tripla in input sia una soluzione ammissibile
\end{enumerate}

La tripla ammissibile viene trovata attraverso il metodo a due fasi:
\\
Introduco un nodo aggiuntivo $q$, lo collego ad ogni sorgente e destinazione come nei casi recedenti.

\subsubsection{Condizione di Ottimalità}
Per gli archi in $N_1$ l'unica cosa possibile da fare è far decrescere la quantità di prodotto inviata, per questo voglio
calcolare quanto varia il costo totale facendo \textbf{diminuire} la quantità di prodotto inviata sull'arco.

Per gli archi in $N_0$, il coefficiente di costo ridotto è non negativo

Per gli archi in $N_1$, il coefficiente di costo ridotto deve essere non positivo.
\[
    \bar{c}_{ij} \ge 0 \quad \forall (i, j) \in N_0  \qquad \bar{c}_{ij} \le 0 \quad \forall (i, j) \in N_1
\]
Se le disuguaglianze sono strette la soluzione ottima è unica.
\subsubsection{Condizione di Illimitatezza}
Se l'aggiunta alla base attuale di un arco in $N_0$, con coefficiente di costo ridotto negativo crea un ciclo orientato
e \textbf{tutti} gli archi del ciclo hanno capacità pari a $+\infty$, allora il problema del flusso a costo minimo ha obbiettivo
illimitato.

\subsubsection{Cambio di base}
Attraverso sempre un approccio greedy, l'arco entrante in base è
\[
    (i, j) \in \arg\max\big\{ \max_{(i, j) \in N_0} -\bar{c}_{ij}, \max_{(i, j) \in N_1} \bar{c}_{ij}\big\}
\]
Per trovare l'arco uscente, aumento la quantità di prodotto $\Delta$ se l'arco aggiunto era in $N_0$, altrimenti diminuiamo
$\Delta$ inviata lungo il ciclo in caso di arco in $N_1$.
Al crescere di $\Delta$:
\begin{itemize}
    \item Arco si azzera, esce dalla base ed entra in $N_0$.
    \item Arco si satura, esce dalla base ed entra in $N_1$.
    \item L'arco fuori base si azzera/satura, la base non cambia, ma l'arco cambia di collocazione (da $N_0$ ad $N_1$ e viceversa)
\end{itemize}
Una volta effettuato il cambio di base, si applica il controllo delle condizioni sulla nuova tripla ammissibile.
Viene ripetuto questo procedimento fino a quando le condizioni sono soddisfatte.
\end{document}
