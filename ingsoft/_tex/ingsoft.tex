\documentclass{article}
\usepackage{geometry}[margin=1in]

\title{Software Engineering}
\author{ale-cci}

\begin{document}
\maketitle
\section{Software Development Process}

\subsection{Waterfall Model}


\begin{itemize}
    \item \textbf{Requirement definition}: the potential requirements of the application are analyzed and written down in a specific document.
        The result is a \textit{requirements document} that defines \textbf{what} the application should do (\textbf{not how} it should do it).

    \item \textbf{Analysis}: the system is analyzed in order to determine what model and business logic should be used in the application.

    \item \textbf{Design}: technical design requirements (i.e. programming language, services, \ldots).
        Usually a design specification that outlines how the business logic covered in \textbf{analysis} will be generated.

    \item \textbf{Coding}: Source code implementation of systems specified in the prior stages.
    \item \textbf{Testing}: Beta testers discover and report issues found in the system. It's not uncommon to cause a \textit{necessary repeat} of the \textbf{coding} phase.
    \item \textbf{Operations}: Deployment and maintenance of the application.
\end{itemize}


\end{document}
