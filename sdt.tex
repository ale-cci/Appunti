\documentclass{article}

\usepackage{graphicx}
\usepackage{wrapfig}
\usepackage{circuitikz}
\usepackage{titling}
\usepackage{fancyhdr}
\usepackage{mathtools}
\usepackage{scalerel}
\usepackage{amssymb}
\usepackage{stackengine}
\usepackage[parfill]{parskip}
\usepackage[margin=1in]{geometry}
\usepackage[makeroom]{cancel}

\date{2\textsuperscript{o} Semestre 2018/19}
\title{Sistemi di Telecomunicazione}
\author{ale-cci }

\renewcommand\maketitlehooka{\null\mbox{}\vfill}
\renewcommand\maketitlehookd{\vfill\null}

\pagestyle{fancy}

\def\myupbracefill#1{\rotatebox{90}{\stretchto{\{}{#1}}}
\def\rlwd{.5pt}
\newcommand\notate[4][B]{%
  \if B#1\else\def\myupbracefill##1{}\fi%
  \def\useanchorwidth{T}%
  \setbox0=\hbox{$\displaystyle#2$}%
  \def\stackalignment{c}\stackunder[-6pt]{%
    \def\stackalignment{c}\stackunder[-1.5pt]{%
      \stackunder[2pt]{\strut $\displaystyle#2$}{\myupbracefill{\wd0}}}{%
    \rule{\rlwd}{#3\baselineskip}}}{%
  \strut\kern9pt$\rightarrow$\smash{\rlap{$~\displaystyle#4$}}}%
}

\DeclareMathOperator*{\argmax}{argmax}
\DeclareMathOperator*{\argmin}{argmin}
\DeclareMathOperator\erf{erf}
\DeclareMathOperator\snr{SNR}
\DeclareMathOperator\erfc{erfc}


\begin{document}
\begin{titlingpage}
\maketitle
\end{titlingpage}

\newpage
%% -- Begin --

\section{Guadagno in potenza}

\[
    P_{out} = gP_{in}  \xrightarrow[]{10\log_{10}{}} P_{out_{db}} = g_{db} + P_{in_{db}}
\]

\begin{itemize}
    \item $g < 1$: Perdita di Trasmisione

    \item $L$: Transmission Loss $\coloneqq \quad g^{-1}=\frac{P_{in}}{P_{out}}$

\end{itemize}

\subsection{Attenuazione di tipo Radio}
Sistemi di telecomunicazione su cao (Dal nome, Segnale si propaga lungo un cavo)

\begin{itemize}
    \item $l$: lunghezza del cavo
    \item $\alpha$: coefficente d'attenuazione $\frac{dB}{Km}$
\end{itemize}

\[
    \notate{P_{out}}{2}{\text{Da fisica}} = \notate{10^{-\frac{\alpha}{10}l} P_{in} }{1}{\text{$\cdot < 1 $ Siccome $\alpha$ e $l$ sono positivi}}
\]

$g = 10^{+\frac{\alpha}{10}l}$

$L=10^{-\frac{\alpha}{10} l}$

$L_{dB} = 10\log_{10} L = \alpha \cdot l$


\begin{center}
    \begin{tabular}{||c|c|c||}
        \hline
        Esempi & Frequenza & \( \alpha \)     \\
        \hline \hline
        Doppino Telefonico & \( 10 kHz \) & 2 \\
                           & \( 100 kHz\) & 3 \\
                           & \( 300 kHz\) & 6 \\
        \hline
        Cavo Coassiale     & \( 100 kHz\) & 1 \\
                           & \( 1 MHz\) & 2 \\
                           & \( 3 MHz\) & 4 \\
        \hline
        Guida d'onda rettangolare     & \( 10 GHz\) & 5 \\

        \hline
        Cavo in fibra Ottica & \( 4\cdot 10^{14} Hz\) & 10 \\

        \hline
    \end{tabular}
\end{center}

Per ``combattere'' l'attenuazione vengono usati i ripetitori

\textbf{Trasmissione Radio}: Perdita di potenza dovuta all'irradiazione stessa

\begin{itemize}
    \item $l$: distanza
    \item $\lambda$: Lunghezza d'onda
    \item $\alpha$: path loss exponent
\end{itemize}


$ L = {(\frac{4\pi l}{\lambda})}^2 $


\framebox{$ C = \lambda f_c $} \quad
$f_c$ frequenza portante

$ L = {(\frac{4\pi l}{\lambda})}^\alpha $

$f_c$ \'e solamente espressa in $GHz$, quindi si ha che:

\[
    L_{dB} = 20\log_{10} {\left(\frac{4\pi}{\lambda}l\right)}^2 + 20\log_{10} 10^9 + 20\log_{10} f_{c_{GHz}} = 92.4 + 20\log_{10} l + 20\log_{10} f_{c_{GHz}}
\]

\subsection{Tramissioni Cablate}
$L_{dB} = \alpha l$

\textbf{Antenna Radio}: Soino dette \textbf{direttive}, se concentrano la potenza su un unica direzione

Piu l'antenna \'e direttiva, pi\'u \'e alto il guadagno

\underline{Esercizio}

\subsection{Formule di Friis}

\[
    P_{out} = P_{in} \frac{g_T g_R}{{(\frac{4 \pi d f_c} {C})}^\nu} \quad \rightarrow  \quad K{\left(\frac{1}{d}\right)}^\nu
\]

% diagram

\begin{itemize}
    \item \textbf{Shadowing}: Fluttuazione dovuta a cambiamento dell'ambiente
    \item \textbf{Short Term Fading}: Fluttuazione a breve distanza, distribuita come la \textit{distribuzione di Reileigh}
\end{itemize}

\subsection{Dominio Frequenziale}

$S_y(f) =S_x(f) {\lvert H(f) \rvert}^2$

% grafico 1 %
% grafico 2 %
$B_c$: Banda di Coerenza $\rightarrow$
Intervallo frequenziale in cui la risposta  in frequenza del canale varia di poco

$B_c > B$: Canale \textbf{NON} selettio di frequenza, La risposta del canale ``non'' cambia su $T_s$ (Tempo di fading) lento ($T_s < T_c$)


$B_c < B$: Canale Selettivo in frequenza
fading veloce: risposta cambia su $T_s$ $(T_s > T_c)$

% immagine antenne %

$B_c = \frac{1}{5\sigma_D}$

$T_s$: Tempo di simbolo

$T_c = \frac{6}{16\pi f_D}$

$f_D$: Frequenza Doppler

\newpage
\section{Modello ISO-OSI}

\begin{enumerate}
    \item Phisical
    \item Data-Link
    \item Network
    \item Transport
    \item Session
    \item Presentation
    \item Application
\end{enumerate}

\textbf{Tabelle di Routing} Protocollo IP Riesce a trovare il percorso tra utente ed endpoint

\subsection{Livello 1}
perso

% image %

\subsection{Livello 2}
Trasmette i grame al nodo successivo
\begin{itemize}
    \item Controlla che il lin sia attivo
    \item Fornisce informazione ai livelli superiori
    \item Correzione errore per frame
\end{itemize}

\textbf{MAC}: Medium Access Control

\textbf{LLC}: Logical Link Control $\rightarrow$ controlla che il link sia attivo


\subsection{Livello 3}
Si occupa solo del \textbf{percorso logico} tra due punti, non compe vengono trasferiti i dati

Nasconde i livelli inferiiori ai layer superiori rendendoli hardware-independent

\subsection{Livello 4}
Consegna messaggio tra due processi
% dual split

\subsection{Livello 5}
Abilita, Modifica, Termina sessioni tra applicazioni

Pi\'u connessioni possono essere viste come una singola sessione

distingue i dati che arrivano tra ``application data'' (dati usati dalle applicazioni) e ``session control data''

Usa dati dei layer 3\&4 per monitorare la comunicazione tra applicazioni

Translation for naming services (google.com $\rightarrow$ 8.8.4.4)

\subsection{Livello 6}
Translation, Compression, Decription and Encapsulation of Data, (Es: Html, JPG, Ascii\ldots)

\subsection{Livello 7}
Fornisce servizi di comunicazione alle applicazioni, esempi ne sono: Http e FTP

\section{TCP/IP}
Inizialmente suddiviso in 4 layer:

\begin{itemize}
    \item Host to Network
    \item Internet
    \item Transport
    \item Application
\end{itemize}

Ma se confrontato al modello OSI ne si possono riconoscere 5:

\begin{itemize}
    \item Application
    \item Transport
    \item Network
    \item Data Link
    \item Phisical
\end{itemize}

In una rete TCP/IP vengono usati 4 livelli di indirizzi

\begin{itemize}
    \item Phisical
    \item Logical
    \item Port
    \item Application-Specific
\end{itemize}

\section{Struttura generale di un sistema di comunicazione}
% image

\textbf{Capacit\'a di canale ($C$)}: Massima velocit\'a a cui possono essere trasmessi i dati

\textbf{Data rate ($bps$)}: Dati effettvamente comunicati

\textbf{BandWidth ($B$)}: la grandezza di banda del segnale trasmesso

\textbf{Bit Error Rate ($BER$)}: Frequenza con cui avvengono erriri di trasmissione

\section{Modulazione}
Aggiungere informazioni al segnale portante $x(t)$
\[
    x(t) = A\cos(2\pi ft + \Phi)
\]

$A$: Ampiezza

$f$: Frequenza

$\Phi$: Fase

Carrier signal: $x(t)$ su cui sono state modulate le informazioni
Analog to analog conversion: Needed only if a bandpass is available

\begin{samepage}
    \begin{itemize}
        \item Aplitude modulation
        \item Frequency modulation
        \item Phase modulation
    \end{itemize}
\end{samepage}

\subsection{Amplitude modulation (AM)}
\begin{itemize}
    \item $m(t)$: Information signal
    \item $A_c\cos(2 \pi f_c t)$: carrier
    \item $f_c$: Carrier frequency
\end{itemize}

Total Bandwidth: $2B$

\[S(t) = A_c(1 + K_o m(t)) \cos(2\pi f_c t)\]

\[\quad \Big\Downarrow \mathcal{F} \quad \]

\[S(f) = \frac{A_c}{2} \left[ \delta(f - f_c) + \delta(f+f_c) + K_o M(f - f_c) + K_o M(f+f_c)\right]\]

%immagine

\subsection{Frequency modulation FM}

\begin{minipage}[]{0.5\textwidth}
    \[
    S(t) = A_c \cos(\theta(t))
    \]
    % first image
\end{minipage}
\begin{minipage}[]{0.5\textwidth}
    \[
        m(t) \coloneqq \frac{ d\theta(t)}{dt} = 2\pi f_c + 2\pi K_f m(t)
    \]
    % second image
\end{minipage}

\textbf{$K_f$}: Frequency derivation constant $\frac{Hz}{V}$

\subsection{Phase Modulation PM}
\[
    S(t) = A_c \cos(2\pi f_c t + K_p m(t))
\]

La variazione di fase si manifesta come una variazione istantanea di frequenza, (proporzionale alla derivata di m)

\textit{Total Bandwidth}: $6B$

Le modulazioni lineari occupano meno banda $\rightarrow$ utilizzate per accomodare pi\'u utenti

Se l'informazione da trasmetter non \'e Analogica ma digitale, la modulazione si chiama Keying

\begin{minipage}[t]{0.5\textwidth}
    \textbf{Amplitude shift Keying ASK}
    \begin{itemize}
        \item $f$ \'e costante
        \item low bandwidth
        \item weak against interference
    \end{itemize}
\end{minipage}
\begin{minipage}[t]{0.5\textwidth}
    \textbf{Frequency shift Keying FSK}

    \[
        FSK(t) =
        \begin{cases*}
            \sin(2\pi f_1 t) \\
            \sin(2\pi f_2 t)
        \end{cases*}
    \]

    % image
    \begin{itemize}
        \item More bandwidth required
    \end{itemize}
\end{minipage}

\subsection{Phase Shift Keying (PSK)}
\[
    \text{PSK}(t) =
    \begin{cases}
        \sin (2\pi f t)\\
        \sin (2 \pi f t + \pi)
    \end{cases}
\]
\begin{itemize}
    \item More complex
    \item Strong against interference
\end{itemize}

\subsection{Binary Phase Shift Keying (BPSK)}
\[
    \text{BPSK}(t) =
    \begin{cases}
        \sqrt{\frac{2E_b}{T_b}} \cos (2\pi f_c t + \delta_c)\\
        \sqrt{\frac{2E_b}{T_b}} \cos (2\pi f_c t + \pi \delta_c)
    \end{cases}
    \qquad \text{con} \quad 0 < t < T_b
\]
\begin{itemize}
    \item $T_b$ Tempo di Bit
    \item $E_b$ Energia di Bit
\end{itemize}

\subsection{Quadrature Phase Shift Keying (QPSK)}
La fase della portante assume 5 valori separati di $\frac{\pi}{2}$. Ogni fase corrisponde ad una coppia unica di bit

\[
    S_{\text{QPSK}}(t) = \sqrt{{2E_b}{T_b}} \cos (2\pi f_c t + i\frac{\pi}{2}) \qquad \text{con} \quad i \in 0..3
\]

La larghezza di banda \'e la met\'a della BPSK

Pu\'o essere espressa come 2 BPSK, il bit-error-rate rimane lo stesso ma la Banda utiilzzata raddoppia

(X: Diagramma a costellazione)

\subsection{Multi-? Phase and Amplitude modulation}
(X: 16Qam, 16psk 16apsk )
Pi\'u i punti sono vicini, pi\'u sono sensibili al rumore

\subsection{Capacit\'a di Canale AWGN (Additive White Gaussian Noise)}
\begin{minipage}{0.4\textwidth}
\[
    C = B\log_2(1 + \frac{P_{in}}{P_{out}})
\]
\[ P_N = B \dot N_0 \qquad \rho \coloneqq \frac{R}{B} \]
\[
    \begin{cases}
        C = B\log_2\left( 1 + \frac{P_{in}}{BN_0}\right)\\
        R \le C
    \end{cases}
\]
\[  \frac{R}{B} \le \log_2\left(1 + \frac{E_b \cdot R}{N_0B}\right) \]
\[  \frac{R}{B} \le \log_2\left(1 + \frac{R}{B}\gamma_b\right) \]
\[ \rho \le \log_2 \left( 1 + \rho \gamma_b\right) \]

\framebox{\textbf{Rumore Bianco} \( \sim \mathcal{N}(0, \sigma^2) \rightarrow n(t) = \frac{1}{\sqrt{2\pi\sigma^2}} e^{-\frac{t^2}{2\sigma^2}} \)}
\end{minipage}
\begin{minipage}{0.6\textwidth}
    \begin{itemize}
        \item $P_{in}$: Potenza Segnale in ingresso
        \item $B$: Banda del Canale
        \item $P_N$: Potenza del Rumore Bianco (Gaussiano)
        \item $N_0$ Densit\'a spettrale di potenza del rumore
        \item $\rho$: Efficenza Spettrale
        \item $R$: Data Rate Utilizzato
        \item $\gamma_b$: Signal Noise ratio $\coloneqq \frac{E_b}{N_0}$
    \end{itemize}
\end{minipage}

\subsection{Calcolo Errore su BPSK}
Ricordando che, da definizione:
\[ \erf(x) = \frac{2}{\sqrt{\pi}}\int_0^x e^{-t^2} dt\]
\[ \erfc(x) = 1-\erf(x) = \frac{2}{\sqrt{\pi}}\int_x^{+\infty} e^{-t^2}dt\]
Calcolo l'errore come:

\begin{minipage}{0.4\textwidth}
(Grafico)
\end{minipage}
\begin{minipage}{0.6\textwidth}
\[ P_e = \int^0_{-\infty} \frac{1}{\sqrt{2\pi \sigma^2}} e^{-\frac{(n - \sqrt{E_b})^2}{2\sigma^2}}dn =\]
\[ = \int^{-\sqrt{E_b}}_{-\infty}\frac{1}{\sqrt{2\pi \sigma^2}} e^{-\frac{n^2}{2\sigma^2}}dn =\]
\[ \textit{Per Simmetria} \rightarrow = \int^{+\infty}_{\sqrt{E_b}}  \frac{1}{\sqrt{2\pi \sigma^2}} e^{-\frac{n^2}{2\sigma^2}}dn =\]
\[\framebox{\(\begin{split}
    z = \frac{n}{\sigma}\\
    \sigma dz = dn
    \end{split} \)} \rightarrow =
    \int^{+\infty}_{\frac{\sqrt{E_b}}{\sigma}}  \frac{\cancel{\sigma}}{\sqrt{2\pi \cancel{\sigma^2}}} e^{-\frac{z^2}{z}}dz =\]
\[ = \frac{1}{2} \erfc\left(\sqrt{\frac{E_b}{2\sigma}}\right)\]
\end{minipage}

Per simmetria, stessi calcoli per $-\sqrt{E_b}$, Quindi:
\begin{itemize}
    \item Trasmetto $-\sqrt{E_b}$ ricevo $+\sqrt{E_b}$:
        \[ P_e \big|_{-\sqrt{E_b}} = \frac{1}{2} \erfc\left(\sqrt{\frac{E_b}{2\sigma}}\right)\]
    \item Trasmetto $+\sqrt{E_b}$ ricevo $-\sqrt{E_b}$:
        \[ P_e \big|_{+\sqrt{E_b}} = \frac{1}{2} \erfc\left(\sqrt{\frac{E_b}{2\sigma}}\right)\]
\end{itemize}
Siccome sono due eventi indipendenti:
\[ P_{-E_b} = P_{+E_b} = \frac{1}{2} \]
\[ P_e = P_e \big|_{-E_b} \cdot P_{-E_b} + P_e \big|_{+E_b} \cdot P_{+E_b}= \frac{1}{2} \erfc\left(\sqrt{\frac{E_b}{2\sigma}}\right) \]


\subsection{Determinare Simbolo ricevuto da QPSK}
(x: Grafico simbolo)
Chiamato $r$ il simbolo ricevuto
\[ R = S+N \]
\[ \text{con} \quad S \in {s_1, s_2, s_3, s_4 } \qquad P{S = s_i} = \sqrt{1}{4}\]
\[ N \sim \mathcal{N}_{\mathbb{C}}(0, \sigma^2)\]

Ricevuto $R=r$, il problema diventa trovare il simbolo ($\bar{s}$) che minimizzi la probabilit\'a d'errore

\[ \bar{s} = s_i \quad \text{t.c.} \quad i = \argmax_{j \in 1\ldots 4} P\{ S=s_j \big| R=r\}  = \]
Per la formula di Bayes Mista:
\[= \argmax_{j \in 1\ldots 4} \frac{f_R(r | S=s_j) P\{ S=s_j\}}{f_R(r)} =\]
Dato che gli altri termini non dipendono da $j$:
\[= \argmax_{j \in 1\ldots 4} f_R(r|S=s_j) =\]
$R = S+N \rightarrow R$ \'e gaussiana a media $s_j$
\[ = \argmax_{j \in 1\ldots 4} \frac{1}{\sqrt{2\pi\sigma^2}} e^{-\frac{(r - s_j)^2}{2\sigma^2}} \]
Siccome $\exp$ \'e strettamente crescente:
\[= \argmax_{j \in 1\ldots 4} -(r - s_j)^2 = \framebox{$\argmin_{j \in 1 \ldots 4} | r - s_j |$} \leftarrow\text{Distanza Euclidea}\]

\section{Wired vs Wireless}
Utilizzando gli stessi approcci di una rete cablata, su una rete wireless, si manifestano due principali problemi:
\begin{itemize}
    \item Sul Wireless la Banda \'e molto pi\'u ridotta
    \item Il canale \'e pi\'u soggetto a fading che collision tra frame
\end{itemize}

Soluzioni:
\begin{itemize}
    \item TDMA: Reserves time slots
    \item FDMA: Reserves Frequencies
    \item CDMA: Reserves Expansion Codes
    \item Random Access Techniques
\end{itemize}

\subsection{FDMA}
Frequency Division Multiple Access: Utenti suddivisi in frequenza
\subsubsection{FDD}
Frequency division Duplex: Uplink e Downlink divisi in frequenze

(X: Grafico)

Pi\'u suscettibile ad interferenze in frequenza
\[ N_{\textit{max}} = \frac{B}{B_c K} \]

\subsubsection{TDD}
Uplink e Downlink divisi in intervalli di tempo

(X: Grafico)

\subsection{TDMA}
Time Division Multiple Access: Ad ogni utente \'e assegnato uno slot di tempo

(X: I due grafici)

\subsection{CDMA}
Code Division Multiple Access: Assegna ad ogni utente una sequenza di chip specifica e unica

\underline{Es}: Maximal length sequences $\rightarrow$ Periodiche con periodo $2^m -1$

Chiamato $x(n)$ un segnale (discreto) periodico

\begin{minipage}{0.5\textwidth}
\[ r_{xx}(l) = \sum^{2^m-1}_{n=0}  x(n)\dot x(n -l)\]
\[ r_{xx}(0) = \sum^{n-1}_{n=0} x^2(n) = 2^m \]
\end{minipage}
\begin{minipage}{0.5\textwidth}
    (X: Grafico coi punti)
\end{minipage}

$r_{xx}(1)= 0$ Siccome mediamente il numero di $+1$ \'e uguale al numero di $-1$

\subsection{DSSS Modulation}
Direct Sequence Spread Spectrum

Chiamata $m(t)$ la sequenza di bit da trasmettere

\[S(t) = \sqrt{\frac{2E}{T}} m(t)\cos(2\pi f_c t + \theta) = \sqrt{E}\,m(t)\sqrt{\frac{2}{T}}\cos(2\pi f_c t + \theta)\]

Schematizzato:

(X: Schema modulazione)

\[r(t) = m(t)p(t)\sqrt{\frac{2E}{T}}\cos(2\pi f_c t) + A_{int}(t) \]
\[p(t)r(t) = \cancel{p(t) p(t)}m(t) \sqrt{\frac{2E}{T}}\cos(2\pi f_c t) + p(t)A_{int}(t)
= \sqrt{\frac{2E}{T}}\cos(2\pi f_c t) + p(t)A_{int}(t) \]

Se N utenti vogliono trasmettere informazioni assegno ad ogniuno id essi un codice di spreading \underline{unico}

(Assumendo che tutti gli N codici di spreading siano quasi ortogonali tra di loro: $r_{xy} = 0$ nell'autocorrelazione con $x \ne y$

\( S_N(t) = \sum^N_{i=1} = m_i(t)p_i(t)s(t)\), Per ottenere $m_k(t)$, basta moltiplicare per il corrispettivo codice di spreading $p_k(t)$

\[ p_k(t) m_k(t) + \sum^N_{i \neq k} m_i(t)\cancelto{\approx 0}{p_i(t)p_k(t)}s(t) \]

\subsection{Vantaggi del CDMA}
\begin{itemize}
    \item Mis different users without any specificcoordination
    \item Nodes doesn't need to be syncronized
    \item Strong against noise
\end{itemize}
\subsubsection{Numero utenti supportati dal CDMA}
\begin{minipage}{0.5\textwidth}
    \[ \snr = \frac{P_S}{P_{th} + P_{int}} \]
\end{minipage}
\begin{minipage}{0.5\textwidth}
    \begin{itemize}
        \item $P_S$: Potenza di Segnale
        \item $P_{int}$: Potenza Interferenza
        \item $P_{th}$: Potenza termica = $FKT_0R_b$
            \begin{itemize}
                \item F Noise figure
                \item K Costante di Boltzmann $= 1.38 \cdot 10^{-23}$
                \item $T_0$ Temperatura Ambiente $= 290K$
                \item $R_b$ Bit Rate
            \end{itemize}
    \end{itemize}
\end{minipage}

\[ \snr \approx \frac{P_S}{P_{int}} = \frac{g P_r}{(N+1) P_r} \xrightarrow{N \gg 1} \frac{g}{N} = \frac{B}{R_b N} \]

Se viene richiesta una $\snr$ minima

\begin{minipage}{0.5\textwidth}
\[
    \begin{cases}
        \snr \ge \snr_{min}\\
        \snr = \frac{B}{R_b N}
    \end{cases} \Rightarrow N_{max} = \frac{B}{R_b \snr_{min}}
\]
\end{minipage}
\begin{minipage}{0.5\textwidth}
    \begin{itemize}
        \item $N$: Numero di interferenze
        \item $B$: larghezza della banda $=gR_b$
        \item $P_r$: Potenza ricevuta
        \item $g$: Spreading factor ($> 1$)
    \end{itemize}
\end{minipage}

$N_{max}$ \'e il massimo numero di utenti attivi contemporaneamente, \'e possibile migiorare le prestazioni del CDMA, modificando la disposiizone delle antenne trasmettitrici

\begin{minipage}{0.5\textwidth}
\[ N_{max} = \frac{B}{R_b \snr_{min}} K \qquad \text{con} \quad K = \frac{G_A G_\nu}{H_0}\]
\end{minipage}
\begin{minipage}{0.5\textwidth}
    \begin{itemize}
        \item $H_0$ Interferenza antenne vicine
        \item $G_A$ Sectorization Gain factor
        \item $G_\nu$ Voice activity factor $\approx 2.5$ (pause nella conversazione)
    \end{itemize}
\end{minipage}

\section{Rumore Termico}
Bianco, a media nulla

\begin{minipage}{0.3\textwidth}
    \begin{center}
    \begin{circuitikz}
        \draw (0, 0) to[R, o-o, l=$R$, v=$V$] (0, 2.5);
    \end{circuitikz}
    \end{center}
\end{minipage}
\begin{minipage}{0.2\textwidth}
    \[\bar{V} = 0\]
    \[ V^2 = \frac{2\theta^2 K^2 T^2}{3h}T \]
\end{minipage}
\begin{minipage}{0.5\textwidth}
    \begin{itemize}
        \item $R$: Resistenza
        \item $K$: Costante di Boltzman $1.37 \cdot 10^{-23}$ [$\frac{W_s}{K}$]
        \item $h$: Costante di plank $6.62\cdot10^{-34}$ [$Ws^2$]
        \item $T$: temperatura [$K$]
    \end{itemize}
\end{minipage}

$V$ ha una distribuzione Gaussiana:
\[ \bar{V} = 0 \]
\[ \bar{V^2} = \text{VAR}[V] \]

$S_V(f)$: (Densit\'a spettrale) \'e approssimabile ad una costante fino a valori di $f \approx 10^{12}Hz$

Da un punto di vista ingegneristico:
\[ S_V(f) \approx \frac{KT}{2} = \frac{N_0}{2} \]
Potenza di rumore:
\[ P = \cancel{2}B\frac{N_0}{\cancel{2}} = BN_0 = KT_0B \]

Banda di rumore equivalente su filtro con risposta $H(f)$:
\[ P_{N output}  = \int^{+\infty}_{-\infty} \frac{N_0}{2}|H^2(f)| df\]

$B_{eq} \coloneqq$ Banda di un filtro con risposta in frequenza rettangolare, ed in uscita la stessa potenza di rumore
\[ P_{N out} = B_{eq} \cdot \frac{N_0}{2} \]

Eguagliando le due espressioni otteniamo che:

\[ B_{eq} \cancel{\frac{N_0}{2}} = \int^{+\infty}_{-\infty} \cancel{\frac{N_0}{2}} | H(f) | df \]

Temperatura Equivalente di Rumore: Varie sorgenti di rumore si comportano similmente al rumore termico

%% -- Section --
\newpage
\section{Random Access Protocols}

Tutti i protocolli ad accesso multiplo trasmettono su un unico canale.

Nal caso in cui due o piu nodi trasmettano contemporaneamente avviene una collisione.

Per comunicare, i nodi possono utilizzare solamente il canale

In una rete tra computer, non tutti i nodi trasmettono continuamente, quindi partizionare equamente le risorse del canale renderebbe la rete non del tutto utilizzata

\underline{Idealmente}: se il Broadcast channel ha una bit rate R
\begin{enumerate}
    \item Quando M nodi vogliono trasmettere, lo fanno con bit-rate \(\frac{R}{M}\)
    \item Decentralizzata: Non ci deve essere uno a coordinare le trasmissioni; i nodi non dovrebbero essere sincronizzati
\end{enumerate}

\subsection{Algoritmo ALOHA}

\subsubsection{Aloha Pure}
Pros:
\begin{itemize}
    \item No sincronization needed
\end{itemize}
Procedure:

\begin{enumerate}
    \item When new frame is received transmit it immediately
    \item on collision retry after a random interval
\end{enumerate}


\subsubsection{Aloha Slotted}
\begin{figure}[h]
    \centering
    \includegraphics[width=1.8in]{placeholder.jpg}
    \caption{Slotted aloha}
\end{figure}
Assumptions:
\begin{enumerate}
    \item All frames have the same size
    \item Time is divided in equal slots
    \item Nodes transmit frames only at the beginning of a slot
    \item syncronization is needed
    \item Collision are always detected
\end{enumerate}


Procedure:

\begin{enumerate}
    \item New frame received transmit
    \item while no collisions are detected send next slot
    \item If collision: start transmission in the next slot with probability P until success
\end{enumerate}
\hfill

\newpage
\section{Protocollo Ethernet CSMA}

Se il pacchetto \'e molto lungo rispetto ai tempi di propagazione termina la trasmissione per non intasare la rete.

Con il protocollo ethernet \'e possibile misurare l'energia in un cavo, confrontarla con quella ricevuta per poter identificare facilmente i casi di interferenza

Principali problemi con il protocollo Wifi:
\begin{itemize}
    \item Non \'e possibile utilizzare lo stesso sistema di error-detection siccome la potenza del segnale ricevuto dipende dalla distanza e dalla vicinanza dagli altri utenti
    \item Il segnale ethernet \'e quasi sempre a bus condiviso
\end{itemize}

Struttura frame livello 2:
\begin{itemize}
    \item \textbf{Preamble} 8 byte, 7 con 10101010 1 con 10101011; Utilizzati per sincronizzare il ricevitore con il trasmettitore
    \item \textbf{Destination Address}: Target machine / Broadcast
    \item \textbf{Source Address}: Source machine MAC address
    \item \textbf{Type}: Protocollo di livello 3 (ex: IP)
    \item \textbf{CRC}: Cycling Redundance Check; Controlla solo se si sono verificati errori, non si occupa della correzione
\end{itemize}

\subsection{Ethernet (Standard IEEE 802.3)}

CSMA/CD 1 Persistent

% Lezione 13/05/19 (di vodafone)
\newpage

\section{5G}

\subsection{Advanteges over 4G}

\begin{itemize}
    \item Maggiore disponibilit\'a di banda
    \item Bassa latenza: da 9 a 10 ms; Dovuta alla minore distanza cloud-server\\
        Multiaccess Edge Computing: Place more computing resource closer to the point of data creation\\
        Very useful for AR/VR

    \item Maggior numero di dispositivi connessi per km\textsuperscript{2} (1 mln)
    \item Peak data rate from 1\(Gbs\) to 20\(Gbs\)
    \item Available spectrum from 3\(GHz\) to 30\(GHz\)
    \item Higher data traffic; from 7.2 Exabyte/Month to 50 Exabyte/Month

\end{itemize}


Non acora pienamente utilizzato, verr\'a utilizzato maggiormente con \textit{IOT}


5G will become the backbone of Smart Cities, diriverless car… (IOT, Industria 4.0)

% copiare
\newpage
\section{UMTS}
La terza generazione GSM in europa IS-95 negli Stati uniti, Capirono che la telefonia cellulare era un fenomeno di massa.
Discussero per avere una terza generazione unica.

Alla fine lo standard che doveva prendere piede era IMT-2000. Tutte le entit\'a cercarono di spingere le proprie soluzioni significative rispetto alla 2\textsuperscript{a} generazione $\rightarrow$ dovevano reggere applicazioni multimediali.

Il sistema doveva supportare la commutazione di circuito (telefono) e pacchetto (tipo internet) ed il tasso di trasmissione in continuo aumento $\left(2\frac{Mb}{s}\right)$

\begin{itemize}
    \item $1G$ $\rightarrow$ Vari sistemi sparsi
    \item $2G$ $\rightarrow$ Sistemi cellulari e il GSM e IS-95 anche se ci sono stati altri sistemi come per esempio IRIDIUM (costellazione di satelliti) o global\\
        $\Rightarrow$ Fornire connettivit\'a connnessione di satellite geostazionario anche se la gesione ed i costi erano assurdi
\end{itemize}

Con la 3\textsuperscript{a} Generazione si vuole arrivare ad una famiglia di standard, applicabile in tutti i continenti

\textbf{UMTS}: (Standard Principale) e poi \textbf{MC CDMA} (portanti multiple)

%Una evoluzione tra GSM e UMTS ...

Per coordinare gli sforzi di tante entit\'a \'e stato creato un associazione 3GPP
Era una partnership per uno standard comune, ne facevano parte: ZTSI, ARIB ed ANSI

Alla fine degli anni 90 \'e stato introdotto la rete di accesso terrestre \underline{UTRA}
Interfaccia radio \textbf{WCDMA} $\rightarrow$ Bande pi\'u larghe del CDMA classico e  poteva funzionare sia in modalit\'a TDD sia una modalit\'a FDD

\'E pi\'u facile allocare una banda in uplink e downlink

Con il TDD c'\'e pi\'u flessibilit\'a

% \begin{tabular}{||c|c||}
%     \hline
%     GSM & a\\
%     GPRS & b
%     \hline
% \end{tabular}

\subsection{Tasso di trasmissione}
\begin{itemize}
    \item \textbf{GSM} $\rightarrow$ traffico dati $36 \frac{Kb}{s}$
    \item \textbf{GPRS} $\rightarrow$ Potenziale a livello fisico che arrivava fino a $171.2\frac{Kb}{s}$ (teorico)
    \item \textbf{EDGE} $\rightarrow 553.6\frac{Kb}{s}$
    \item \textbf{UTRA} (a livello teorico) $\rightarrow 1920\frac{Kb}{s}$
\end{itemize}




\subsection{Differenze tra W-CDMA e GSM }

\begin{itemize}
    \item  Maggiore spaziatura tra plrtanti
    \item Maggiore frequenza di controllo

\end{itemize}

\subsection{frequenza di controllo} nel GSM non importa la potenza relativa delle trasmissioni, perche ogniuno ha la sua frequenza $\rightarrow$ non ci sono interferenze

Con la 3\textsuperscript{a} generazione viene introdotto un controllo in potenza


\subsection{Controllo qualit\'a}
legato alla pianificazione delle reti (dipende dal numero di celle per kluster e come vangono distribuite le risorse) Con la 3\textsuperscript{a} generazione nascono gli algoritmi di gestione delle risorse


\subsection{Diversit\'a di frequenza}
Serve per prevenire Fading nel canale;
\begin{itemize}
    \item \textbf{GSM} utilizzava frequency hopping
    \item \textbf{W-CDMA} siccome il canale \'e molto grande la banda non viene completamente distrutta (notch); Vengono sfruttati con ricevitori \textbf{Rake}
\end{itemize}

\subsection{dati e pacchetto}

\begin{itemize}
    \item GSM pu\'o assegnare all'utente pi\'u time slot successivi
    \item con \textbf{$3g$} viene potenziato lo scheduling per la trasmissione dei pacchetti $\rightarrow$ il sistema assegna pi\'u risorse per aumentare la velocit\'a di trasmissione in downlonk
\end{itemize}

% image 1
\begin{figure}[h]
    \centering
    \includegraphics[width=3in]{img/diversi_cammini.jpg}
    \caption{Diversit\'a di cammino}
\end{figure}
\subsection{Diversit\'a di trasmissione in downlink}

Antenne multiple in trasmissione ed antenna unica in recezione $\rightarrow$ diversit\'a di trasmissione

Usando antenne multiple in recezione, aumenta la velocit\'a di canale esponenzialmente; (Con \textbf{5G} verranno utilizzate almeno 2 antenne per device)


% Desegno 115
Per garantire l'ottimizzazione dell'assegnazione delle assegnazzione delle risorse $\rightarrow$ 4G intrpduce il concetto di ``negoziazione'' delle risorse attraverso \textit{Radio Bearer} $\rightarrow$ Canale di trasporto che consente  di negoziare dati (date bearer) o caratteristiche a livello fisico (Signal bearer)

Gli attributi definiti da questi pacchetti di controllo riguardano:

\begin{enumerate}
    \item throughput
    \item Ritardo trasmissivo
    \item Tasso di errore massimo tollerabile
\end{enumerate}

$\Rightarrow$ Tutto questo porta all'introduzione del concetto di classe  di \textbf{QoS} \textit{Quality of Service}

\begin{samepage}
    \subsection{Classi di QoS di UMTS}
    \begin{itemize}
        \item \textbf{Conversazionale}: applicazione principale per voce, videogiochi e videotelefonia;
            \begin{itemize}
                \item deve essere preservata l'interazione temporale tra le informative del flusso
                \item Bassa varianza dei ritardi
                    % Image 2

                    \begin{figure}[h]
                        \centering
                        \includegraphics[width=3in]{img/classe_conversazionale_grafico.jpg}
                        \caption{Classe Conversazionale}
                    \end{figure}
                \item \'e sopportabile la perdita di alcuni frame, ma non il ritardo eccessivo dell'arrivo di ogni frame

            \end{itemize}

        \item \textbf{Streaming}: Streaming multimediale
            \begin{itemize}
                \item no vincoli su ritardi
                \item perservazione temporale tra i frame

                    % Image 3
                    \begin{figure}[h]
                        \centering
                        \includegraphics[width=3in]{img/streaming_grafico.jpg}
                        \caption{Grafico Streaming}
                    \end{figure}
            \end{itemize}

        \item \textbf{Interattiva}: Cadono i requisiti delle entit\'a informative, siccome non \'e piu presente un flusso di dati
            \begin{itemize}
                \item ottenere risposta da entit\'a remota
                \item preservare integrit\'a dei dati
            \end{itemize}

        \item \textbf{Background}: Chiede solo di preservare l'integrit\'a dei dati, (Es: Download delle email in background)

            Hanno la minore priorit\'a rispetto alle altre classi di traffico

    \end{itemize}
\end{samepage}
% Disegno 117
Sistema UMBS \'e di fatto formato da 2 sottoinsiemi (che sono l'evoluzione del BSS e NSS)

\textbf{NSS} $\rightarrow$ \textit{Core Network} (CN) $\rightarrow$ autenticazione utenti connessi, commutazione autenticazione connessione, interconnessione verso reti esterne (di altri provider o internet)

\textbf{BSS} $\rightarrow$ \textit{UMTS} Radio Access Network (\textit{UTRAN})


Nella fase iniziale del $3G$ coesistenza di sistemi di 2\textsuperscript{a} e 3\textsuperscript{a} generazione

Con la 3\textsuperscript{a} generazione si inizia ad affiancare la rete d'accesso con la rete d'accesso \textit{UTRAN}

\textbf{CS}: Circuit switching

\textbf{PS}: Packet Switching

\textbf{RNC}: Radio network Controller $\rightarrow$ novit\'a introdotta dalla 3G, Collegati con interfaccia \textbf{Iur}, idealmente sono tutti connessi tra loro

Idea del 3G:\@ Delegare all'esterno pi\'u responsabilit\'a ed intelligenza, mentre lascia all'interno della core Network la tariffazione

%Disegno 118

\textbf{UTRAN} $\rightarrow$ formata da vari RNS, ogniuno dei quali \'e formato da diversi Node B. Ogni Node B gestisce un certo numero di \notate{\text{celle (da 3 a 6)}}{1}{\text{In ogni cella \'e supportato FDD e TDD}}

\begin{samepage}
\subsection{Interfacce Considerate nello Standard}
\begin{itemize}
    \item \framebox{UU} $\rightarrow$ Interfaccia telefono e Node B, ha il compito di
        \begin{itemize}
            \item trasportare i servizi per l'utente
            \item Gestione della mobilit\'a: Trasporto delle informazioni necessarie
        \end{itemize}
    \item \framebox{IU} $\rightarrow$ Interfacciamento tra \textit{UTRAN} e \textit{Core Network}
        \begin{itemize}
            \item \textbf{Iu-CS} Circuit Switching $\rightarrow$ per traffico vocale
            \item \textbf{Iu-PS} Packet Swtiching $\rightarrow$ per traffico dati
        \end{itemize}
        \item \framebox{Iub} $\rightarrow$ Interfaccia di collegamento fra Node B ed il proprio RNC
        \item \framebox{Iur} $\rightarrow$ Collega RNC appeartenenti a diversi RNS
            \begin{figure}[h]
                \centering
                \includegraphics[width=6in]{img/architettura_uts.jpg}
                \caption{Architettura UTS}
            \end{figure}

\end{itemize}

\end{samepage}

La novit\'a a livello architetturale dell'UMTS \'e la presenza degli RNC nella \textit{UTRAN} $\Rightarrow$ \underline{RNC} gestisce tutte le funzionalit\'a lato utente:
\begin{itemize}
    \item Mobilit\'a
    \item Assegnazione risorse
\end{itemize}

Senza passare da core network

\underline{Node B}
\begin{itemize}
    \item Realizza le trasmissioni radio (modulazione, potenza trasmissiva, gestire il controllo di potenza)
    \item Riceve dal proprio RNC le indicazioni sulle risore da assegnare agli utenti

    \item Effettuare misure sulla potenza e QoS
\end{itemize}


% Drift and Serving RNC
% Slide 255
\textbf{HSPA} High Sped Packet Access

Dispositivi di generazioni successive devono garantire il funzionamento delle generazioni precedenti fino a quando non vengono dismesse

% Lezione 21 - 5 -19
\section{LTE (4G)}

% slide 261
% slide 262

Machine to machine legata ad IOT (dovrebbe rimanere piccola siccome i dati trasmessi sono pochi)

Con LTE le comunicazioni dati iniziano ad essere deflesse sulla comunicazione dati

% slide 263
% slide 264

Obbiettivo \textbf{LTE}: supportare sempre pi\'u traffico dati

3GPP creata per generare un potenziale standard per la 4\textsuperscript{a} generazione

\subsection{Motivation for LTE}
\begin{itemize}
    \item Higher data rates $\Rightarrow$ More spectral efficiency
    \item  Sistema completamente ottimizzato per packet switching non \textit{Commutazione per circuito}
    \item Higher quality service

        Aumentare l'esperienza di always on

        Suddivisione logica  tra \textit{User Plane} e \textit{Control Plane} (Non visibile all
        utente)
    \item Infrastruttura pi\'u economica

        Semplificazione architettura con riduzione elementi in rete
\end{itemize}


\subsection{LTE performance requirements}
\begin{itemize}
    \item Data rate con picchi di $100\frac{Mb}{s}$, massima banda assegnabile ad un utente $20MHz$

        Da 3\textsuperscript{a} Generazione ci si \'e accorti che il traffico \'e asimmetrico, downlink \'e molto pi\'u utilizzato rispetto all' uplink, per questo motivo, l'efficenza spettrale dell'uplink \'e la met\'a rispetto a quella del downlink

    \item Cell range ideale di qualche Km, idealmente si puntava ad avere celle di $30$ e $100Km$ di raggio

    \item Cell Capacity fino a 200 utenti attivi per cella
    \item Mobilit\'a (Sistema ottimizzato per basse mobilit\'a)
    \item Latency
        \begin{itemize}
            \item User Plane (nell'ordine di pochi millisecondi $\approx 5ms$), essenziale per AR \& VR
            \item Control Plane
        \end{itemize}

    \item Improved Spectral efficiency
    \item Broadcasting: Tutte le applicazioni legate al Digital Video Broadcasting, (es Eventi in diretta)
    \item Ottimizzato verso l'IP, per essere direttamente compatibili con Internet
    \item Bande scalabili (Attraverso \textbf{OFDMA} consente assegnazioni flessibili di risorse frequenziali)
    \item Co-esistenza con precedenti versioni delle reti cellulari
\end{itemize}

\subsection{3 principali limitazioni del 3G}
\begin{enumerate}
    \item Massima bit rate  % perso
    \item 3\textsuperscript{a} gen non era stata progettata per tenere conto dei vincoli sulla latenza,

        diventa difficile utilizzare applicazioni interattive

        Interazioni ancora peggiori con il resource assignment
    \item 3G richiedeva dei terminali complessi, per funzionare bene c'era bisogno utilizzare dei ricevitori RAKE (molto complessi), i quali richiedevano un grande consumo di batteria
\end{enumerate}

Latenza $\coloneqq$ Round trip delay

Struttura 4G: Core Network + parte di controllo; Scompare l'RNC $\rightarrow$ Prima semplificazione rispetto alla 3\textsuperscript{a} generazione

\subsection{LTE/SAE Key features}
SAE $\coloneqq$ System Artchitecture Evolution

\textbf{E-UTRAN}: Evolved UTRAN, parte d'accesso

\textbf{eNB}: Evolved Node B, parlano direttamente con un Gateway

Esistono due tipologie di Gateway

\textbf{P-GW}: Packed Data Network Gateway

\textbf{EPC}: Evolved Packet Core

Architettura semplice dal punto di vista dell'user plane: 3 passi perraggiungere l'esterno

A livello di controllo gli eNB si collegano all'MME \textit{Mobility management entity}

HLR (Home location register) e VLR vengono integrati nell'HSS

PCRF, si occupa della tariffazione

% image 1
\begin{figure}[h]
    \centering
    \includegraphics[width=1.8in]{placeholder.jpg}
    \caption{Architettura LTE}
    \label{fig:lte_architecture}
\end{figure}


Si pu\'o notare la sudivisione tra piu livelli nell'architettura dell'LTE (Figura~\ref{fig:lte_architecture})

LTE in uplink viene utilizzato SC-FDMA, in downlink OFDMA

HARQ: Garantisce ritrasmissione selettiva di pacchetti persi

Da punto di vista dell'accesso, si concentra tutto sugli eNB

Ruoi delegati all'eNB
\begin{itemize}
    \item Resource Scheduling
    \item QoS Aware
    \item Autoconfigurante, idea: installare eNB e lui doveva autoconfigurarsi rispetto alla sua posizione nella rete
\end{itemize}

\begin{itemize}
    \item Scompare completamente il CS domain, viene tutto orientato al Packet Switching
    \item IETF: Internet Engeneering Task Force, si occupa di standardizzare qualsiasi protocollo legato ad internet,

        fondalmente libera, per proporre qualcosa di nuovo vengono richiesti degli RFC (Request For Comments)

        Si occupano di sviluppare i protocolli, non di come vengono trasmessi i dati

        Entra lo standard (MIP-v6, i.e. IPv6)

        4G considerava gi\'a di supportare protocolli per ``IP in mobilit\'a''

    \item  Prepared for Non-3GPP Access

        ePC deve essere in grado di gestire flussi informativi non necessariamente legati a rete cellulare,
        ma anche da access point WiFi
\end{itemize}

\subsection{Key Features}

\begin{itemize}
    \item al posto di NodeB arriva eNB
    \item Scompare RNC
        Delegando le capacit\'a gestionali all'eNB, il sistema diventa pi\'u semplice e reattivo
    \item Transport Layer (trasporto informazione): tutto diventa orientato all'IP

    \item Resource Scheduling in Uplink e Downlink

        eNB Contiene uno scheduler che assegna le risorse a tutti gli utenti connessi
\end{itemize}

Idea fondamentale per rendere possibile assegnazione dinamica delle risorse

\textbf{Sistemi multiportante}
% slide 282
Con CDMA veniva assegnata tutta la banda e si sperava che i notch fossero pochi rispetto alla grandezza di banda,

$\Rightarrow$ approccio pi\'u efficente, dividere banda in pi\'u bande piccole ed assegnarne un diverso numero ad un unico utente:
Vengono assegnate solo quelle dove il canale ha una `buon' risposta, in modo da utilizzare al massimo possibile ogni porzinoe di banda

\textbf{OFDM}: Orthogonal Frequency Division Moldulation

\textbf{HARQ} (H, H+):  Hybrid Authomatic Retransmission reQuest
Cercare di accelerare il pi\'u possibile la trasmissione, non interrompendo il flusso

\subsection{LTE key parameters}

Ogni risorsa frequenziale prende il nome di \textbf{Resource Block}, e corrsiponde ad una sottobanda di $180KHz$.
Vengono assegnati tra i 6 ed i 100 blocchi di risorse in base al tipo di richiesta

Protocolli di accesso multiplo e modelli di scheduling sono identici,

\textbf{OFDMA} legato all'OFDM, Include anche il livello fisico

In Uplink, anche se ogniuno ha una singola antenna, nel caso di Cooperazione viene creata una \textit{Antenna Array} $\rightarrow$ Multiuser Collaborative MIMO (Mai implementata)

Uplink max: $75Mbps$

\end{document}

