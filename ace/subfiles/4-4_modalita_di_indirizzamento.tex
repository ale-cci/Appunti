% Lezione 4.4 Modalità di indirizzamento, continuo di lezione precedente
\documentclass[../ace.tex]{subfiles}

\begin{document}
\section{Modalità di Indirizzamento}
\subsection{Formato di Istruzione}
Per definire un' istruzione all'interno del linguaggio, è necessario definire:
il \textbf{codice operativo} (numero operandi espliciti), gli operandi ed il risultato
e l'indirizzo della prossima istruzione.
Per salvare tutte queste caratteristiche in memoria, diventa fondamentale definire
un formato in cui codificare e decodificare l'istruzione.

\subsection{Modalità di indirizzamento}
La modalità di indirizzamento decide come indicare l'indirizzo in memoria in
cui prendere i dati.

Indipendentemente dal tipo di memoria utilizzata (lineare, segmentata...) e dal tipo di
operazione da effettuare, per indicare all'istruzione richiesta dove trovare l' operando,
posso:
\begin{itemize}
    \item passarlo attraverso un registro
    \item passarne il valore all'istruzione (modalità immediata)
    \item leggerlo dalla memoria
\end{itemize}

Quello che cambia tra le varie ISA sono quante e quanto complesse sono le operazioni per
l' accesso alla memoria.
Più modalità di indirizzamento ho più diventa facile l'accesso, ma allo stesso tempo aumenta
la complessità della rete logica.

Esistono diverse opzioni per quanto riguarda alla lettura dell' operando dalla memoria.
In caso di accesso \textbf{diretto} viene indicato l'indirizzo a cui prendere il dato in memoria.
Diversamente se viene utilizzata una modalità di indirizzamento \textbf{indiretta} viene indicato
l'indirizzo di memoria dell' operando in un registro.
\\
Entrambi gli approcci diretto ed indiretto possono combinarsi nella modalità di indirizzamento
\textbf{base} dove il registro base utilizzato come offset è unito ad un indirizzo diretto per trovare
la posizione in memoria dell' operando.

Esiste un ulteriore versione non implementata in tutte le ISA chiamato \textbf{indiciato}: indico
due registri e l'indirizzo in memoria dell' operando è la somma dei due registri.
Viene chiamato in questo modo perché il primo registro funziona da registro base, mentre il
secondo si comporta da indice (es. Accesso ai dati di un vettore).

Altre tipologie di indirizzamento, (meno frequentemente adottate dalle ISA) sono: L'accesso
\textbf{indiretto} dove viene indicata la cella di memoria contenente l'indirizzo dell' operando,
la modalità di indirizzamento \textbf{scalato} che si comporta esattamente come l' indiciato,
ma viene specificato un ulteriore valore di offset, e per finire le modalità di \textbf{autoincremento} e
\textbf{autodecremento}, le quali funzionano esattamente come la modalità d'accesso tramite registro,
solo che dopo la letture il valore contenuto nel registro viene automaticamente incrementato o
decrementato.

\begin{figure}[!h]
    \centering
    \begin{tabu}{|c|c|}
        \hline
        Tramite registro & \lstinline{mov BL AL} \\
        \hline
        Immediato & \lstinline{mov BL, 12} \\
        \hline
        Diretto & \lstinline{mov AX, [12]} \\
        \hline
        Indiretto tramite registro & \lstinline{mov AX, [BX]}\\
        \hline
        Indiretto tramite indice & \lstinline{mov AX, [SI]}\\
        \hline
    \end{tabu}
    \caption{Rispettive istruzioni Assembly}
\end{figure}
Le modalità di accesso possono essere combinate: \lstinline{mov AX [12 + BX + SI]}.

\warnbox{
    Se nella modalità di indirizzamento è presente il registro \textbf{BP}, il segmento di riferimento
    sarà \textbf{SS} (l'indirizzo è relativo allo stack), altrimenti il riferimento sarà sempre \textbf{DS}
    (segmento dati).
}
Se si vuole comunque accedere ad un altro segmento di memoria, è possibile effettuare un segment override,
specificando il segmento di memoria dove si vuole accedere: \lstinline{mov AX, [CX:BX + 5]}.

\subsection{Modi di indirizzamento nel trasferimento di controllo}
Con questo nome si intende come viene specificato il valore del program counter (o instruction pointer) al momento di
un salto.
\\
Normalmente viene indicato in modo diretto dall'istruzione di jump, ed è esprimibile o in modo assoluto (indicando l'indirizzo completo)
o in modo relativo.

I salti che partono e terminano dallo stesso code segment prendono il nome di intrasegment,
mentre i salti che terminano in un code segment differente prendono il nome di intersegment.

A seconda dei casi si può avere un indirizzamento diretto o indiretto, ovvero viene indicato
direttamente il termine del salto o l'indirizzo di termine è contenuto in un registro.

Nell' ISA Intel è disponibile anche gli indirizzamenti intrasegment ed intersegment, sia in modo diretto che indiretto.


\subsection{Modi di indirizzamento I/O}
Dal punto di vista dell' ISA esistono due metodi di indirizzamento: il \textbf{memory mapped I/O},
dove gli indirizzi per l'interazione con i dispositivi I/O sono contenuti in memoria,
e \textbf{separated I/O}, dove gli spazi di indirizzamento I/O sono separati dalla memoria,
per accedere ai dispositivi ho istruzioni diverse (\lstinline{in} e \lstinline{out}).

Per comunicare con i dispositivi I/O è possibile utilizzare un metodo ad indirizzamento
diretto: \lstinline{in AL, 100}, ma l'indirizzo massimo è limitato a 256.
Se si vuole utilizzare indirizzi con valori maggiori a 256, è necessario utilizzare un metodo
ad indirizzamento tramite registro. Il registro (a 16b) dedicato a queste operazioni è DX.

\subsection{Tipi e struttura degli operandi}
Sono i tipi di dato supportati dall' ISA. Possono essere di tipo intero (signed/unsigned), floating point
(single, double o extended precision), caratteri (ascii/unicode), bool o multimediali.

Ed ovviamente è necessario scegliere quali operazioni sono previste dall' ISA per lavorare con
i tipi di dato supportati.
\end{document}
