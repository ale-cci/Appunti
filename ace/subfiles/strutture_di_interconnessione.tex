\documentclass[../template]{subfiles}

\begin{document}
\section{Strutture di interconnessione}

I bus nascono dalla necessità di comunicazione tra uno o più moduli (es. Modello di Von Neumann)
Internamente al bus, quando un dispositivo trasmette, tutti gli altri collegati riescono a leggere i dati. Non è possibile però che due dispositivi trasmettano in contemporanea.

Utilizzare il bus come unico metodo di comunicazione, è che si comporta come collo di bottiglia.

Le linee di comunicazione sono a singolo bit, e possono essere seriali o parallele.

Caratteristiche del bus di dati sono il tipo: dedicato o generico, l'ampiezza del bus ($n_a$), il metodo di arbitraggio, centralizzato (master-slave) o distribuito (dispositivi rispettano un protocollo di scrittura), temporizzazione (sincrona o asincrona) ed i tipi di trasferimento dati.

Ogni scelta è un compromesso tra prestazione e costo di produzione.

\subsection{Tipi di bus}
Nei bus dedicati sono assegnati tipicamente ad un insieme fisico di componenti, rendendo possibile la separazione trasmissione dati-indirizzi, utilizzabile tipicamente attraverso una linea di controllo aggiuntiva.

I bus multiplexati, necessitano di un controllo per distinguere le informazioni trasmesse (DEN, indirizzo, dati, \ldots).
Vincola ad avere un tempo limitato per la lettura dell'indirizzo, dato che richiede di trasmettere successivamente i dati.

Un bus multiplexato ha costi inferiori, in quanto richiede meno collegamenti ma più logica fisica, riducendo quindi le prestazioni.

\subsubsection{Metodo di arbitraggio}
Può essere centralizzato, dove il processore o un'altro dispositivo dedicato gestisce il controllo del bus, o distribuito, dove attraverso un'algoritmo i dispositivi cooperano per l'accesso.

L'algoritmo utilizzato per la gestione distribuita del bus deve evitare le collisioni.

\subsubsection{Temporizzazione}
Gli eventi possono essere coordinati nel bus in modo sincrono o asincrono. Quando il bus è sincrono, è presente un clock che scandisce tutte le operazioni, campionando i dati sui fronti di salita o discesa.
Sono necessari anche segnali di stabilizzazione, per ridurre i disturbi generati da segnali paralleli.

I pro sono la semplicità di progetto e controllo, il contro è lo spreco di tempo, dovuto alla richiesta di dividere ogni operazione in un numero intero di cicli di clock.

In caso di bus asincrono, ogni operazione è innescata dalla precedente, garantendo una maggior efficenza nell'uso di cicli, ma una maggiore complessità di progetto e controllo.

\subsubsection{Ampiezza del bus}
Aumentare l'ampiezza del bus vuole dire aumentare il numero di bit trasferiti, quindi il bit rate.
Aumentare il bus degli indirizzi implica un aumento nello spazio di indirizzamento: il massimo intervallo di locazioni indirizzabili.

Inoltre aumentare il parallelismo porta una maggiore velocità, tutti i bus moderni sono seriali. Un suo svantaggio è l'aumento della complessità di realizzazione.
I bus seriali permettono di fare linee più lunghe, diminuendo l'interferenza tra le diverse linee.

\subsubsection{Tipi di trasferimento}
Tutti i bus permettono lettura e scrittura, ma esistono anche particolari tipi di funzionamento:
\begin{itemize}
    \item Read modify write: lettura seguita da una scrittura allo stesso indirizzo, in un solo ciclo d'accesso il dato viene letto, modificato e riscritto.
    \item Read after write, operazioni di controllo, indivisibili. Utilizzate nei bus per comunicare con i dispositivi IO, garantendo la corretta scrittura del dato.
    \item Trasferimento a blocchi, leggendo un indirizzo, prosegue la lettura fino ad $n$ indirizzi successivi.
\end{itemize}

\subsubsection{Bus di sistema}
Il bus di sistema connette i principali elementi di un calcolatori: CPU, memorie e IO.
I bus hanno lo stesso problema delle memorie, vorrebbero un'elevata capacità e prestazioni con un basso costo di produzione, ma non è possibile. Per questo esiste una gerarchia di bus interna.
\\
Elencati dal più veloce al più lento:
\begin{itemize}
    \item Processor Bus, utilizzato per comunicare internamente al processore, collegando i diversi registri
    \item Cache Bus, dedicato alla cache interna al processore
    \item System Bus, utilizzato per legare il processore alle memorie
    \item Local IO Bus, bus ad alta velocità utilizzato per collegare le periferiche critiche, come HDD e scheda video
    \item Standard IO Bus, utilizzato per le periferiche più lente.
\end{itemize}

\subsubsection{Bus Unico PDP}
Un unico bus, creato dalla PDP, utilizzato per fare tutte le operazioni di IO. Ha il vantaggio di essere modulare, e facilmente standardizzato.

Lo svantaggio di questo bus, è che tutti i dispositivi sono collegati allo stesso bus, indipendentemente dalla loro velocità.
Per questo la velocità di clock può risultare troppo veloce / lenta.
\\
Inoltre all'aumentare dei dispositivi connessi, la lunghezza del bus aumenta, implicando una lentezza di propagazione dei segnali.
\\
Il bus è anche limitato in ampiezza, dato che tutti i dispositivi connessi hanno lo stesso parallelismo dei dati.

\subsubsection{Bus PCI - Peripheral Component Interconnect}
% 39:44

\end{document}
