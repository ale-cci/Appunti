\documentclass[../template]{subfiles}

\begin{document}
\section{Strutture di interconnessione}

I bus nascono dalla necessità di comunicazione tra uno o più moduli (es. Modello di Von Neumann)
Internamente al bus, quando un dispositivo trasmette, tutti gli altri collegati riescono a leggere i dati. Non è possibile però che due dispositivi trasmettano in contemporanea.

Utilizzare il bus come unico metodo di comunicazione, è che si comporta come collo di bottiglia.

Le linee di comunicazione sono a singolo bit, e possono essere seriali o parallele.

Caratteristiche del bus di dati sono il tipo: dedicato o generico, l'ampiezza del bus ($n_a$), il metodo di arbitraggio, centralizzato (master-slave) o distribuito (dispositivi rispettano un protocollo di scrittura), temporizzazione (sincrona o asincrona) ed i tipi di trasferimento dati.

Ogni scelta è un compromesso tra prestazione e costo di produzione.

\subsection{Tipi di bus}
Nei bus dedicati sono assegnati tipicamente ad un insieme fisico di componenti, rendendo possibile la separazione trasmissione dati-indirizzi, utilizzabile tipicamente attraverso una linea di controllo aggiuntiva.

I bus multiplexati, necessitano di un controllo per distinguere le informazioni trasmesse (DEN, indirizzo, dati, \ldots).
Vincola ad avere un tempo limitato per la lettura dell'indirizzo, dato che richiede di trasmettere successivamente i dati.

Un bus multiplexato ha costi inferiori, in quanto richiede meno collegamenti ma più logica fisica, riducendo quindi le prestazioni.

\subsubsection{Metodo di arbitraggio}
Può essere centralizzato, dove il processore o un'altro dispositivo dedicato gestisce il controllo del bus, o distribuito, dove attraverso un'algoritmo i dispositivi cooperano per l'accesso.

L'algoritmo utilizzato per la gestione distribuita del bus deve evitare le collisioni.

\subsubsection{Temporizzazione}
Gli eventi possono essere coordinati nel bus in modo sincrono o asincrono. Quando il bus è sincrono, è presente un clock che scandisce tutte le operazioni, campionando i dati sui fronti di salita o discesa.
Sono necessari anche segnali di stabilizzazione, per ridurre i disturbi generati da segnali paralleli.

I pro sono la semplicità di progetto e controllo, il contro è lo spreco di tempo, dovuto alla richiesta di dividere ogni operazione in un numero intero di cicli di clock.

In caso di bus asincrono, ogni operazione è innescata dalla precedente, garantendo una maggior efficenza nell'uso di cicli, ma una maggiore complessità di progetto e controllo.

\subsubsection{Ampiezza del bus}
Aumentare l'ampiezza del bus vuole dire aumentare il numero di bit trasferiti, quindi il bit rate.
Aumentare il bus degli indirizzi implica un aumento nello spazio di indirizzamento: il massimo intervallo di locazioni indirizzabili.

Inoltre aumentare il parallelismo porta una maggiore velocità, tutti i bus moderni sono seriali. Un suo svantaggio è l'aumento della complessità di realizzazione.
I bus seriali permettono di fare linee più lunghe, diminuendo l'interferenza tra le diverse linee.

\subsubsection{Tipi di trasferimento}
Tutti i bus permettono lettura e scrittura, ma esistono anche particolari tipi di funzionamento:
\begin{itemize}
    \item Read modify write: lettura seguita da una scrittura allo stesso indirizzo, in un solo ciclo d'accesso il dato viene letto, modificato e riscritto.
    \item Read after write, operazioni di controllo, indivisibili. Utilizzate nei bus per comunicare con i dispositivi IO, garantendo la corretta scrittura del dato.
    \item Trasferimento a blocchi, leggendo un indirizzo, prosegue la lettura fino ad $n$ indirizzi successivi.
\end{itemize}

\subsubsection{Bus di sistema}
Il bus di sistema connette i principali elementi di un calcolatori: CPU, memorie e IO.
I bus hanno lo stesso problema delle memorie, vorrebbero un'elevata capacità e prestazioni con un basso costo di produzione, ma non è possibile. Per questo esiste una gerarchia di bus interna.
\\
Elencati dal più veloce al più lento:
\begin{itemize}
    \item Processor Bus, utilizzato per comunicare internamente al processore, collegando i diversi registri
    \item Cache Bus, dedicato alla cache interna al processore
    \item System Bus, utilizzato per legare il processore alle memorie
    \item Local IO Bus, bus ad alta velocità utilizzato per collegare le periferiche critiche, come HDD e scheda video
    \item Standard IO Bus, utilizzato per le periferiche più lente.
\end{itemize}

\subsubsection{Bus Unico PDP}
Un unico bus, creato dalla PDP, utilizzato per fare tutte le operazioni di IO. Ha il vantaggio di essere modulare, e facilmente standardizzato.

Lo svantaggio di questo bus, è che tutti i dispositivi sono collegati allo stesso bus, indipendentemente dalla loro velocità.
Per questo la velocità di clock può risultare troppo veloce / lenta.
\\
Inoltre all'aumentare dei dispositivi connessi, la lunghezza del bus aumenta, implicando una lentezza di propagazione dei segnali.
\\
Il bus è anche limitato in ampiezza, dato che tutti i dispositivi connessi hanno lo stesso parallelismo dei dati.

\subsubsection{Bus PCI - Peripheral Component Interconnect}
% 39:44
Sviluppato da intel ed introdotto nel 1993 su PC di classe pentium. Il protocollo che utilizza non è legato al tipo di piattaforma su cui è impiegato, ed attraverso un chipset dedicato fornisce funzionalità di plug'n'play e bus mastering.
Lo standard è amministrato dal PCI Special Interest Group.

Non è banale passare dai livelli alti di gerarchia, come bus sistema, ai livelli bassi come local bus o standard ISA/EISA (periferiche); sono necessari dei dispositivi chiamati bridge (PCI bridge, ISA bridge), utilizzati per convertire livelli di tensione, velocità e protocollo di comunicazione i diversi bus.

Le funzionalità principali del PCI sono:
\begin{itemize}
    \item Burst Mode: dopo aver individuato l'indirizzo iniziale c'è un flusso molteplice di dati, trasferiti a burst.

        Il dispositivo che vuole trasferire a burst deve prima ottenere lo stato di "master", diventando initiator. Successivamente inizia il trasferimento di dati verso il dispositivo di destinazione, chiamato "target".

        Ovviamente, un solo dispositivo alla volta può essere master del bus, ed una volta diventato, la trasmissione non può essere interrotta.

    \item Bus Mastering: collegamento peer-to-peer sul bus, con accesso diretto alla memoria centrale. Assicurato tramite un bus arbiter centralizzato, con un dispositivo ad hoc, che funziona da master.
    \item Funzionamento sincrono ed asincrono
    \item Expansions: sul bus PCI c'è la possibilità di inserire più schede di espansione attraverso connettori di tipo "Edge"
    \item Linee Multiplexate: Le stesse linee fisiche sono utilizzate da più linee logiche, diminuendo il numero di linee generali a scapito delle prestazioni.

        Le tipiche schede PCI a 32bit utilizzando circa 50 linee, di cui 32 sono linee dati e indirizzi multiplexate.

        All'iniziare di un ciclo di clock, il PCI mette l'indirizzo, attivando il sengale FRAME\#.
        Nei cicli di clock successivi, inizia il trasferimento dei dati, termina disattivando il segnale FRAME\#.

    \item PCI IDE Bus Mastering: Il controller dei dischi collegato al Bus PCI, garantisce funzionalità DMA (Direct Memory Access)
\end{itemize}

\subsection{Transizione}
Internamente alla transizione, sono presenti anche i segnali IRDY e TRDY, che indicano i segnali di ready dell'iniziato e del target.
Il segnale C/BE\# (comando di bus) indica se sul bus si sta trasferendo indirizzo o dati.
Ed il segnale di DEVSEL\# è utilizzato per la selezione del dispositivo.

\begin{enumerate}
    \item L'initiator richiede lo stato di master all'arbitro centrale, attraverso il segnale \code{REQ\#}.
        Riceve conferma attraverso la linea \code{GNT\#}

    \item Preso il controllo, l'initiator asserische \code{FRAME\#},
        attiva i segnali \code{C/BE\#} durante la address phase per segnalare al sistema il tipo di transizione.

    \item Il target riconosce il proprio indirizzo sulla linea \code{AD}
    \item L' initiator smette di usare \code{AD} e prepare uso bus per target, dichiarando di essere pronto a trasmettere con \code{IRDY\#}

    \item
        Il target riconosce di essere selezionato \code{DEVSEL\#}, e mette a 0 il segnale di \code{TRDY\#}.
        Si ha un trasferimento valido, solo quando sia \code{IRDY\#} e \code{TRDY\#}

    \item
        Quando un dato viene letto, l'abilitazione dei byte viene cambiata (toggle byte enable)
    \item
        L'initiator riabiliterà \code{BE} quando sul bus saranno pronti altri dati da leggere.
    \item Nel caso contrario (lettura) in cui il target pone un terzo blocco, ma l'initiator non è encora pronto per leggerli, l'initiator disattiva il ready.


    \item Al termine l'initiator segnala che è l'ultimo blocco è stato trasferito, annullando \code{FRAME\#}
    \item L'initiator, terminata la conversazione toglie anche il segnale \code{IRDY\#}, tornando alla situazione iniziale.
\end{enumerate}

\subsection{AGP - Accelerated Grarphics Port}
È un interfaccia introdotta da intel, per utilizzare applicazioni di grafica 3D e digital video.
È fisicamente separato dal bus PCI ed offre nuove funzionalità come l'accesso alla memoria di tipo pipeline dedicato. Porta un fantaggio di togliere una quantità di traffico significativa sul bus PCI.

Le schede AGP, dunque, hanno accesso diretto alla memoria centrale, non passando neanche dal bus di sistema, utilizzando un bridge.
Questo serve ad esempio per memorizzare textures.

\subsubsection{Seriale vs Parallelismo}
Il parallelismo aumenta le prestazioni, soprattutto con più dispositivi, ma chiaramente necessita di risolvere conflitti e risulta costoso all'aumentare della lunghezza del bus.

Si è abbandonato il parallelismo per due principali motivi: la mancanza di conflitti non necessita di arbitraggio, (es. collegamento punto a punto), e per il timing skew: per linee lunghe e parallele si forma un disallineamento tra i segnali, rischiando di leggere dati non corretti.
\end{document}
