\documentclass[../template]{subfiles}

\begin{document}
\section{Istruzioni di aritmetica binaria}
\subsection{Operazioni di aritmetica binaria}
\begin{lstlisting}
; Operazioni ad 1 parametro
inc var1        ; Incrementa di 1 var1
dec var1        ; Decrementa di 1 var1

mul sorg    ; Moltiplicazione sorg * al oppure sorg * ax
div sorg    ; Divisione:      sorg / al oppure sorg / ax

imul sorg   ; mul con segno
idiv sorg   ; mul con segno

neg var1    ; nega il registro var1 (negato aritmetico, non binario)


; Operazioni a 2 parametri
add dest, sorg  ; Salvano entrambe il risultato dell'
sub dest, sorg  ; operazione in `dest`

cmp dest, sorg  ; uguale a sub ma non salva il
                ; risultato in dest

adc dest, sorg  ; add with carry: dest = dest + sorg + carry_flag
sbb dest, sorg  ; sub with carry: dest = dest - sorg - carry_flag
\end{lstlisting}
\subsubsection{NOTE}
Nelle operazioni a due parametri, entrambi i registri devono avere
stessa dimensione. Ad esempio \lstinline{add AX, BL} non è permesso.

\vspace{10pt}
\noindent
Nel caso in cui \lstinline{div} sia troppo grande per essere contenuto nel
registro destinazione, o il divisore sia 0, viene generato un \lstinline{int 0h} (Divisione per zero).

\vspace{10pt}
\noindent
Sono supportati i formati \textbf{signed}, \textbf{unsigned}, numeri decimali \textbf{packed} \footnote{Ogni byte contiene due numeri decimali, la cifra più significativa è allocata nei 4 bit superiori. Es: 35=0011.0101} e \textbf{unpacked}
\footnote{Ogni byte contiene un solo numero decimale BCD nei 4 bit inferiori. Es: 35=0000.0011 0000.0101}.
\\
Nel caso di numeri decimali unpacked, i 4 bit superiori devono essere a 0 se il numero è usato in un'operazione di moltiplicazione o divisione.

\subsection{Operazioni su 32 bit}
\newcommand\ax{\lstinline{ax}}
\newcommand\bx{\lstinline{bx}}
\newcommand\dx{\lstinline{dx}}
\newcommand\al{\lstinline{al}}
\newcommand\ah{\lstinline{al}}


Considerando come unico numero a 32bit i registri \bx e \ax (con 16bit più significativi salvati in \ax e
16 meno significativi salvati in \bx), ed un altro numero a 32bit salvato in analogo modo in \lstinline{dx} \lstinline{cx},
somma e sottrazione possono essere eseguite nel seguente modo
\begin{lstlisting}
add ax, cx  ; Somma parti meno significative
adc bx, ds  ; Somma parti più significative con carry

sub ax, cx  ; Analogo per sottrazione
sbb bx, ds
\end{lstlisting}
% 27:42
\newpage
\subsection{Moltiplicazione e Divisione}
Esistono due tipi, quelle che operano con segno (\lstinline{imul} e \lstinline{idiv}), e quelle che operano in
modo unsigned (\lstinline{mul} e \lstinline{div}).
\\
Prendono un solo operando, che può essere un registro generale o una variabile. Il secondo operando viene scelto dinamicamente
in base alla dimensione del primo. Nel caso di moltiplicazione:

\begin{itemize}
    \item se è di tipo \textbf{byte}: 8bit, il secondo è \al, ed il risultato è salvato in \ax
    \item se è di tipo \textbf{word}: 16bit, è \ax ed il risultato è messo in \dx:\ax
        \footnote{Indico con \ax:\bx, un numero i cui bit più significativi sono salvati in \ax, ed i bit meno significativi sono salvati in \bx.}
    \item se è di tipo \textbf{dword}: 32bit
\end{itemize}
Se viene preso dalla memoria è necessario specificare manualmente la dimensione attraverso le keyword elencate sopra (es: \lstinline{mul word [0100]})

\vspace{10pt}
\noindent
In caso di divisione le operazioni di tipo byte utilizzano \ax come secondo operando, e salvano risultato e resto in \al ed \ah rispettivamente.
Per operazioni di tipo word, \dx:\ax è il secondo operando, resto e risultato sono salvati in \dx e \ax.

\vspace{10pt}
\noindent
Esempio di divisione a 16 bit
\begin{lstlisting}
    mov dx, 0234h
    mov ax, 5678h
    mov cx, 1000h
    div cx

    ; dx = 678
    ; ax = 2345
\end{lstlisting}
Se \dx fosse maggiore di $1000h$, il risultato della divisione risulterebbe a 20byte e non sarebbe possibile salvarlo in \ax. Quindi
genera un interrupt.

\subsubsection{Operazioni su numeri decimali}
Esistono istruzioni che lavorano con i numeri salvati in formato packed ed unpacked, ma non prendono parametri, dato che
lavorano solamente attraverso i registri AL

\begin{itemize}
    \item AAA converte il risultato di una somma in decimale unpacked
    \item AAS converte il risultato di una sottrazione in decimale unpacked
    \item AAM converte il risultato di una moltiplicazione in decimale unpacked
    \item AAD converte il dividendo di una divisione da decimale unpacked a binario
    \item DAA converte il risultato di un addizione in decimale packed
    \item DAS converte il risultato di una sottrazione in decimale packed.
\end{itemize}

\end{document}
