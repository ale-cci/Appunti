\documentclass[../template]{subfiles}

\begin{document}
\section{Trasferimento di controllo}
\subsection{Salti}
Tutti i salti prendono come unico argomento l'indirizzo di destinazione. L'istruzione per
il salto incondizionato (equivalente a goto in C) è \lstinline{jmp}.
Esistono anche i salti condizionati, i quali solitamente sono preceduti da un'istruzione \lstinline{cmp}.

\begin{table}[h]
    \label{asm_jump_table}
    \begin{tabu}{|c|c|}
        \hline
        Instruction & Jump if\\
        \hline
        \hline
        \lstinline{JE} & \zf = 1 \\
        \lstinline{JNE} & \zf = 0\\
        \hline
        \hline
        \lstinline{JA} o \lstinline{JNBE} &  \cf = 0 e \zf = 0\\
        \lstinline{JAE} o \lstinline{JNB} & \cf = 0\\
        \lstinline{JB} o \lstinline{JNAE} & \cf = 1\\
        \lstinline{JBE} o \lstinline{JNA} & \cf = 1 o \zf = 1\\
        \hline
        \hline
        \lstinline{JG} o \lstinline{JNLE} & \zf = 0 e \sf = \of \\
        \lstinline{JGE} o \lstinline{JNL} & \sf = \of \\
        \lstinline{JL} o \lstinline{JNGE} & \sf $\neq$ \of \\
        \lstinline{JLE} o \lstinline{JNG} & \zf = 1 o \sf $\neq$ \of\\
        \hline
        \multicolumn{2}{c}{}\\
        \multicolumn{2}{c}{}\\
        \multicolumn{2}{c}{}\\
        \multicolumn{2}{c}{}
    \end{tabu}
    \begin{tabu}{|c|c|}
        \hline
        \multicolumn{2}{|c|}{Flag}\\
        \hline
        \lstinline{JC} - \lstinline{JNC} & Jump if Carry (Carry flag a 1)\\
        \lstinline{JO} - \lstinline{JNO} & Jump overflow \\
        \lstinline{JS} - \lstinline{JNS}& Jump Sign / Jump Not Sign\\
        \lstinline{JZ} - \lstinline{JNZ}& Jump Zero (alias di \lstinline{JE} e \lstinline{JNE})\\
        \lstinline{JP} o \lstinline{JPE} & Jump Parity (Even). (bit di parità)\\
        \lstinline{JNP} o \lstinline{JPO} & Jump Not Parity, o Jump Parity Odd\\
        \lstinline{JCXZ} & Jump if \cx (registro contatore) Zero. \\
        \hline
        \hline
        \multicolumn{2}{|c|}{Legenda}\\
        \hline
        A & Above\\
        B & Below\\
        G & Greater\\
        L & Less\\
        E & Equal\\
        N & Not\\
        \hline
    \end{tabu}
\end{table}


Esempio di utilizzo di salti condizionati
\begin{lstlisting}
init:   mov ax, 10
        mov bx, 5

check:  cmp ax, bx
        ja halt         ; jump to halt only if ax > bx

        inc ax
        jmp check

halt:   mov  ax, 4c00h
        int 21h
\end{lstlisting}

\subsection{CALL e RET}
Una procedura è una label, la cui chiamata corrisponde ad un salto incondizionato, i parametri sono passati via stack.
La differenza da un normale salto incondizionato è che al momento di una call, è salvato l'instruction pointer nello stack.

Una procedura, nel caso sia all'interno di uno stesso segmento di codice (inter-segment) è detta di tipo \textbf{NEAR}, mentre se può esser
chiamata all'interno di un segmento di codice qualsiasi (intra-segment) è detta di tipo \textbf{FAR}.

Nel momento in cui effettuo una \lstinline{call} di tipo NEAR, l'unica cosa che cambia è l'instruction pointer, dato che non cambia il segment.
Diversamente se effettuo una \lstinline{call} FAR, siccome cambia anche il code segment, viene anch'esso pushato all'interno dello stack.

\begin{lstlisting}
start:      call function

halt:       mov ax, 4c00h
            int 21h

function:   mov ax, 10h
            ret
\end{lstlisting}
\warnbox{{\color{code-keyword}jmp} e {\color{code-keyword}call} hanno la stessa sintassi. Per questo se confuse il compilatore non dà errore.
Se una funzione è invocata con {\color{code-keyword} jmp} l'istruzione {\color{code-keyword}ret} fa comunque il {\color{code-keyword} pop} di un valore dallo stack e cambiato l'instruction pointer.}

\subsection{LOOP}
L'istruzione \lstinline{loop etichetta} o \lstinline{loope etichetta} è equivalente ad effettuare le operazioni:
\begin{lstlisting}
dec  cx
cmp  cx, 0
je  etichetta
\end{lstlisting}
Esistono anche le varianti: \lstinline{loopz} e \lstinline{loopne} che controllano inoltre lo zero flag.
\\
Esempio di utilizzo di \lstinline{loop}:
\begin{lstlisting}
start:  mov ax, 0h
        mov cx, 10h
cycle:  add ax, 10h     ; Eseguita 10h = 16 volte
        loop cycle
\end{lstlisting}

\subsection{INT ed IRET}
Gli Interrupt interrompono l'esecuzione normale del programma.
Possono essere di tipo hardware o invocati via software (es, tramite istruzione \lstinline{int})
Il programma, una volta fermato, passa il controllo ad una procedura di tipo FAR, chiamata RRI (\textit{Inserire acronimo}).
Al termine dell'esecuzione di questa procedura è eseguita l'istruzione \lstinline{iret}.

Dato che il programma è interrotto e deve riprendere la sua normale esecuzione, al momento di un'interrupt
vengono eseguite in ordine le operazioni di:
\begin{itemize}
    \item Salvare nello stack il register flag (\lstinline{pushf})
    \item Trap Flag = 0 (disabilita esecuzione step by step per ragioni di sicurezza), e
        IF = 0 (Interrupt Flag = 0, per evitare l'interruzione di altri interrupt mascherabili).
    \item Salvare nello stack CS e carica CS della RRI
    \item Salvare nello stack IP e carica IP della RRI
\end{itemize}
L'istruzione duale \lstinline{iret}, recupera le istruzioni di IP, CS e register flag precedentemente
salvate nello stack.

Esistono due possibili categorie di interrupt:
\begin{itemize}
    \item Interrupt BIOS, che dal nome agiscono direttamente a livello di BIOS.
        Esempi sono la 10h per l'output su video e la 16h per l'input da tastiera.
    \item Interrupt DOS, che agiscono a livello di sistema operativo.
        Esempio è 21h, utilizzata sempre per I/O da tastiera e terminazione processo.
\end{itemize}
Ogni interrupt ha un elenco di funzioni, ed il registro \ah specifica quale utilizzare.

\begin{center}
\begin{tabu}{|c|c|c|l|}
    \hline
    Code & Function & Description & Info \\
    \hline
    10h & 0Eh &Write chatacter on TTY & AL = Character ASCII code\\
        & & & BH = page number (0 current page) \\
        & & & BL = foreground color (only gui mode)\\
    \hline
    16h & 00h & Keyboard Read & AL = Read ASCII code\\
        &     &               & AH = scan code (specifies input source)\\
    \hline
    21h & 02h & Character Output & DL = ASCII Code\\
    \hline
    21h & 03h & Keyboard Read and echo & AL = Read ASCII code\\
    \hline
    21h & 4Ch & Terminate Process and EXIT & AL = Exit Code\\
    \hline
\end{tabu}
\end{center}

\subsubsection{Esempio di utilizzo interrupt}
\lstinputlisting{interrupt_example.asm}

\end{document}
