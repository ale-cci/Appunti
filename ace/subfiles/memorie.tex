\documentclass[../template]{subfiles}

\begin{document}
\section{Memorie}
\subsection{Distinzione delle memorie}
Una memoria è un'unità logica, dedicata alla memorizzazione dei dati nel calcolatore.
Diversi tipi di memoria, sono classificate in base a 5 parametri:
\begin{itemize}
    \item Capacità: numero delle parole memorizzabili
        Chiamato $L$ il numero di linee, ed $N$ il numero di parole $L =\log_2 N$.
        Ogni parola è un dato ad $M$ bit.
        \\
        Dato che di solito $M = 8$, l'unità di misura la capacità di memoria si misura in byte.
    \item Caratteristiche fisiche:
        Di cui tipo (semiconduttore come RAM, a superficie magnetica HD, ottica DVD), il consumo,
        l'affidabilità misurata come MTBF (\textit{Mean Time Between Failure}), alterabilità (solo lettura o lettura/scrittura) e durevolezza (volatile o non-volatile)
    \item Organizzazione:
        memorie interne al calcolatore sono organizzate in una gerarchia,
        parallelismo ed interlacciamento (come si interfacciano con memorie di livello superiore)
    \item Modalità d'accesso: in quale modi si accede ai dati in memoria,
        \begin{itemize}
            \item Sequenziale: per accedere ad un dato in una posizione fissa è necessario leggere tutti i dati precedenti (cassette mangianastri)
            \item Diretto: è possibile accedere direttamente alla posizione in memoria, in molti casi però il tempo d'accesso è dipendente dalla posizione del dato rispetto alla posizione appena letta o scritta (cd)
            \item Casuale: un accesso diretto con tempo d'accesso costante (RAM, ROM)
            \item Associativo: si accedono agli indirizzi di memoria attraverso delle chiavi hash (tag), per controllare se un dato è presente vengono controllate tutte le locazioni esistenti.
        \end{itemize}

    \item Prestazioni:
        Per misurare le prestazioni di una memoria, esistono diversi parametri, uno di questi, come citato al punto precedente è il tempo d'accesso,  ovvero il tempo impiegato dalla memoria per raggiungere l'indirizzo da quando è fornito.

        Il tempo di ciclo $T_\text{rc}$ è definito come il tempo di accesso più il tempo necessario per terminare l'operazione prima di poter compiere un altro accesso.

        Mentre la velocità di trasferimento (bit rate) è l'inverso del tempo d'accesso, misurando il numero di bit trasferiti.

        Solitamente sono tutti dati imposti dalla CPU, a cui la memoria si adegua o introduce ritardi.

\end{itemize}

Il numero di pin in ingresso del componente della memoria non coincide necessariamente con $n_a$ ed $n_d$ del bus dei dati.
In un processore moderno, normalmente $n_a = 36$ quindi può indirizzare al massimo $64G$ di memoria RAM. Le dimensioni dell'unità di memoria può essere inferiore a $64G$, magari si hanno 4 memorie da $16G$, con $L=34$. $n_a$ rappresenta la massima memoria indirizzabile del processore, $L$ rappresenta la memoria effettiva di un cip specifico di memoria.

Le modalità d'accesso sono


\subsection{Gerarchie di memoria}
La memoria ideale sarebbe di capacità infinita, con tempo d'accesso, costo e consumo nullo.

L'insieme delle memorie presenti in un calcolatore mira a combinare le caratteristiche migliori di ogni tipo di memoria per raggiungere l'obbiettivo di memoria ideale.

Salendo la gerarchia diminuisce il tempo d'accesso e la capacità delle memorie e cresce il costo per bit.

Al livello più alto della gerarchia sono presenti i registri interni alla CPU, segue la cache, poi la memoria centrale o principale, ed in fondo alla gerarchia le memorie secondarie, con alte capacità ma bassi costi e permanenti.
\subsubsection{Principio di località}

Un programma in un certo istante $t$ necessita di determinati dati
Se un programma ad un istante $t$ accede ad un certo dato,
c'è alta probabilità che siano richiesti anche i dati adiacenti (proprietà di località spaziale), e che all'istante successivo si acceda nuovamente alla memoria (proprietà di località temporale).

Gli algoritmi della MMU (\textit{Memory Management Unit}) si concentrano a minimizzare il numero di accessi in memoria a livelli più bassi.


\subsubsection{Blocco di memoria}
Con blocco si fa riferimento all'unità di informazione minima scambiata fra i livelli di memoria.
Quando un particolare dato viene richiesto ad una memoria, l'evento che il dato non sia presente prende il nome di "miss" mentre se il dato è già presente si chiama "hit".
\\
Lo scopo di una buona gerarchia di memoria è massimizzare il rapporto tra hit e miss.
La frequenza di accessi trovati direttamente al livello superiore prende il nome di hit-rate ($h$).
La miss rate è quindi complementare a quest'ultimo.
\\
Il tempo per recuperare il dato al livello successivo in caso di miss prende il nome di "Miss Penalty", mentre se il dato è già disponibile, il tempo per recuperarlo è pari al tempo di hit.

Il miss penalty $T_{mp}$ può essere visto come la somma tra hit time ed il tempo richiesto per il trasferimento dei dati dai blocchi inferiori ($T_{miss}$).
\\
Ottenendo la relazione:
\[
    T_{acc} = h \cdot T_h + (1-h) T_{mp} = T_h + (1 -h) T_{miss}
\]
Dalla formula, minimizzando la miss rate, si tende alla memoria ideale, portando il tempo d'accesso uguale al tempo di hit.


\subsection{Gerarchie di memoria}
Una gerarchia di memoria è definita dai suoi livelli di memoria: il loro numero, dimensione, velocità e componenti.
\\
Il piazzamento del blocco, chiamato anche funzione di traduzione o di mapping, sceglie dove allocare il blocco nel livello di memoria corrente.  Nel caso della cache ad esempio è necessario calcolare un hash per determinarne la locazione.

L'identificazione come del blocco indica come risalire alla posizione del blocco di memoria.

Il rimpiazzamento si occupa di inserire i dati provenienti dai livelli inferiori della gerarchia, scegliendo in quale posizione del livello corrente inserirlo.

I dati, una volta modificati, vengono scritti sui livelli inferiori utilizzando strategie di scrittura.


Ad esempio, nel caso dei registri, l'identificazione è nominale, ed il nome viene indicato nel codice sorgente, mentre identificazione e rimpiazzamento sono definiti dal compilatore.

In memoria centrale, il piazzamento del blocco dipende dall'istruzizone (specificato dall'indirizzo di memoria), i dati sono identificati dall'indirizzo e le scritture sono scielte a livello di codice.

Le memorie centrali nei calcolatori sono le RAM, volatili e di lettura e scrittura). Di memorie RAM ci sono due categorie:
SRAM (\textit{Static RAM}) utilizzate per memorie cache e DRAM (\textit{Dynamic RAM}) utilizzate come memoria centrale.

Le memorie SRAM sono molto veloci e realizzate attraverso flip flop D, che permettono di memorizzare il dato senza la perdita.
La tecnologia richiesta per realizzare uno di questi flip flop è molto costosa, per questo al giorno d'oggi non è possibile utilizzare queste memorie per immagazzinare grandi quantità di dati.

Oltre ai flip flop in queste memorie sono presenti logiche di indrizzo e la selezione di lettura scrittura.

Le memoria DRAM ogni bit è rappresentato da un condensatore / transistor (nel caso di tecnologia mos). Il loro costo è molto inferiore quindi rispetto alle precedenti.
\\
Siccome i condensatori col tempo si scaricano, necessitano di circuiti automatici di refresh per ricaricare il condensatore.

Oltre a memorie RAM sono presenti le memorie ROM, a sola lettura. Esse sono divise in:
le ROM originali erano scrivibili un'unica volta e non sono volatili.
PROM (\textit{Programmable ROM}), scrivibili un'unica volta,
EPROM (\textit{Erasable PROM}), cancellabili attraverso la luce ultravioletta,
EEPROM (\textit{Electrically Erasable PROM}), cancellabili con segnali elettrici.
FLASH leggibili e scrivibili con dati non volatili (come SSD ).
\end{document}
