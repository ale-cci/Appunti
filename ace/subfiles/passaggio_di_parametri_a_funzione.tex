\documentclass[../template]{subfiles}

\begin{document}
\section{Passaggio di parametri con stack}
Per effettuare passaggio di parametri ad una funzione, oltre ad utilizzare registri o variabili in memoria, è possibili utilizzare lo stack.
\begin{lstlisting}
    push ax
    push bx
    call funct
    add sp, 4
\end{lstlisting}
Una volta caricato un parametro nello stack con \lstinline{push} è necessario, una volta invocata la funzione, ritornare il valore dello stack pointer al valore originario, presente prima di \lstinline{push}.
Per effettuare l'operazione viene solitamente utilizzata un \lstinline{add} e non \lstinline{pop} per evitare di "sprecare registri".

Si che viene \textbf{sommato} e non sottratto 2 per ogni numero di parametri passati alla funzione, perché il valore dello stack pointer parte inizialmente dal valore \code{0xFFFF}, crescendo verso \code{0x0000}.

Per recuperare i parametri dallo stack, viene utilizzato il base pointer \lstinline{bp}, dato che \lstinline{sp} può essere soggetto a variazioni all'interno della funzione.
Inoltre, l' operazioni come \lstinline{[bp]} fanno riferimento direttamente allo stack segment, diversamente da quelle come \lstinline{[bx]} che fanno riferimento al data segment.

\subsubsection{Recupero di valori dallo stack}
\begin{lstlisting}
funct:  push bp         ; per tenere salvato il valore di bp
        mov bp, sp

        mov ax, [bp + 6]  ; primo valore pushato 2 + 2*2
        mov bx, [bp + 4]  ; secondo valore pushato

        pop bp
        ret
\end{lstlisting}

\end{document}
