\documentclass[../template]{subfiles}

\begin{document}
\section{Introduzione ai linguaggi Assembly}
Per evitare ai programmatori di ricordarsi a memoria il linguaggio macchina, ogni microprocessore ha un proprio
linguaggio assembly, in grado di tradurre con una corrispondenza biunivoca le istruzioni a basso livello o indirizzi di memoria in codice macchina.

Con statement o pseudo-istruzione si intende una riga del programma assembly. A tale riga corrisponde una direttiva
dell'assemblatore.  Inoltre se a tale direttiva corrisponde un'istruzione in linguaggio macchina, prende il nome di
istruzione.

Ogni istruzione è composta da un'etichetta (label) che rappresenta l'indirizzo di memoria in cui l'istruzione è
memorizzata, un codice operativo (opcode) simbolo mnemonico per l'operazione e da nessuno, uno o più operandi.

\begin{lstlisting}
label:      mov ax, bx
            jmp label
\end{lstlisting}

Le etichette sono sostituite automaticamente dall'assemblatore in indirizzi di memoria. Permettono di astrarre gli
indirizzi fisici, semplificando la modifica e comprensione del programma.

I codici operativi (es. \lstinline{mov}, \lstinline{add} \dots) sono gli alias dati alle istruzioni eseguibili dalla
CPU.

Gli operandi funzionano come argomenti passati ai codici operativi (nel caso di assembly 8086 sono al massimo 2).
Durante l'esecuzione del programma, la CPU provvede a reperire il valore degli operandi, che può essere passato
direttamente all'istruzione per valore, tramite registro, contenuto in memoria o da una porta di I/O.

Le pseudo-istruzioni sono utilizzate durante il processo di assemblaggio: esempi sono i segmenti dati, commenti  e le
macro.
\subsection{Linguaggio assembly 8086}
Specifico per il processore general purpose a 16 bit Intel 8086.
Ha 14 registri interni a 16 bit, 7 modi di indirizzamento con capacità di 1Mb.

\begin{table}[h]
    \centering
    \begin{tabu}{|c|l|c|}
        \hline
        $n$   & parallelismo del processore & 16\\
        \hline
        $n_a$ & parallelismo della memoria & 20\\
        \hline
        $n_d$ & parallelismo del buffer di dati & 16\\
        \hline
    \end{tabu}
    \caption{Parallelismi del processore Intel 8086}
\end{table}

Tutti i processori Intel sono backward-compatible, per questo i concetti di base per questo processore sono gli stessi
utilizzati da un processore moderno.

%TODO: Processore specifico 8086
Il processore 8086 è diviso in due unità funzionali concettualmente separate: la BIU (\textit{Bus Interface Unit}) ed EU
(\textit{Execution Unit}).
Come dice il nome, la BIU si interfaccia con il bus dei dati attraverso un unico bus in ingresso controllato dal
\textit{Bus Control}. L'EU non ha un collegamento diretto con la memoria ed è dedicata ai calcoli.

Siccome il fetch di è in media media più veloce dell'esecuzione, le istruzioni, una volta prelevate dal bus,
vengono salvate temporaneamente in una coda (\textit{coda di prefetch}) in attesa di essere processate dalla EU.

Inoltre questa architettura la BIU ha un sommatore dedicato utilizzato per calcolare il valore dell'IP o l'indirizzo di
accesso per i dati in memoria.

EU Control si occupa di decodificare le istruzioni e genera i segnali necessari al resto dei componenti nella EU:
op-alu, abilitazioni dei registri (temporanei, generali e di flag).
Flag register descrive lo stato dell'ultima operazione effettuata dall'ALU (vedi flag a pag.  \pageref{8086_flags})
\subsection{Scelte progettuali di un ISA}
\begin{itemize}
    \item dove sono memorizzati gli operandi nella CPU
    \item con che istruzioni si accede agli operandi
    \item modello di memoria
    \item formato delle istruzioni
    \item modello di indirizzamento (come indicare gli indirizzi di memoria)
    \item tipo e struttura degli operandi
    \item tipo di istruzioni previste
\end{itemize}

\subsubsection{Dove sono memorizzati gli operandi nella CPU}
Un metodo per memorizzare gli operandi è attraverso lo stack, in questo modo viene garantita un'indipendenza dal
register set, ma con lo svantaggio di avere difficoltà di accesso agli operandi. Inoltre dato che il tempo di accesso allo
stack è elevato rispetto al tempo di esecuzione si forma un bottleneck.

Un altro metodo è l'utilizzo di un unico registro accumulatore, da cui passano tutte le operazioni. La gestione diventa
molto semplice ma è ovvia la formazione di bottleneck.

Il metodo più utilizzato è a set di registri, ovvero gli operandi vengono salvati direttamente nei registri del
processore.
In particolare, nella  CPU 8086, i registri possono essere di tre categorie: general purpose (generici non ortogonali), segment o miscellaneous.
I registri sono \ax, \bx, \cx, \dx, \lstinline{si}, \lstinline{di}, \lstinline{bp}, \lstinline{sp}, tutti a 16 bit. Per
i primi 4 è possibile accedere agli 8 bit più e meno significativi, sostituendo alla \lstinline{x} una \lstinline{h} o
una \lstinline{l} rispettivamente (es. \ah per accedere ai primi 8 del registro \ax).
\begin{table}[h]
    \centering
    \begin{tabu}{|c|l|}
        \hline
        \multicolumn{2}{|c|}{\bfseries Registri generici}\\
        \hline
        \ax & Accumulatore: registro di base per le operazioni\\
        \bx & Base: unico utilizzato per indirizzamenti di memoria\\
        \cx & Conteggio: utilizzato per cicli\\
        \dx & Dati: utilizzato per indirizzi di istruzioni i/o o overflow\\
        \lstinline{si} \lstinline{di}& source  e destination per operazioni con stringhe di byte\\
        \lstinline{bp}& Base Pointer accesso a parametri e variabili di funzioni.\\
        \lstinline{sp}& Stack Pointer: puntatore a top dello stack\\
        \hline
    \end{tabu}
    \begin{tabu}{|c|l|}
        \hline
        \multicolumn{2}{|c|}{\bfseries Registri speciali}\\
        \hline
        IP & Instruction Pointer\\
        FLAG & Register flag\\
        \hline
    \end{tabu}
\end{table}

\begin{table}[ht]
    \centering
    \begin{tabu}{|c|l|}
        \hline
        Overflow flag& Operazione ha un risultato troppo grande\\
        Direction& Indica se incrementare / decrementare per le istruzioni con stringhe\\
        Interrupt Enable& Abilita/Disabilita mascheramento degli interrupt\\
        Trap& Utilizzato dai debugger, genera un int 3 dopo ogni istruzione\\
        Sign& 1 quando il risultato è negativo\\
        Zero& il risultato dell'operazione è 0\\
        Auxiliary Carry& 1 quando è presente un riporto tra la parte alta e la parte bassa del registro.\\
        Parity Flag& 1 quando il numero di bit è pari, 0 quando è dispari\\
        Carry Flag& Riporto o prestito nella parte alta dell'ultimo risultato.\\
        \hline
    \end{tabu}
    \caption{Flag architettura}
    \label{8086_flags}
\end{table}


\end{document}
