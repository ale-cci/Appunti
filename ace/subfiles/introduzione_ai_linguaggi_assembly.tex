\documentclass[../template]{subfiles}

\begin{document}
\section{Introduzione ai linguaggi Assembly}
Per evitare al programmatore di ricordarsi specifiche sequenze di zeri ed uno, ogni microprocessore ha un proprio
linguaggio assembly, in grado di tradurre con una corrispondenza 1:1 istruzioni a basso livello o indirizzi di memoria
in codice macchina.

Con statement o pseudo-istruzione si intende una riga del programma assembly. A tale riga corrisponde una direttiva
dell'assemblatore.  Inoltre se a tale direttiva corrisponde un'istruzione in linguaggio macchina, prende il nome di
istruzione.

Ogni istruzione è composta da un'etichetta (label) che rappresenta l'indirizzo di memoria in cui l'istruzione è
memorizzata, un codice operativo (opcode) simbolo mnemonico per l'operazione e da nessuno, uno o più operandi.

\begin{lstlisting}
label:      mov ax, bx
            jmp label
\end{lstlisting}

Le etichette sono sostituite automaticamente dall'assemblatore in indirizzi di memoria. Permettono di astrarre gli
indirizzi fisici, semplificando la modifica e comprensione del programma.

I codici operativi (es. \lstinline{mov}, \lstinline{add} \dots) sono gli alias dati alle istruzioni eseguibili dalla
CPU.

Gli operandi funzionano come argomenti passati ai codici operativi (nel caso di assembly 8086 sono al massimo 2).
Durante l'esecuzione del programma, la CPU provvede a reperire il valore degli operandi, che può essere passato
direttamente all'istruzione per valore, tramite registro, contenuto in memoria o da una porta di I/O.


Le pseudo-istruzioni sono utilizzate durante il processo di assemblaggio: esempi sono i segmenti dati, commenti  e le
macro.
\subsection{Linguaggio assembly 8086}
Specifico per il processore general purpose a 16 bit Intel 8086.
Ha 14 registri interni a 16 bit, 7 modi di indirizzamento con capacità di 1Mb.

\begin{tabular}{|c|c|c|}
    \hline
$n$   & parallelismo del processore & 16\\
$n_a$ & parallelismo della memoria & 20\\
$n_d$ & parallelismo del buffer di dati & 16\\
\hline
\end{tabular}

\end{document}
