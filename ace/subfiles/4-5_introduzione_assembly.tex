% Lezione 4.5 Introduzione al linguaggio Assembly
% Per questo corso verrà utilizzato NASM come assemblatore e TLINK per linkare il file oggetto in programma binario.
\documentclass[../ace.tex]{subfiles}

\begin{document}
\section{Linguaggio Assembly 8086}
\subsection{In due parole}
Non è case sensitive.
\\
Ogni statement è terminato da \lstinline{\\n}, lo statement può proseguire alla riga successiva solo
se questa comincia con il carattere \lstinline{&}.
\\
Gli identificatori hanno una lunghezza massima di 31 caratteri ed il nome non può iniziare con un numero.

\subsection{Tipi di costante}
\begin{lstlisting}
    mov ax, 13              ; decimale, anche 13D
    mov ax, 13h             ; esadecimale (devono iniziare come un numero)
    mov ax, 00100B          ; binario
    mov ax, 13O             ; ottale

    mov ax, 2.34            ; numeri reali
    mov ax, 112E-3          ; rappresentabili anche in esponenziale

    mov ax, 'T'             ; Costanti carattere
    mov ax, 'test'          ; o anche stringa
\end{lstlisting}

\subsection{Istruzioni per il trasferimento dati}
\begin{lstlisting}
    mov dest, sorg          ; sposta il contenuto del secondo operando
                            ; nel primo
    mov [bx], al            ; salva nell'indirizzo indicato da BX il
                            ; valore di al
    xchg dest, sorg         ; scambia il contenuto dei due operandi
    push w0rd               ; inserisce una word nello stack
    pop  w0rd               ; estrae una word dallo stack
    in   accum, porta       ; legge un dato dalla porta specificata
    out  porta, accum       ; scrive un dato sulla porta specificata
\end{lstlisting}
Si ricorda che istruzioni come \lstinline{mov [bx], [si]} non sono permesse perché siccome non
stiamo utilizzando una macchina \textit{memory-memory}, si può avere al più 1 riferimento alla memoria
nella stessa istruzione.

Esistono altri trasferimenti non ammessi dalla \lstinline{mov}:
\begin{itemize}
    \item \lstinline{mov ds, 100}, modificare direttamente il valore di un registro.
        Occorre utilizzare un registro general purpose:
\begin{lstlisting}
mov ax, 100
mov ds, ax
\end{lstlisting}

    \item \lstinline{mov dx, es}, trasferimento da segment register a segment register.
    \item \lstinline{mov cs, 100}, qualsiasi trasferimento che abbia \cs\ come destinazione.
        Ovvero cambiare il codice in esecuzione.
\end{itemize}
Lo stack pointer \lstinline{sp} parte con valore iniziale 0xffff, ad ogni istruzione \lstinline{push},
\lstinline{sp} diminuisce di 2, mentre ad ogni \lstinline{pop} aumenta di 2.
\warnbox{
    Quando tolgo i dati dallo stack con {\color{code-keyword}pop}, la cella di memoria non viene azzerata.
}

\end{document}
