% Lezione 4.3 - Modelli di Memoria
\documentclass[../ace.tex]{subfiles}

\begin{document}
\section{Modelli di Memoria}
\subsection{Accesso alla memoria}
Per scegliere gli operandi con i quali accedere alla memoria, si dividono le ISA in base a due numeri:
il \textbf{numero di riferimenti diretti in memoria} indicati nelle istruzioni dell ALU
ed il \textbf{numero di operandi indicati in modo esplicito nelle istruzioni}.
Entrambi possono assumere solo valori compresi tra 0 e 3 inclusi.

Ad esempio nelle architetture chiamate \textit{register-register},
il numero di riferimenti diretti in memoria è 0, ed il numero di operandi che indica in modi indicati
in modo esplicito è 3.
In altre parole le uniche operazioni che possono accedere in memoria sono LOAD e STORE.
\\
Il numero di operandi indicati in modo esplicito indica il numero massimo di operandi specificati in
modo esplicito come parametri di una funzione.

Quindi per effettuare un operazione come \lstinline{c = a + b} sono necessarie le operazioni:

\begin{lstlisting}[emph={load,store,add}, emphstyle={\bfseries\color{code-keyword}}]
    load    r1, var1
    load    r2, var2
    add     r1, r2, r3
    store   var3, r3
\end{lstlisting}

Utilizzando lo stesso esempio per una architettura \textit{register-memory}, (1, 2) otteniamo:
\begin{lstlisting}
    mov     AX, var1
    add     AX, var2            ; AX funziona sia da sorgente
                                ; che destinazione
    mov     var3, AX
\end{lstlisting}
In questo caso posso avere al massimo solo un operando che fa riferimento alla memoria.

Un' ultimo esempio di architettura è la \textit{memory-memory} (3, 3) dove sia sorgente che destinazione
sono completamente esplicitati.

\subsection{Ordinamento della memoria}
La memoria è sempre organizzata come un lungo array di celle a 8bit.
Quando un dato di lunghezza più grande di 8bit deve essere salvato in memoria, può essere
utilizzata sia la codifica \textbf{little endian} (memorizza l' Least Significant Bit all'indirizzo
più basso), sia la \textbf{big endian} (all'indirizzo più basso viene salvato il Most Significant
Bit).

\subsection{Allineamento della memoria}
Se la memoria è forzata a salvare i dati in modo allineato, allora riesco a leggere i dati
di grandezza maggiore di 1 byte in un \textbf{singolo ciclo}.
L'unico svantaggio è che per salvare dei dati di grandezza inferiore 4byte, avrò delle
celle inutilizzate.
\\
Se la memoria non è allineata, risparmio più spazio in memoria, ma l'accesso può richiedere
più di un ciclo di CPU.

\subsection{Memoria lineare e segmentata}
Si definisce con \textbf{effective address}, l' indirizzo reale in memoria.
\\
La \textbf{riallocazione della memoria} (RAM) si intende il riordinamento dei blocchi di memoria,
in modo da raggruppare un unico blocco di memoria, tutti i blocchi non utilizzati, isolati
dalla memoria in uso.

Il problema che porta con se la riallocazione di memoria, è che una volta che un blocco di
memoria viene spostato, tutti gli effective address utilizzati nel codice contenuto al suo interno
sono invalidati, e devono essere aggiornati uno ad uno dal processore con il nuovo effective
address.
\\
Per questo motivo alcune ISA preferiscono utilizzare un modello di memoria segmentato,
in cui il codice utilizza \textbf{indirizzi relativi} anziché lavorare direttamente con gli effective address.
Per accedere alla memoria attraverso un indirizzo relativo, vengono salvati in due registri (CS: Code Segment
e DS: Data Segment) l'indirizzo di memoria da cui partono codice e dati del programma.
Al momento di una riallocazione, per un modello di memoria segmentato, l'unica cosa invalidata
sono i due registri segmento di ciascun programma.

\subsection{Modello di memoria Intel 8086}
Nel caso di Intel 8086, la memoria viene vista come un gruppo di paragrafi e segmenti:
i \textbf{paragrafi} sono una zona di memoria a 16bit, i quali non si possono sovrapporre,
mentre un \textbf{segmento} è un unita logica indipendente formata da locazioni continue di memoria, di
dimensione massima 64k, ha inizio ad un indirizzo di memoria
multiplo di 16, (in modo da essere allineato ad un paragrafo) ed a differenza dei paragrafi,
sono sovrapponibili.

La dimensione massima di un segmento (64k) deriva direttamente dalla dimensione massima che può avere
un indirizzo relativo.
Dato che l'accesso ad un indirizzo avviene attraverso i registri, la dimensione massima è $2^n$,
e per Intel 8086: $n = 16 \Rightarrow 2^{16} = 64k$.

L'indirizzo di inizio di un segmento è salvato in un indirizzo di memoria a 20bit,
ed è ottenuto da un registro a 16bit moltiplicato per 16.

La sovrapposizione dei segmenti era utilizzata in DOS nel tipo di eseguibile `.com`,
file che utilizzavano il modello di memoria `tiny`.
Prevedeva un unico segmento a cui corrispondevano DS SS (Stack Segment) e CS.
I segmenti erano sovrapposti solo come indirizzo, non come dati.
Siccome tutto era contenuto in un unico segmento, tra codice, dati e stack
non era concesso di superare i 64k.

\end{document}
