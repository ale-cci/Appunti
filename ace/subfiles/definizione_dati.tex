\documentclass[../template]{subfiles}

\begin{document}
\section{Definizione Dati}
La sintassi utilizzata per definire un dato in assembly è
necessaria una label, non richiesta ma utile per avere un riferimento per accedere al dato (facoltativo) ;
il tipo del dato, ed i valori per l'inizializzazione, che possono essere uno o più separati da virgola.

I tipi di dato a disposizione sono \lstinline{DB} (\textit{Define Byte} 8bit), \lstinline{DW} (\textit{Define Word} 16bit) e \lstinline{DD} (\textit{Define doubleword} 32bbit).
\\
Per dichiarare dei dati evitando l'inizializzazione è possibile utilizzare rispettivamente \lstinline{RESB}, \lstinline{RESW} e \lstinline{RESW}.

\begin{lstlisting}
ByteVar:    DB 0                ; Byte inizializzato a 0
ByteArray:  DB 1,2,3,4          ; Array di 4 byte
String:     DB '8086',0dh,0ah   ; Array di 6 caratteri (4 + CR/LF)
FiveTh:     DW 100*50           ; Assegnamento risultato operazione
Zeros:      times 256 DB 0      ; Array di 256 0

Table:      RESB 50             ; Array di 50 byte non inizializzati

NearPtr:    DW String           ; Contiene l'offset di String (NEAR)
FarPtr:     DD String           ; Contiene offset ed indirizzo del segmento
                                ; di String (FAR)
\end{lstlisting}

\subsubsection{Esempio utilizzo Variabili}
\lstinputlisting{ascii_repr.asm}
All'inizio del programma è richiesto che \ds punti alla sezione dei dati per accedere alle variabili dichiarate
\subsection{Istruzioni di logica binaria}
\begin{table}[h]
    \centering
    \begin{tabu}{|l|}
        \hline
        \lstinline{and   dest,sorg}  \\
        \lstinline{not   dest}   \\
        \lstinline{or    dest,sorg}   \\
        \lstinline{test  dest,sorg}   \\
        \lstinline{xor   dest,sorg}   \\
        \hline
    \end{tabu}
\end{table}
\lstinline{and}, \lstinline{not}, \lstinline{or}, \lstinline{xor} eseguono l'operazione logica sui bit dei registri forniti come parametro.
\lstinline{test} funziona come \lstinline{and} modifica i flag ma non salva il risultato. È spesso utilizzato per controllare se determinati bit siano ad 1 es: \lstinline{test al, 0010000b}.

\subsection{Shift e rotate}
\begin{table}[h]
    \centering
    \begin{tabu}{|l|l|}
        \hline
        \lstinline{shl dest, count} & Shift a sinistra di \code{count} positioni\\
        \lstinline{shr dest, count} & Shift a destra di \code{count} positioni\\
        \lstinline{sal dest, count} & Spostamento aritmetico a destra\
        \hline
    \end{tabu}
\end{table}

\end{document}
