% Lezione 1.2 - Introduzione ai calcolatori elettronici
\documentclass[../ace.tex]{subfiles}

\begin{document}
\section{Introduzione}
\subsection{Calcolatore Elettronico}
Un calcolatore elettronico è un sistema gerarchico che possiede le funzioni di elaborazione,
memorizzazione, trasmissione e di controllo.  Queste funzioni corrispondono in prima
approssimazione agli elementi: CPU, memoria, sistema I/O e Bus.

La CPU (unità di controllo) è ulteriormente divisa in 4 parti:
\begin{itemize}
    \item ALU: esegue le operazioni aritmetiche e logiche.
    \item Control Unit: comanda le unità del processore.
    \item Registri: memorie interne al processore, utilizzate per tenere temporaneamente i
        dati che il processore deve elaborare.
    \item Bus: Interconnessione interna per il trasferimento dati nel processore.
\end{itemize}

\subsection{Architettura di Von Neumann}
\begin{figure}[h]
    \centering
    \begin{tikzpicture}
        \node[block] (ms) {Memoria Secondaria};
        \node[block, below of=ms, node distance=1.5cm] (mc) {Memoria Centrale};
        \node[block, below of=mc, node distance=1.5cm] (cpu) {CPU};
        \node[block, right of=cpu, node distance=3cm] (out) {Output};
        \node[block, left of=cpu, node distance=3cm] (in) {Input};

        \draw[block, <->, thick] (ms)--(mc);
        \draw[block, <->, thick] (mc)--(cpu);
        \draw[block, ->, thick] (cpu) -- (out);
        \draw[block, ->, thick] (in) -- (cpu);
    \end{tikzpicture}

    \caption{Computer secondo architettura di Von Neumann}
    \label{img:von_neumann}
\end{figure}

\noindent La novità che porta l'architettura di Von Neumann è l'introduzione di una memoria.
Prima di questa architettura, i programmi erano salvati esternamente al calcolatore in schede perforate.
La memoria centrale (RAM) dell'architettura serviva per memorizzare temporaneamente i dati e permanentemente
nella memoria secondaria (HD).

Come si può vedere dallo schema in figura \ref{img:von_neumann}, si interfaccia solamente con la memoria
centrale.

\subsection{Architettura di Harvard}
\begin{figure}[h]
    \centering
    \begin{tikzpicture}
        \node[block] (alu) {ALU};
        \node[block, below of=alu, node distance=1.5cm] (uc) {Unità di Controllo};
        \node[block, right of=uc, node distance=3cm] (md) {Memoria Dati};
        \node[block, below of=uc, node distance=1.5cm] (io) {I/O};
        \node[block, left of=uc, node distance=4cm] (mp) {Memoria di Programma};

        \draw[block, <->, thick] (alu)--(uc);
        \draw[block, <->, thick] (uc)--(md);
        \draw[block, <->, thick] (uc)--(io);
        \draw[block, <->, thick] (uc)--(mp);
    \end{tikzpicture}
\end{figure}

\noindent La differenza fondamentale di questa architettura è la memoria del programma è separata dalla memoria dati, e le
istruzioni devono obbligatoriamente passare dalla CPU.
Questa architettura ha dato spunto a memorie separate come cache.
L'idea di poter accedere memoria e dati velocizza le prestazioni del calcolatore.

\subsection{Legge di Moore}
\begin{enumerate}
    \item Le prestazioni dei processori, e il numero di transistor ad esso relativo, raddoppiano ogni 18 mesi.
    \item IL costo di una fabbrica di chip raddoppia da una generazione all'altra.
\end{enumerate}

\subsection{Legge di Amdahl}
Nella storia dei computer, si è capito che il continuo aumento della frequenza di clock del numero di transistor del processore
non era una cosa plausibile. Per questo si è iniziato a ragionare sul parallelismo.

Nel momento in cui voglio parallelizzare un algoritmo, ho due parti fondamentali: una componente sequenziale ed una componente
parallelizzabile.
Chiamata $f$ la frazione di algoritmo parallelizzabile, ed ho a disposizione un numero di processori $N$, l'aumento di
velocità di esecuzione $S$ è calcolabile attraverso la legge di Amdahl:
\[
    S = \frac{1}{(1 - f) + \frac{f}{N}}
\]
È da tenere quindi presente che se utilizzo un grande numero di core per un algoritmo non parallelizzabile, il suo tempo di
esecuzione sarà identico se eseguito con un solo core.

Il parallelismo è stato quindi stato adottato dai processori attraverso pipelining, coprocessori paralleli (processori dedicati
a specifiche operazioni) ed architetture multicore.

\end{document}
