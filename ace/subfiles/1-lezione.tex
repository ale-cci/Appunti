% Lezione 1.2 - Introduzione ai calcolatori elettronici
\documentclass[../ace.tex]{subfiles}

\begin{document}
\section{Introduzione}
\subsection{Calcolatore Elettronico}
Un calcolatore elettronico è un sistema gerarchico suddiviso in elaborazione,
memorizzazione, trasmissione e di controllo.  Queste funzioni corrispondono
gli elementi: CPU, memoria, sistema I/O e Bus.

La CPU (unità di controllo) è ulteriormente divisa in 4 parti:
\begin{itemize}
    \item ALU: esegue le operazioni aritmetiche e logiche.
    \item Control Unit: comanda le unità del processore.
    \item Registri: memorie interne al processore, utilizzate per tenere temporaneamente i
        dati che il processore deve elaborare.
    \item Bus: Interconnessione interna per il trasferimento dati nel processore.
\end{itemize}

\subsection{Architettura di Von Neumann}
\begin{figure}[h]
    \centering
    \begin{tikzpicture}
        \node[block] (ms) {Memoria Secondaria};
        \node[block, below of=ms, node distance=1.5cm] (mc) {Memoria Centrale};
        \node[block, below of=mc, node distance=1.5cm] (cpu) {CPU};
        \node[block, right of=cpu, node distance=3cm] (out) {Output};
        \node[block, left of=cpu, node distance=3cm] (in) {Input};

        \draw[block, <->, thick] (ms)--(mc);
        \draw[block, <->, thick] (mc)--(cpu);
        \draw[block, ->, thick] (cpu) -- (out);
        \draw[block, ->, thick] (in) -- (cpu);
    \end{tikzpicture}

    \caption{Computer secondo architettura di Von Neumann}
    \label{img:von_neumann}
\end{figure}
\noindent
La caratteristica principale dell'architettura è l'introduzione di una memoria interna.
Precedentemente, i programmi erano salvati esternamente in schede perforate.
La memoria centrale (RAM) è temporanea e fa da tramite alla memoria secondaria (HD), utilizzata per salvare
permanentemente i dati.

\subsection{Architettura di Harvard} \label{sec:architettura_harvard}
\begin{figure}[h]
    \centering
    \begin{tikzpicture}
        \node[block] (alu) {ALU};
        \node[block, below of=alu, node distance=1.5cm] (uc) {Unità di Controllo};
        \node[block, right of=uc, node distance=3cm] (md) {Memoria Dati};
        \node[block, below of=uc, node distance=1.5cm] (io) {I/O};
        \node[block, left of=uc, node distance=4cm] (mp) {Memoria di Programma};

        \draw[block, <->, thick] (alu)--(uc);
        \draw[block, <->, thick] (uc)--(md);
        \draw[block, <->, thick] (uc)--(io);
        \draw[block, <->, thick] (uc)--(mp);
    \end{tikzpicture}
\end{figure}

\noindent
Ideata ad Harvard, questa architettura è caratterizzata da una separazione tra la memoria del programma e la memoria dati.
A causa di questa separazione, le istruzioni sono obbligate a passare attraverso la CPU.
Questa architettura ha dato spunto a memorie separate a rapido accesso come la cache.
L'idea di poter accedere separatamente a memoria del programma e memoria dati velocizza le prestazioni del calcolatore.

\subsection{Leggi di Moore}
\begin{enumerate}
    \item Le prestazioni dei processori, e il numero di transistor ad esso relativo, raddoppiano ogni 18 mesi.
    \item IL costo di una fabbrica di chip raddoppia da una generazione all'altra.
\end{enumerate}

\subsection{Legge di Amdahl}
L'aumento progressivo della frequenza di clock e di transistor interni al processore non è sostenibile.
Per questo si è iniziato a ragionare sul parallelismo.

Nel momento in cui si vuole parallelizzare un algoritmo, è possibile suddividere le istruzioni del programma in due gruppi: una componente sequenziale ed una parallelizzabile.
Chiamata $f$ la frazione di algoritmo parallelizzabile, ed $N$ il numero di processori a disposizione, l'aumento di
velocità di esecuzione $S$ (\textit{Speedup}) è calcolabile con la legge di Amdahl:
\[
    S = \frac{1}{(1 - f) + \frac{f}{N}}
\]
Da come si può evincere dalla formula, avere a disposizione un elevato numero di core per un algoritmo non parallelizzabile ($f = 0$), non porta alcun miglioramento:
\[
    \lim_{N \to +\infty} \frac{1}{1 + \frac{1}{N}} = 1
\]
Il parallelismo è  utilizzato in architetture pipeline, coprocessori paralleli (processori dedicati a specifiche operazioni) ed architetture multicore.

\end{document}
