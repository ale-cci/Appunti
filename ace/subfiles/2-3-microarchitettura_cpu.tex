\documentclass[../ace.tex]{subfiles}

\begin{document}
\section{Microarchitettura CPU}
\subsection{La CPU}
Rete logica combinatoria: rete priva di memoria, che dati ingressi da' delle uscite indipendentemente
dal tempo.
Rete sequenziale: rete che ha memoria, macchina a stati finiti, che dati degli ingressi fornisce
delle uscite dipendenti dallo stato in cui si trova.

Dal punto di vista funzionale possiamo vedere la CPU, come composta da due parti: il data path e
l'unità di controllo.
Il data path di cui componente fondamentale è l'ALU, è una rete logica che si occupa di eseguire le istruzioni, mentre
l'unita di controllo è una macchina a stati finiti che comanda il data path su quali istruzioni eseguire.
\\
Possiamo schematizzare l'unità di controllo come una macchina a stati finiti a tre stati: fetch, decode ed execute.

\subsection{Architettura di riferimento RISC}
Analizziamo adesso un esempio di architettura RISC.

Gli indirizzi di memoria sono riferiti ai byte. Le istruzioni occupano sempre e solo una parola a 32 bit.
Le istruzioni ed i dati si trovano sempre in indirizzi multipli di 4, per questo motivo il program counter
è incrementato di 4 ad ogni istruzione.
Avremo anche 32 registri di uso generale a 32bit. L'insieme dei registri è chiamato register file.

In generale l'esecuzione all'interno di un processore passa attraverso 5 fasi:
una prima fase chiamata Instruction Fetch (IF) dove il processore carica dalla memoria l'istruzione,
successivamente la decodifica nella fase (FT) e la esegue nella fase di execute (EX).
Dopodiché in alcuni casi si ha un accesso alla memoria nella fase di Memory (ME), ed infine
in alcuni casi ho una fase di write-back (WB), dove il risultato delle operazioni è riscritto
nei registri.

Tutte le istruzioni di questa ISA, come già detto hanno una lunghezza fissa di 32bit, ed appartengono a
3 categorie:
\begin{itemize}
    \item Un primo aritmetico logiche,
        Nei primi 6 bit opcode (codice operativo),
        seguiti da 5 che indicano il primo registro sorgente,
        altri 5 bit che indicano il secondo registro,
        5 che indicano il registro destinazione
        ed i rimanenti 11 fanno riferimento all'operazione specifica dell'ALU (somma, sottrazione, ...).
        \begin{lstlisting}
    add  r4,r2,r5
    \end{lstlisting}

    \item Un secondo formato utilizzato per accesso alla memoria e salti condizionati.
        I primi 6 bit di codice operativo,
        a seguire altri 5 ad indicare il primo registro
        ed altri 5 ad indicare il secondo registro (come nel primo caso)
        16 che indicano un offset o un dato.

        \begin{lstlisting}
    je  r2,r3,0045h
        \end{lstlisting}
    \item Un terzo ed ultimo caso utilizzato per i salti incondizionati.
        I primi 6 bit indicano sempre l'opcode ed i rimanenti indicano l'indirizzo di termine del
        salto.

        \begin{lstlisting}
    jmp 0045h
        \end{lstlisting}
\end{itemize}
In questa prima architettura assumiamo che non ci sia lo stack (zona di memoria organizzata a pila LIFO),
accessibile attraverso istruzioni apposite come \lstinline{push} e \lstinline{pop}.

Oltre a salto condizionato ed incondizionato esiste un terzo tipo di salto: la chiamata a funzione.
È un tipo di salto incondizionato particolare, dato che quando eseguito è necessario salvare il valore
del program counter per poter proseguire l'esecuzione una volta terminata la funzione.

Siccome non abbiamo uno stack, il program counter è salvato nell'indirizzo $R_{31}$
%27: ragionamento di funzionamento dell'ALU

Per selezionare le operazioni da far eseguire all'ALU è specificato un ingresso chiamato opalu, dove vengono
specificate le operazioni. Oltre all'uscita dell'operazione vengono tornati dei valori aggiuntivi come
il flag di zero (messo ad uno quando il risultato dell'ALU è zero), ed il flag di segno (messo ad 1 quando il
risultato è negativo).

%35
Load e store permettono la lettura e scrittura in memoria.

\end{document}
