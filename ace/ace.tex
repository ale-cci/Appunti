%!TEX program = xelatex
\documentclass{custom}
\usetikzlibrary{calc,positioning,shapes,arrows,backgrounds,external,fit}

\lstset{inputpath=assembly/src}
\usepackage{wrapfig}
\usepackage{graphicx}
\graphicspath{{./img/}}
\usepackage{circuitikz}

\usepackage{tabu}
\taburulecolor{code-comment}
\tabulinesep = 2mm

\tikzstyle{block} = [fill=blue!20, draw, rectangle, minimum height=3em, minimum width=6em]

\definecolor{code-bg}{HTML}{EFF0F1}
\definecolor{code-keyword}{HTML}{0054A3}
\definecolor{code-comment}{HTML}{A5ACB2}
\definecolor{important-box}{HTML}{D1E5F1}

% Document specific commands
\newcommand{\cs}{\lstinline{cs}}

\newcommand\ax{\lstinline{ax}}
\newcommand\bx{\lstinline{bx}}
\newcommand\cx{\lstinline{cx}}
\newcommand\dx{\lstinline{dx}}
\newcommand\al{\lstinline{al}}
\newcommand\ah{\lstinline{ah}}
\newcommand\cf{\lstinline{cf}}
\newcommand\zf{\lstinline{zf}}
\renewcommand\sf{\lstinline{sf}}
\newcommand\of{\lstinline{of}}
\newcommand\ds{\lstinline{DS}}


\newcommand{\code}[1]{\oldlstinline[keywordstyle=\ttfamily]{#1}}
\setcounter{tocdepth}{2} % Show only sections in table of contents

\title{Architettura dei calcolatori elettronici}
\author{ale-cci}

\begin{document}
\maketitle{}
\setlength\parskip{0pt}
\pagestyle{empty}
\thispagestyle{empty}
\tableofcontents
\cleardoublepage

% Paragraph styling
\setlength\parindent{0em}
\setlength\parskip{1em}
\newpage

\pagestyle{plain}
\setcounter{page}{1}

\subfile{subfiles/1-lezione}
\subfile{subfiles/2-1-lezione}
\subfile{subfiles/2-3-microarchitettura_cpu}
\subfile{subfiles/architetture_avanzate.tex}
\subfile{subfiles/introduzione_ai_linguaggi_assembly}
\subfile{subfiles/4-3_modelli-di-memoria}
\subfile{subfiles/4-4_modalita_di_indirizzamento}
\subfile{subfiles/4-5_introduzione_assembly}
\subfile{subfiles/4-6_aritmetica_binaria}
\subfile{subfiles/trasferimento_di_controllo}

\subfile{subfiles/definizione_dati}
\subfile{subfiles/istruzioni_logiche}
\subfile{subfiles/operazioni_su_stringhe_di_dati}
\subfile{subfiles/passaggio_di_parametri_a_funzione}
\subfile{subfiles/memorie}
\subfile{subfiles/memorie_cache}
%[nextsection]
\end{document}
