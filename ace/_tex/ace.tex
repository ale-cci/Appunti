%!TEX program = xelatex
\documentclass[a4paper,10pt]{article}

\usepackage[margin=1.2in]{geometry}
\usepackage{amsmath}
\usepackage{subfiles}
\usepackage{titlesec}
\usepackage{listings}
\usepackage{xcolor}
\usepackage{fancyhdr}
\usepackage{tabu}   % tabular with colors

\usepackage{fontspec}
\setmainfont{Roboto}


\definecolor{code-bg}{HTML}{EFF0F1}
\definecolor{code-keyword}{HTML}{0054A3}
\definecolor{code-comment}{HTML}{A5ACB2}
\definecolor{important-box}{HTML}{D1E5F1}

\renewcommand{\familydefault}{\sfdefault}
\newcommand{\warning}[1]{
    \begin{center}
    {
        \setlength\fboxsep{.5cm}
        \colorbox{important-box}{
            \parbox{.8\linewidth}{#1}
        }
    }
    \end{center}
}

\taburulecolor{code-comment}
\tabulinesep = 2mm

\titleformat{\section}
    {\color{black}\Huge}
    {}
    {0em}
    {}[\color{code-comment}\titlerule]

\titlespacing{\section}{0em}{1em}{3em}
\newcommand{\sectionbreak}{\clearpage}

\titleformat{\subsection}
    {\color{code-keyword}\large\bfseries}
    {}
    {0em}
    {}

\lstset{
    language=[x86masm]{Assembler},
    backgroundcolor=\color{code-bg},
    basicstyle=\small\ttfamily,
    commentstyle=\color{code-comment},
    keywordstyle=\color{code-keyword}\bfseries,
    showstringspaces=false
}

\makeatletter
\def\@maketitle{
    \null
    \vfill
    \begin{center}\leavevmode
        \normalfont
        {\LARGE\raggedleft \@author\par}%
        {\color{code-keyword}\hrulefill\par}
        {\huge\raggedright \@title\par}%
        \vskip 1cm
        %    {\Large \@date\par}%
    \end{center}%
    \vfill
    \null
    \thispagestyle{empty}%
    \clearpage
    \pagenumbering{arabic}
}
\makeatother

\let\oldlstinline\lstinline
\renewcommand{\lstinline}[1]{\colorbox{code-bg}{\oldlstinline{#1}}}

\title{Architettura Calcolatori Elettronici}
\author{ale-cci}


% Document specific commands
\newcommand{\cs}{\lstinline{cs}}

\begin{document}
\maketitle{}

\subfile{lectures/1-lezione}
\section{Lezione 2}
ISA: Instruction Set Architecture, lista id istruzioni disponibili al programmatore utilizzate dal processore

CISC: Complex Instruction Set Computer
\\
Facilita la programmazione, fornendo al programmatore un vasto numero di istruzioni possibili\\
, aumentando però la complessitá a livello di microcodice.


RISC: Reduced Set Computer, Ha un ISA più vicina all Hardwere, e piu lontana dal programmantore.
\\
Ha un set efficente e limitato di istruzioni, basato sul fatto che solo un ridotto numero di istruzioni è utilizzato.
\\
\textit{Confronto 'RISC' e 'CISC' a lezione 3.1 in 20:19}

Memoria Allineata: Accesso più veloce ai ai dati, non efficiente per 'storing' dei dati.
\\
Per cercare di accedere il meno possibile alla memoria vengono utilizzati \textbf{Registri}.
L'architettura RISC utilizza molti registri oriogonali (ogni registro può effettuare ogni operazione),
mentre CISC utilizza meno registri non ortogonali, ovvero esistono registri specifici per specifiche operazioni.
\section{Lezione 3}
CPU: Unità di controllo a stati finiti (fetch, decode, execute)

PC: Program Counter

MR: Memory register

DR: Data register

La memoria si interfaccia con:
\begin{itemize}
    \item Bus di indirizzi (solo in input)
    \item Bus di dati, \\utilizzato sia per leggere che per scrivere i dati sulla memoria
\end{itemize}

Con $R_{2..n}$ vengono indicati i vari registri

Register File: (slide 43)

\subsection{CPU Monociclo}
\textbf{Modello Harvard}: Due memorie separate.% Aggiungere descrizione

Richiede due memorie separate, siccome accede due volte alla memoria nello stesso ciclo di clock.

\begin{enumerate}
    \item fetch: accede alla memoria per leggere istruzioni della CPU
    \item decode
    \item execute
    \item memoria
\end{enumerate}

Writeback (WB): Il contenuto letto dalla memoria viene scritto su un registro.

Siccome tutte le operazioni vengono eseguite in un unico ciclo di clock, il tempo impiegato da ciascuna istruzione diventa uguale al tempo impiegato dalla 'funzione' più lenta: $T_{mono} = 83 ns$.
\\
Il tempo di esecuzione del programma diventa $N \cdot 83ns$

\subsection{Lezione 3.6 - Architettura Multiciclo}
A differenza della architettura monociclo, in ogni singolo stadio opera in un ciclo di clock.
Motivato dal fatto che non tutte le istruzioni impiegano lo stesso tempo di esecuzione.

Per calcolare il tempo impiegato da ogni istruzione $\sum{T_{multi}}$.
A volte $\sum{T_{mutli}} \ge T_{mono}$.

Non richiede più di separare la memoria, dato che l'esecuzione dell'istruzione è separata in più cicli di clock.

\begin{itemize}
    \item Fetch
        \begin{itemize}
            \item Recupero istruzione dalla memoria
            \item Aumento PC
                \\
                Posso usare l'ALU per il fetch ($Pc = Pc + 5$)
        \end{itemize}
    \item Decode:
        Decodifica operandi, Anticipo il calcolo dei registri
        \\
        Viene calcolato a prescindere anche il valore di eventuali salti condizionali,
        sfruttando l'ALU non ancora utilizzata in questo passaggio

    \item Execute:
        \\
        Varia da istruzione ad istruzione,
\end{itemize}

\section{Lezione 4 - Architetture Avanzate}
Benchmark: Strumento per la valutazione delle prestazioni di esecuzione di un calcolatore

\subsection{Tempo di esecuzione}
Valuta il tempo di esecuzione della CPU.
In generale il tempo di CPU, può essere misurato tramite tempo di esecuzione, il modello più semplice per farlo è attraverso il
\textit{CPI}: Clock per Istruzione (in media),

\begin{itemize}
    \item $CPI = N_{cc}/N_i$
    \item $N_{cc}$: Numero di cicli di clock
    \item $N_i$: Numero delle istruzioni
\end{itemize}

\[
\begin{cases}
    T_{cpu} &= N_{cc} \cdot T_{ck}
    \\
    N_{cc} &= N_i \cdot C_{pi}
\end{cases}
\]


Obbiettivo RISC:
\\
Ridurre il clock per instruction, rimuovendo istruzioni non necessarie, ottimizzando il numero ridotto di istruzioni dell'architettura.

\textit{MIPS}: Milioni di istruzioni al secondo $ = N_i / CPU_{time} * 11^6$

\subsection{Pipeline ed Alee}
\begin{itemize}
    \item Alee Strutturali: Due blocchi richiedono la stessa risorsa
    \item Alee di Dato: Un blocco è in attesa di una risorsa prodotta da un altro blocco
    \item Alee di Controllo: Quando il processore non ha ancora determinato l'indirizzo di un salto condizionato
\end{itemize}

\subsection{Alee di Dato}
\begin{itemize}
    \item \textbf{RAW} \textit{Read After Write}:
        $B$ ha bisogno di un dato prodotto da $A$, ma $B$ deve leggere il dato prima che $A$ abbia finito di scriverlo.
        \\
        Caso più frequente
    \item \textbf{WAR} \textit{Write After Read}:
        $B$ deve modificare un dato prima che sia letto da $A$
    \item \textbf{WAW} \textit{Write After Write}:
        $B$ deve scrivere un dato prima che $A$ lo abbia scritto.
\end{itemize}

Prima di poter risolvere un \textit{Alea di Dato}, occorre riconoscerla.
\\
Per questo viene prima confrontata la fase di \textit{decode} con quella di \textit{memory}, \textit{execute} e \textit{writeback}.

Oppure occorrono campi specifici sulle pipeline (\textit{latch}), che vanno ad indicare quali registri vengono utilizzati da una determinata istruzione.

\subsubsection{Metodi Risolutivi}
\begin{itemize}
    \item \textbf{Stallo}:
        \\
        Attendo il termine dell'istruzione (molto penalizzante e da evitare)
    \item \textbf{Anticipazione}:
        Produco il dato immediatamente, introducendo.
        \\
        Costosa e non banale da realizzare
    \item \textbf{Sovrapposizione}:
        Permetto la produzione di 3 risultati in un unico ciclo di clock.
        \\
        Sovrappongo le due istruzioni che richiedono la stessa risorsa, e eseguo un istruzione nel fronte di clock di salita,
        e l'altra nel fronte del clock in discesa.
        \\
        Di fatto è come raddoppiare il clock.

    \item \textbf{Riordinamento}:
        \\
        Viene cambiato l'ordine di esecuzione delle istruzioni.

\end{itemize}

\subsection{Alee di controllo}
Avvengono sia in caso di salti condizionati, che in caso di salti\\ incondizionati.
\\
È risolvibile inserendo uno \textit{stallo} di 1 ciclo di clock, altrimenti anche attraverso il riordinamento, ma non sempre é possibile.


\section{Architettura Supercalcolatori}
Parallelismo pipelines

\textbf{ROB}: Buffer di riordinamento


\subfile{lectures/4-3_modelli-di-memoria}
\subfile{lectures/4-4_modalita_di_indirizzamento}
\subfile{lectures/4-5_introduzione_assembly}

\end{document}
