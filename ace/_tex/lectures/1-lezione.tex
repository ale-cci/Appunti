% Lezione 1.2 - Introduzione ai calcolatori elettronici
\documentclass[../ace.tex]{subfiles}

\begin{document}
\section{Introduzione}
\subsection{Calcolatore Elettronico}
Un calcolatore elettronico è un sistema gerarchico che possiede le funzioni di elaborazione,
memorizzazione, trasmissione e di controllo.  Queste funzioni corrispondono in prima
approssimazione agli elementi: CPU, memoria, sistema I/O e Bus.

La CPU (unità di controllo) è ulteriormente divisa in 4 parti:
\begin{itemize}
    \item ALU: esegue le operazioni aritmetiche e logiche.
    \item Control Unit: comanda le unità del processore.
    \item Registri: memorie interne al processore, utilizzate per tenere temporaneamente i
        dati che il processore deve elaborare.
    \item Bus: Interconnessione interna per il trasferimento dati nel processore.
\end{itemize}

\subsection{Architettura di Von Neumann}
xoxo

\end{document}
