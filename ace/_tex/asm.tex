\documentclass{article}

\title{Linguaggi Assembly}
\begin{document}
\section{Architettura Logica CPU Intel 8086}
Processore 'general purpose'
\\
16bit: I registri interni, ALU e risultati sono tutti a 16bit

n : numero di registri interni = 16

na : parallelismo bus indirizzi = 20

nb : parallelismo bus dati = 16



Bus control: Unico punto d'accesso al BUS

Coda di Per fetch: Salva le istruzioni prese durante la fase di fetch in una coda.

EU Control: Unita di controllo, genera tutti i segnali che servono a tutte le altre parti della EU, es: abilitazioni registri, operazioni che l' ALU deve eseguire.
\\
Registro di Flag: contengono delle informazioni riguardante lo stato dell'ultima operazione eseguita dall' ALU

IP viene aggiornato allo stesso tempo in cui i dati vengono passati all'EU-Control, (aggiornato solo in caso di salto)

ALU Data BUS: bus interno al processore, utilizzato per passare informazioni tra ALU-I/O e BU

\section{Le 7 Scelte progettuali}
\subsection{1 - Dove sono memorizzati gli operandi nella CPU}

\begin{enumerate}
    \item Stack:
        Memoria gestita a pila (LIFO)
        \\
        Indipendente dai registri, utilizzato da JVM
    \item Accumulatore
        Unico registro accumulatore, da cui passavano tutte le operazioni
        \\
        Problema: Accumulatore è un bottleneck.
    \item Set di registri
        Insieme di registri, più o meno ampio. Metodo più utilizzato
\end{enumerate}

3 Categorie di register

\subsubsection{GPR - General Purpose register}
8 AX, BX, CX, DX, SI, DI BP, SP


Tutti a 16bit. I primi 4 possono essere utilizzati anche a lunghezza più corta, chiamandoli con AL (bit da 0 a 7) per la parte basse ed AH per i bit da 8 a 15, per il caso di AX.

A: Accumulatore, registro base per le operazioni

B: Base, utilizzato quasi per tutte le operazioni.
\\
Alcune operazioni richiedono esclusivamente BX.
C: Counter, unico utilizzabile nei cicli per es. Counter
\\
D: Data, utilizzabile per indirizzi di istruzioni IO, ed anche per dati che eccedono i 16bit.
\\
SI, DI: Source-Destination Indice
\\
BP: Base pointer
\\
SP: Stack pointer, points at the top of the stack. Utilizzati per l'accesso ai parametri delle funzioni.

\subsubsection{SR: Segment Register e MR: Miscellaneous register}
IP: Instruction pointer
\\
FLAG: history Flag, registro a 16 bit dove ognuno di essi indica la modalità di funzionamento del software/ stato del sistema

Non si può fare riferimento a registri di FLAG

I flag sono di 2 tipi:
\begin{itemize}
    \item Stato:
        Indicano uno stato, una situazione di un operazione di ALU
    \item Controllo:
        Modificati dal programmatore per cambiare il funzionamento della macchina
\end{itemize}

\begin{itemize}
    \item OF:
        Overflow register, segnala quando avviene un overflow
    \item DF:
        Direction Flag, Indica se incrementare o decrementare l'indice di lettura
    \item IE:
        Interrupt Enable. Permette di specificare sezioni di codice dove ignorare i segnali di interrupt
    \item TF, TRAP:
        Utilizzato dai debugger, per eseguire i programmi 1 step alla volta
    \item SF, SIGN:
        1 se il risultato di un operazione è negativo
    \item ZF: ZERO
        1 se il risultato di un operazione è 0
    \item AF, AUXILIARY CARRY:
        Indica la presenza di riporto tra le due parti a 8 bit di AX
    \item PF: Parity Flag
        1 quando il numero di 1 nel risultato dell'operazione è pari
    \item CF: Carry Flag:
        1 se c'è riporto nell'ultima operazione, utilizzato per fare somme a 32 bit
\end{itemize}



\subsection{2 - Con che istruzioni si accede agli operandi}
\subsection{3 - Modello della memoria}
\subsection{4 - Formato delle istruzioni}
Quali sono le istruzioni supportate
\subsection{5 - Modalità di indirizzamento}
Come faccio ad utilizzare un dato (es indirizzo di memoria).
\subsection{6 - Struttura degli operandi}
\subsection{7 - Operazioni Previste}
\end{document}
