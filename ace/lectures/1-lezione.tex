% Lezione 1.2 - Introduzione ai calcolatori elettronici
\documentclass[../ace.tex]{subfiles}

\begin{document}
\section{Introduzione}
\subsection{Calcolatore Elettronico}
Un calcolatore elettronico è un sistema gerarchico che possiede le funzioni di elaborazione,
memorizzazione, trasmissione e di controllo.  Queste funzioni corrispondono in prima
approssimazione agli elementi: CPU, memoria, sistema I/O e Bus.

La CPU (unità di controllo) è ulteriormente divisa in 4 parti:
\begin{itemize}
    \item ALU: esegue le operazioni aritmetiche e logiche.
    \item Control Unit: comanda le unità del processore.
    \item Registri: memorie interne al processore, utilizzate per tenere temporaneamente i
        dati che il processore deve elaborare.
    \item Bus: Interconnessione interna per il trasferimento dati nel processore.
\end{itemize}

\subsection{Architettura di Von Neumann}
\begin{center}
    \begin{tikzpicture}
        \node[block] (ms) {Memoria Secondaria};
        \node[block, below of=ms, node distance=2cm] (mc) {Memoria Centrale};
        \node[block, below of=mc, node distance=2cm] (cpu) {CPU};
        \node[block, right of=cpu, node distance=3cm] (out) {Output};
        \node[block, left of=cpu, node distance=3cm] (in) {Input};

        \draw[block, <->, thick] (ms)--(mc);
        \draw[block, <->, thick] (mc)--(cpu);
        \draw[block, ->, thick] (cpu) -- (out);
        \draw[block, ->, thick] (in) -- (cpu);
    \end{tikzpicture}
\end{center}

La novità che porta l'architettura di Von Neumann è l'introduzione di una memoria.
\end{document}
