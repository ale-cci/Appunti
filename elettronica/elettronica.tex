\documentclass{article}

\usepackage[margin=1in]{geometry}
\usepackage{amsmath}
\usepackage{float}
\usepackage{wrapfig}
\usepackage{graphicx}
\usepackage[makeroom]{cancel}
\usepackage[usestackEOL]{stackengine}
\usepackage{mathtools}
\usepackage[oldvoltagedirection]{circuitikz}
\usepackage{multicol}
\usepackage{tikz}
\usepackage{pgfplots}

\graphicspath{{../}}
\ctikzset{bipoles/length=1.2cm}
\pgfplotsset{compat=1.16}

\begin{document}

% \section{Semiconduttori}
Per studiare le caratteristiche corrente-tensione dei circuiti basati su strutture a semiconduttore, biogna studiare il flusso di carica degli elettroni che si muovono nei materiali.\\
Gli elettroni fanno parte della struttura di un singolo atomo

\subsection{Struttura dell'atomo}

Il \textbf{nucleo} \'e formato da un coglomerato di neutroni e protoni, con carica complessiva positiva, attorno al quale ruotano gli elettroni con carica negativa.

L'\textbf{elettrone} che orbita attorno al nucleo ha una propria energia cinetica. La sua `posizione' deriva dal bilanciamento di due forze:
\begin{itemize}
    \item L'accelerazione centrifuga, che allontana l'elettrone dal nucleo
    \item La fora Coulombiana generata dalle due cariche di segno opposto che \'e inversamente proporzionale al quadrato della loro distanza.
\end{itemize}

Piu `veloce' l'elettrone gira attorno al nucleo pi\'u larga \'e la sua orbita, quindi minore \'e l'energia di legame con i protoni in quanto \'e inversamente proporzionale al quadrato della distanza.
L'elettrone oltre ad avere propriet\'a di natura corpuscolare (cio\'e ha una massa discreta che risponde alla legge della meccanica classica), ha propriet\'a ondulatorie ed inoltre non tutte le orbite sono possibili, e quindi, siccome ad un orbita l' elettrone ha una particolare energia, l'elettrone pu\'o avere solo alcuni valori di energia.

\textbf{L'energia \'e quantizzata}: i livelli di energia permessi sono
\begin{itemize}
    \item diversi per ogni tipo di atomo
    \item distribuiti non continuamente
    \item occupati al pi\'u da due elettroni (con spin opposto) $\rightarrow$ \underline{il numero di elettroni per ogni livello \'e finito}\\
        \textbf{per semplicit\'a} ogni livello pu\'o avere solo un elettrone
\end{itemize}

In un \textbf{sistema in quiete}, non perturvato, gli elettroni tendono spontaneamente ad assumere la posizione ocn il livello di energia inferiore.

Un materiale \'e un insieme di atomi che interagiscono tra di loro cio\`e gli elettroni di un atomo sentono la presenza degli altri atomi attraverso la repulsione e l'attrazione. Le orbite variano perch\`e il bilancio delle forze \'e diverso (pi\'u cariche positive, nuclei, e pi\'u cariche negative, elettroni) e quindi le energie si modificano.

Il sistema costituito da due atomi, per esempio, ha diversi livelli energetici dei due atomi isolati.


Il numero di livelli di due atomi che interagiscono non pu\'o variare in quanto non varia il numero di elettroni; varia soltanto la struttura.

Tipicamente ad ogni livelo energetico di un singolo atomo ne corrispondono due vicini e distinti.

Le propriet\'a rimangono uguali in particolare l'energia \'e discretizzata in livelli in cui pu\'o stare un solo elettrone (coppia di spin opposti)

\subsection{Modello a bande energetiche}

L'interazione di pi\'u atomi provoca l'avvicinamento di rispettivi livelli $\rightarrow$ si forniscono ``Grappoli'' di livelli nel quale c'\'e sempre un alternanza tra regioni ricche di livelli permessi e regioni proibite

\begin{itemize}
    \item \textbf{Modello a bande}: ci sono intervalli in cui c'\'e un elevato numero di livelli tutti discreti
        \begin{itemize}
            \item la distribuzione \textbf{non \'e continua}
            \item ogni manda ha un numero finito di livelli e quindi (siccome le propriet\'a di un singolo atomo valgono ancora) ha un numero finito di elettroni che riempiono varie bande partendo da quelle con poca energia.
            \item \textbf{livello energetico di Fermi}: livello limite al di sotto del quale, in assenza di perturazione, tutti i livelli energetici sono occupati al di sopra del quale tutti i livelli energetici sono vuoti.
            \item la caratteristica di un materiale dipende dalla posizione del livello energetico di Fermi
        \end{itemize}
\end{itemize}

\subsection{Tipologia di un materiale}
Per indurre corrente in un materiale bisogna perturbarlo per poter accelerare gli elettroni $\rightarrow$ la perturbazione varia la velocit\'a degli elettroni che cambiano energia cinetica $\rightarrow$ varia la posizione dell'elettrone che ``salta'' in un altro livello energetico

Per avere corrente, quindi occorre che:
\begin{enumerate}
    \item ci sia energia prodotta da una sorgente esterna
    \item l'energia sia sufficente a far ``saltare'' l'elettrone da un livello ad un altro
\end{enumerate}

\subsection{Materiali conduttori}
Il livello di Fermi cade in una banda permessa $\rightarrow$ la distanza minima di energia necessaria a modificare lo stato energetico degli elettroni (creazione di corrente) \'e la distanza tra due livelli che stanno nella stessa banda ed essendo la banda densa di livelli d'energia necessaria \'e piccola

\subsection{Materiali isolanti}
Il livello di Fermi cade in una banda proibita $\rightarrow$ la quantit\'a minima necessaria a muovere un elettrone \'e la distanza tra l'ultimo livello occupato e il primo libero cio\'e almeno pari al gap energetico

L'energy gap \'e molto pi\'u grande della distanza minima tra due livelli energetici della stessa banda e per questo i materiale isolanti conducono molto pi\'u difficilmente elettroni rispetto a materiali conduttori

Per\'o un materiale isolante pu\'o sempre condurre elettroni ma, al contrario dei materiali conduttori, ha bisogno di pi\'u energia (pari almento all'energy gap)

Si pu\'o notare che un elettrone per potersi muovere deve essere in una banda con alcuni livelli energetici liveri quindi una banda complentamente vuota o completamente piena non d\`a nessun contributo alla corrente.
\begin{itemize}
    \item le bande completamente piene vengono trascurate dall'analisi della corrente in quanto i loto elettroni hanno una remota possibilit\'a di ``saltare'' in un livello energetico libero
    \item la situazione \'e come quella di una bottiglia: il liquido si pu\'o muovere grazie ad una perturbazione se la bottiglia non \'e completamente vuota o piena (il liquido c'\'e ma non ha spazio per muoversi).
    \item Vengono definite le due bande fondamentalmente (quelle a cavallo del livello energetico di Fermi)
\end{itemize}

\textbf{Banda di conduzione}: sopra il livello di Fermi


\textbf{Banda di valenza}: Sotto il livello di Fermi


\subsection{Semiconduttore}
La differenza tra i semiconduttori e gli altri materiali \'e la diversa interazione con il mondo esterno.

Metodi  di perturbazione
\begin{itemize}
    \item \textbf{termica}: (siccome la termperature \'e diversa dallo 0 assoluto, gli elettroni che compongono le molecole dei vari materiali si muovono in funzione della temperatura in modo caotico)
    \item \textbf{ottica}: (n fotone porta una cera energia che viene trasformata da una radiazione)
\end{itemize}

Il livello teorico di Fermi \'e in condizinoe della non perturbazione cio\'e allo 0 assoluto.\\
Nella realt\'a ci sono continui scambi di energia

\begin{itemize}
    \item in un \textbf{isolante} l'energia media scambiata non \'e sufficiente a foar ``saltare'' un elettrone dalla banda di valenza a quella di conduzione
    \item in un \textbf{semiconduttore} il gas energetico \'e paragonabile all'energia scambiata con l'ambiente per un semplice effetto termico
\end{itemize}

\subsection{Caratteristiche del semiconduttore}
\begin{enumerate}
    \item il livello energetico di Fermi cade nella banda proibita
    \item l'energy gap \'e paragonabile all'energia media scambiata con l'ambiente esterno
\end{enumerate}

La probabilit\'a dell'elettrone che riceva un energia pari ad 1 gap \'e elevata in un semiconduttore a temperatura ambiente.

$\rightarrow$ si pu\'o generare corrente in quanto
\begin{enumerate}
    \item l'elettrone (che si troa in un livello energetico pi\'u alto e ha quindi una maggiore energia cinetica $\rightarrow$ maggiore velocit\'a) si trova in una banda con molti livelli energetici vuoti, serve poca energia per farlo saltare in un altro livello energetico (per questo la \textbf{banda} si chiama \textbf{di conduzione} $\rightarrow$ si comporta come un conduttore)
    \item \textbf{la banda di valenza} diventa parzialmente occupata in cui si pu\'o verificare un moto di elettroni con piccola energia
\end{enumerate}

\subsection{Differenza tra un conduttore ed un semiconduttore}
\begin{itemize}
    \item In un conduttore esiste solo una banda parzialmente occupata che costituisce il trasporto di corrente
    \item In un semiconduttore necessariamente esistono due bande entrambe parzialmente occupate $\rightarrow$ questo provoca la caratteristica non lineare
\end{itemize}

\subsubsection{Osservazione}
La quantit\'a di elettroni presenti nella banda di conduzione dipende dalla temperature $\rightarrow$ la conducibilit\'a elettrica consumata all'aumentare della temperatura (in quanto aumenta il n\textsuperscript{o} di elettroni che possono ``saltare'' nella banda di conduzione)
\begin{itemize}
    \item questo effetto \'e utilizzato per i sensori di temperature in quanto $\theta$ \'e funzione della temperature.
    \item questo effetto pu\'o essere pericoloso perch\'e la corrente provoca calore per effetto Joule $\rightarrow$ il componente si autoriscalda
    \item \'e un meccanismo instabile che pu\'o provocare la rottura di un dispositivo (problema dell'elettronica di potenza)
\end{itemize}

\section{Meccanismo di conduzione nei Semiconduttori}
I materiali semiconduttori sono prevalentemente della quarta colonna, della tabola periodica degli elementi.

\begin{itemize}
    \item \underline{Silicio}, Germanio o anche di leghe di materiali della 3\textsuperscript{a} e 5\textsuperscript{a} colonna p
    \item GaAs, Arsenuro di gallio

        InP solfuro di indio
\end{itemize}

Il \underline{silicio} \'e il materiale pi\'u diffuso in natura. Per poterlo utilizzare bisogna lavorarlo in maniera complessa
\begin{itemize}

    \item Gli atomi di silicio si legano tra di loro formando un reticolo regolare. Gli elettroni di valenza (quelli esterni) formano un legame covalente cio\'e gli atomi condividono una coppia di elettroni costituendo un legame forte.
    \item Il silicio ha 4 elettroni di valenza disponibili a stabilire un legame formando una struttura cristallina (reticolo cubico).

        Ogni nucleo ha 4 legami covalenti. Gli elettroni di legame sono quelli pi\'u lontani dal nucleo e quindi con una maggiore energia cinetica ed una minore energia di legame $\rightarrow$ sono nella banda di valenza (occupata dagli elettroni di valenza che fromano i legami con gli altri atomi)
\end{itemize}

In condizione di perturbazione (termica o ohmica) un elettrone pu\'o ricevere energia saltando nella banda di conduzione $\rightarrow$ entra in un livello energetico superiore e quindi ha un orbita pi\'u grande e un minor legame con il nucleo $\rightarrow$ ha bisogno di un ulteriore piccola energia per cambiare il suo nuovo stato orbitale.

In presenza di un campo elettrico esterno ($\overrightarrow{F}=q\overrightarrow{E}$) l' elettrone riesce ad allontanarsi dall'atomo

Il moto dell'elettrone \'e ostacolato dal reticolo cristallino

L'elettrone che dalla banda di valenza entra nella banda di conduzione \'e debolmente legato al suo nucleo e sottto ad una forza esterna (campo elettrico) pu\' facilmente cambiare il suo stato di moto $\rightarrow$ \textbf{elettrone libero} $\rightarrow$ pu\' contribuire alla corrente.

Per il principio di conservazione dalla carica, quando l'elettrone si allontana globalmente la carica totale \'e equilibrata ma localmente no $\rightarrow$ l'assenza di un elettrone comporta uno sbilanciamento di carica positiva $\rightarrow$ il legame vagante \'e il livello lasciato lbero dall'elettrone che \'e andato nella banda di conduzione.

Questo livello livero pu\'o essere facilmente riempito a spese di un picco di energia


\subsubsection{Microscopicamente accade che}
\begin{enumerate}
    \item un elettrone si mjuove in banda di conduzione
    \item tanti elettroni si muovono in piccoli scostamenti (nella banda di valenza) provocando un apparente spostamenteo di una carica positiva nella direzione opposta della carica negativa in banda di conduzione
\end{enumerate}

i meccanismi di conduzione nei semiconduttori sono diverse dal meccanismo di conduzione nei conduttori

\begin{itemize}
    \item \textbf{Nella banda di conduzione}: il \underline{moto} \'e dovuto dallo svincolamento di una \underline{carica negativa} dal nucleo. (livera di muoversi sotto l'effetto di un campo elettrico)

    \item \textbf{Nella banda di valenza}: il \underline{moto} \'e dovuto ad una successione dei piccoli movimenti di elettroni quali corrisponde microscopicamente un moto in direzione contraria di una \underline{carica positiva}


    \item \textbf{Lacuna} (= assenza di carica negativa) corrisponde ad un moto in cascata di diversi elettroni che rappresenta macroscopicamente un moto di un unica particella fittizia dotata di carica positiva

    \item \textbf{Elettrone} viene considerato soltanto l'elettrone livero di muoversi nella banda di conduzioneo
\end{itemize}

In un \textbf{conduttore} esiste un'unica specie di portatori (trasferimento di elettroni)

In un \textbf{semiconduttore} esistono due meccanismi di conduzione $\rightarrow$ carica negativa (elettrone) e carica positiva (lacuna) $\rightarrow$ esistono due specie di portatori di carica

La generazione di un elettrone provoca la nascita di una lacuna

$n$ = n\textsuperscript{o} elettroni per cm\textsuperscript{3}

$p$ = quantit\'a di lacune per cm\textsuperscript{3}

$n=p$ perche il materiale \'e uniforme (Silicio Puro)

\begin{itemize}
    \item L'evento di sumltanea generazione di un elettrone libero o della corrispondente lacuna si chiama \textbf{evento di generazione} di una coppia elettrone-lacuna (coppia $e$-$h$ dove $h$=hole)

        Pu\'o essere provocato sia in maniera elettrica (campo elettrico) che per via ottica (grazie a due fotoni come nelle macchine fotografiche digitali) che per via acustica/meccanica
    \item ci deve essere il fenomeno opposto alla generazione in quanto il sistema \'e costante nel tempo cio\'e l'alterazione pu\'o cedere l'energia in eccesso che aveva liberato e ritornare nella banda di valenza. L'elettrone occupa una lacuna. $\rightarrow$ effetto di ricombinazione

        L'energia viene ceduta sottoforma di di calore o radiazione luminosa (LED)
\end{itemize}

Sapendo che $q=1.602\times10^-14 C$ per conoscere la densit\'a di carica
\begin{itemize}
    \item degli elettroni liberi $\rightarrow$ $-q\times n$
    \item delle lacune $\rightarrow$ $+q \times n$
\end{itemize}

\begin{multicols}{2}
    \textbf{Evento di generazione di una coppia elettrone-lacuna}

    \textbf{$G$ = tasso di generazione}

    \begin{itemize}
        \item \'e il numero di coppie elettrone-lacuna che vengono generati nell'unit\'a di volume e nell'unit\'a di tempo (velocit\'a di generazione)
        \item dipende dalla temperatura

        \item l'elettrone, ricevendo energia si alloontana dal legame covalente creando una lacuna interpretata come una carica positiva in eccesso
    \end{itemize}


    \columnbreak{}
    \textbf{Evento di ricombinazione}

    \textbf{$R$ = tasso di ricombinazione}
    \begin{itemize}
        \item \'e il numero di coppie elettrone-lacuna che si ricombinano nell'unit\'a di volume e nell'unit\'a di tempo
        \item dipende dalla temperatura e dai meccanismi di ricominazione.
        \item un elettrone libero incontra una lacuna, cede energia e occupa lo stato libero di legame.
            L'energia pu\'o essere ceduta sottoforma di energia termica o radiazione ottica
    \end{itemize}

\end{multicols}

Siccome il sistema \'e in una condizione stazionaria $G=R$ perch\'e la velocit\'a con cui vengono generate le coppie \'e uguale a quella con cui vengono ricombinate $\rightarrow$ il tempo di ``sopravvivenza'' della coppia si chiama tempo di vita della coppia

Siccome in condizioni stazionarie $p=n$ possono avere lo stesso nome

$n\textsubscript{i}(T)$ = concentrazione intrinseca di portatori (lacune o elettroni); dipenda in prima approssimazione dalla temperatura (in modo esponenziale)

Valore tipico $n\textsubscript{i}=10^{10} [\frac{coppie}{cm^3}]$ a 300$K$, i.e. 27\textsuperscript{o}C

Densit\'a degli atomi del reticolo cristallino di silicio = $10^{23} [\frac{atomi}{cm^3}]$; \'e estremamente pi\'u grande di $n\textsubscript{i}$ che \'e per\'o sufficente a cambiare completamente il comportamento del dispositivo.

Per poter sfruttare la presenza di lacune ed elettroni bisogna sbilanciare l'equazione $p=n$ cio\'e non bisogna utilizzare silicio puro.

\section{Materiali Estrinseci}
\subsubsection{Atomi Donatori (5\textsuperscript{a} colonna)}

Si introduce nel reticolo cristallino di silicio una piccola quantit\'a di atomi della 5\textsuperscript{a} colonna della tavola periodica degli elementi (ex. Fosforo)

\begin{itemize}
    \item Questi atomi hanno 5 elettroni di valenza
    \item La piccola quantit\'a non influenza la struttura del reticolo
\end{itemize}

L'atomo di Fosforo \'e un atomo drogante e non forma una lega col silicio in quanto la proporzione \'e incongrua: il numero di atomi inseriti deve essere talmente piccolo da non portare perturbazione alla struttura reticolare

\textbf{atomo di Fosforo} $\rightarrow$ quattro elettroni di valenza formano un legame covalente col silicio.

Il \textbf{quinto elettrone} non forma legami, \'e debolmente legato al nucleo e si allontana da esso non crea una lacuna (=posto vacante in un legame covalente).
La carica positiva che si forma (che bilancia la carica negativa che si \'e allontanata) \'e un protone del nucleo, \'e una carica fissa $\rightarrow$ \textbf{il fosforo si ionizza positivamente}

La carica positiva non pu\'o spostarsi, come faceva la lacuna, in quanto nessun legame vacante ha un ``buco'' $\rightarrow$ \textbf{si genera un elettrone ma non una lacuna}

Dal punto di vista energetico la struttura a banda cambio:
l'atomo di fosforo introduce dei nuovi livelli caratteristici della sua natura
\begin{itemize}
    \item la banda proibita per il silicio cambia grazie all'introduzione di qualche molecola di fosforo che aggiungono qualche livello energetico in prossimit\'a della banda di conduzione
    \item l'elettrone che occupa il nuovo livello pu\'o facilmente ``saltare'' nella banda di conduzione senza creare una lacuna nella banda di valenza

\end{itemize}

Il Fosforo \'e un atomo donatore perch\'e ``regala'' un elettrone libero senza modificare la struttura.

N\textsubscript{0} = concentrazione di atomi donatori = $\frac{n\textsuperscript{o}\,di\,atomi\, donatori}{cm^3}$

Siccome la struttura del cristallo deve essere quella del silicio N\textsubscript{0} $<$ densit\'a diegli atomi del reticolo cristallino di silicio e siccome il n\textsuperscript{o} di elettroni \'e spontaneamente n\textsubscript{i} allora
\[
    10^{10} < N\textsubscript{0} < 10^{20}
\]

Quando $N\textsubscript{0}=10^{20}$ non riesce a modificare la struttura del silicio

Siccome l'energia utile per far ``saltare'' l'elettrone dal livello donatore alla banda di conduzione \'e molto piccola alla temperatura ambiente, tutti gli elettroni di quel livello sono gi\'a nella banda di conduzione, quindi $N\textsubscript{0} \approx n$  (cio\'e ogni atomo donatore cede un elettrone, approssimazione quasi esatta)
\[
    N\textsubscript{D} \approx n > p
\] con atomi donatori (5\textsuperscript{a} colonna)

\subsection{Atomi accettori (3\textsuperscript{a} colonna)}
Si introduce nel reticolo cristallino di silicio una piccola quantit\'a di atomi della 3\textsuperscript{a} colonna della tavola periodica degli elementi (BORO)
\begin{itemize}
    \item Questi atomi hanno 3 elettroni di valenza
    \item la piccola quantit\'a non influisce la struttura del reticolo
\end{itemize}

Il legame valente comporta lo spostamento di un elettrone al quale \'e associato ad uno spostamento di una carica virtuale positiva

\textbf{Atomo di Boro}
\begin{itemize}
    \item si genera una lacuna ma non un elettrone mobile
    \item si crea una carica fissa negativa (ione di Boro negativo)
\end{itemize}

Dal punto di vista di energetico il Boro introduce un nuovo livello energetico (livello accettore), vicino alla banda di valenza
\begin{itemize}
    \item questo livello \'e uno stato legato ad una positione cio\'e gli elettroni nel nuovo livello non si spostano
    \item questo nuovo livello ``cattura'' facilmente un elettrone a spese di una piccola energia
    \item la popolazione di lacune \'e maggiore
\end{itemize}

$N\textsubscript{A}$ = concentrazione di atomi accettori = $\frac{n\textsuperscript{o}\,atomi\,accettori}{cm^3}$

\[
    10^{10} < N_{A} < 10^{20}
\]
$N\textsubscript{A} \approx p > n$ con atomi accettori (3\textsuperscript{a} colonna)

\begin{itemize}
    \item Per un materiale \textbf{intrinseco} (solo silicio puro)

        ci sono soo le coppie elettrone-lacuna

        $p=n=n\textsubscript{i}\quad$ e $\quad p \times n = n\textsubscript{i}^2$ $\qquad$ $N\textsubscript{D} = N\textsubscript{A} = 0$

    \item Per un materiale \textbf{estrinseco} (drogato con atomi accettori o donatori)

        $p \neq n \neq n\textsubscript{i} \quad$ e $\quad p \times n = n\textsubscript{i}$

        Densit\'a di carica $\Rightarrow$ \'e dovuta da
        \begin{itemize}
            \item un elettrone libero
            \item una lacuna
            \item atomi donatore inoizzati positivamente perch\'e hanno ceduto un elettone
            \item atomi accettore ionizzati completamente a temperatura ambiete negativamente
        \end{itemize}

        Solo la carica mobile pu\'o produrre corrente

        $p = N\textsubscript{D} \times q - N\textsubscript{A} \times q - n \times q + p \times q$

        In condizioni di equilibrio sia per materiale intrinseche che estrinseche

        $p \times n = n\textsubscript{i}^2$

    \item \textbf{Bozza di dimostrazione di} $p\times n = n\textsubscript{i}^2$

        In condizione di equilibrio stazionario $G=R$

        G \'e funzione della temperatura $G = \alpha (T)$

        R \'e funzione dia della temperatura, sia del numero di elettroni e di lacune (pi\'u elettroni liberi sono presenti e pi\'u \'e alta la probabilit\'a che troviamo una lacuna)

        $R = \beta (T) \times n \times p$

        $G = \alpha (T) = \beta (T) \times n \times p = R$ $\leftarrow$ questa equazione \'e indipendente dal drogaggio del materiale

\end{itemize}

\[
    n \cdot p = \frac{\alpha(T)}{\beta(T)} = \gamma(T) = n_i^2
\]

In condizione di equilibrio ed ad una temperatura finita $\gamma(T)$ \'e una costante indipendente da $N_D$ e $N_A$ e, siccome per $N_D = N_A$, $n \cdot p = n_i^2$, allora $\gamma(T)$ \'e uguale a $n_i^2$, in quanto non varia al variare del drogaggio

$\Rightarrow$ Siccome $ n \cdot p = n_i^2 = \text{cost}$ \underline{in condizioni stazionarie}, aumentando la concentrazione di elettroni, la concentrazione di lacune cala, in quanto: $n = \frac{N_i^2}{p}$

\subsection{Calcolo di $n(N_D, N_A)$ all' equilibrio, con $N_D$ ed $N_A$ costanti}
\begin{itemize}
    \item $N_D$ e $N_A$ non sono variabili nello spazio
    \item A livello globale la carica \'e nulla
\end{itemize}

Siccome il materiale \'e uniforme, la carica locale \'e nulla $\Rightarrow \rho = 0$

\[
\begin{cases*}
    \rho = q(N_D - N_A -n + p) = 0\\
    p = \frac{N_i^2}{n} \rightarrow \text{Perch\'e $n$ \'e all'equilibrio}
\end{cases*}
\]

\[
    N_D - N_A + \frac{n_i^2}{n} - n = 0; \qquad n(N_D - N_A) + n_i^2 - n^2 = 0
\]
\[
    n^2 - n(N_D - N_A) - n_i^2 = 0 \qquad n = \frac{N_D - N_A}{2} \pm \sqrt{\frac{(N_D - N_A)^2}{4} + n_i^2}
\]

Siccome $n$ \'e positivo per definizione:

\begin{center}
    \framebox{$n = \frac{N_D - N_A}{2} +\sqrt{\frac{(N_D - N_A)^2}{4} + n_i^2}$}\\
    \vspace{10px}
    \framebox{$p = -\frac{N_D - N_A}{2} +\sqrt{\frac{(N_D - N_A)^2}{4} + n_i^2}$}
\end{center}

\'E importante la differenza tra $N_D$ e $N_A$ e non il loro valore:
\begin{itemize}
    \item Principio di complementazione
    \item Bisogna rimanere entro certi limiti (non bisogna alterare la struttura cristallina del silicio)
\end{itemize}

Dal punto di vista energetico se c'\'e un atomo accettore e uno donatore:
% placeholder

Questo \'e utile per la realizzazione diversamente drogante:
Prima si droga con un tipo di atomo tutto il materiale e, dove serve, si aggiunge l'altro atomo drogante.

\begin{itemize}
    \item con $N_D = N_A$ allora $ n = p = n_i$
    \item con $N_D \gg N_A$ e $N_D \gg n_i  \quad n = \frac{(N_D - \cancel{N_A})}{2} + \sqrt{(\frac{N_D - \cancel{N_A}}{2})^2 + \cancel{n_i^2}} \rightarrow n \approx N_D$
    \item con $N_A \gg N_D$ e $N_A \gg n_i  \quad p = -\frac{(\cancel{N_D} - N_A)}{2} + \sqrt{(\frac{\cancel{N_D} - N_A}{2})^2 + \cancel{n_i^2}} \rightarrow p \approx N_A$
\end{itemize}

Gli atomi donatori incrementano la popolazione di elettroni e decrementano la popolaizone di lacune

\begin{minipage}[t]{0.45\textwidth}
    \textbf{Materiale estrinseco di tipo N}
    \begin{itemize}
        \item Il silicio \'e drogato con atomi donatori con concentrazione superiore a quella degli atomi accettori
        \item predominano gli elettroni rispetto alle lacune
    \end{itemize}
    ($N_D$ da $10^{14}$ a $10^{20} cm^{-3}$)

    \begin{center}
        \framebox{\Longstack[c]{$N_D \gg N_A, ni$ \\ $n \approx N_D \quad\text{e}\quad p = \frac{n^2}{N_D}$}}
    \end{center}
\end{minipage}
\begin{minipage}[t]{0.5\textwidth}
    \textbf{Materiale estrinseco di tipo P}
    \begin{itemize}
        \item Il silicio \'e drogato con atomi accettori con concentrazione superiore a quella degli atomi donatori
        \item Predominano le lacune rispetto agli elettroni
    \end{itemize}
    \begin{center}
        \framebox{\Longstack[c]{$N_A \gg N_D, n_i$ \\ $p \approx N_A \quad \text{e}\quad n = \frac{n_i^2}{N_A}$}}
    \end{center}
\end{minipage}

\subsection{Effetto di un campo elettrico su un semiconduttore}
Il campo elettrico esercita una forza sulla carica elettrica pari a

$\vec{F} = -q\vec{E}$ (Forza Coulombiana)

$\vec{F} = m\vec{a}$ con $m=9.81\cdot 10^{31}Kg$

$-q\vec{E} - m\vec{a} \quad \Rightarrow \quad \vec{a} = -\frac{q}{m}\vec{E}$

se il campo elettrico \'e uniforme, il moto \'e uniformemente accelerato (accade solo nel vuoto)

L'elettrone all'interno di un materiale urta contro gli altri atomi (Forza attrattiva del nucleo e forza repulsiva deglu altri elettroni)

Il moto dell'elettrone all'interno di un materiale sotto l'effetto di un campo elettroco esterno \'e:



% \section{Diodo}
\begin{center}
\begin{circuitikz}
    \draw(0, 0) to[diode, *-*] (2, 0);
\end{circuitikz}
\end{center}

Polarizzare una giunzione significa applicare una differenza di potenziale tra anodo e catodo

\begin{enumerate}
    \item Equilibrio
        \begin{align*}
            & V = 0\\
            & J_{Diff} = J_{Drift}
        \end{align*}
        Corrente totale nulla

    \item Polarizzazione in Diretta
        \begin{align*}
            & V = V_{AL} > 0 \\
            & J_{Diff} > J_{Drift}
        \end{align*}
        Corrente da n a p,  $I \approx I_S e^{\frac{V}{V_T}}$

    \item Polarizzazione Inversa
        \begin{align*}
            & V = V_{AL} < 0 \\
            & J_{Drift} > J_{Diff}
        \end{align*}
        Corrente da p a n, $ I \approx -I_s$
\end{enumerate}



%\section{Diodi}
\begin{figure}[H]
\begin{circuitikz}
    \draw
        (0, -2) to[open, v>=$V_i$] (0, 0)
        to[diode, v=$V$, o-] (3, 0)
        to[R, v=$V_u$] (3, -2)
        to[short, -o] (0, -2);
\end{circuitikz}
\centering
\caption{\label{diodo_1} Circuito base con diodo}
\end{figure}

\bigbreak \( I = I_S \left(e^{\frac{V}{V_T}} -1\right) \)

Su intervallo limitato $V$ ha una regione sufficentemente piccola, da essere approssimabile con una costante

Analogamente su intervallo limitato $I$ \`e approssimabile con una costante

\textbf{Polarizzazione Diretta}: $V = V\gamma$

\textbf{Polarizzazione Inversa}: $I= 0$


\(
\begin{rcases}
    \begin{cases}
        V = V_\gamma\\
        I > 0
    \end{cases}\\
    \begin{cases}
        I = 0\\
        V < V_\gamma
    \end{cases}
\end{rcases}
\)
Modello a soglia


Svolgimento circuito in figura~\ref{diodo_1}:

\begin{itemize}
\item Diodo Spento

    \(
    \begin{rcases}
    I = 0\\
    V_u = RI
    \end{rcases}
    \)
    $V_u = 0$

    \begin{align*}
        &V < V_\gamma\\
        &V = V_i - \cancel{V_u}
    \end{align*}

\item Diodo Acceso

    \(
    \begin{rcases}
        V = V_\gamma\\
        V_u = V_i - V
    \end{rcases}
    \)
    $V_u = V_f - V_\gamma$
\end{itemize}

\begin{tikzpicture}
\draw [->](-1, 0) -- (5, 0);
\draw [->](0, -0.5) -- (0, 3);
\draw [thick, color=red](-1, 0) -- (1, 0)
    -- (5, 2);
\end{tikzpicture}

\subsection{Esercizio}

\begin{circuitikz}
    \draw (0, 0) to[R] (3, 0)
            to[diode] (3, -1)
            to[american voltage source, invert] (3, -2)
            -- (0, -2)
            to[open, v^>=$V_i$] (0, 0);

    \draw (3.5, 0) to[open, v^=$V_u$, anchor=west] (3.5, -2);
\end{circuitikz}

\begin{enumerate}
\item $V_u = E+C$
\item $I = \frac{V_i - V_u}{R}$
\end{enumerate}

\begin{itemize}
    \item Diodo OFF

        \( V < V_\gamma \)

        \(
        \text{\textcircled{2}}
        \begin{cases}
        I = 0\\
        V_i - V_u = 0
        \end{cases}
        \Rightarrow
        \begin{cases}
        I = 0\\
        V_u = V_i
        \end{cases}
        \)

    \item Diodo ON

    \(
    \begin{cases}
        V = V_\gamma\\
        V_u = E + V
    \end{cases}
    \Rightarrow
    \begin{cases}
        V= V_\gamma\\
        V_u = E + V_\gamma
    \end{cases}
    \)

    \( V_u + V_\gamma = E \)
\end{itemize}

\begin{tikzpicture}
\draw [->](-1, 0) -- (5, 0);
\draw [->](0, -0.5) -- (0, 3);
\draw [thick, color=red](-0.5, -1) -- (1, 2)
    -- (5, 2);
\end{tikzpicture}

\subsection{Esercizio}


\begin{circuitikz}
    \draw (0, 0) to[R] (3, 0)
            to[diode, invert] (3, -1)
            to[american voltage source] (3, -2)
            -- (0, -2)
            to[open, v^>=$V_i$] (0, 0);

    \draw (3.5, 0) to[open, v^=$V_u$, anchor=west] (3.5, -2);
\end{circuitikz}

\( I = -\frac{V_i - V_u}{R} \)

\( V_u = -E - V \)

\begin{itemize}
    \item Diodo Acceso

        \(
        \begin{cases}
            I = 0 \\
            V_u = V_i\\
            V_u > 0
        \end{cases}
        \Rightarrow
        \begin{cases}
            V_u > -E - V_\gamma\\
            V_i > - E - V_\gamma
        \end{cases}
        \)
    \item Diodo Spento

        \(
        \begin{cases}
            V = V_\gamma\\
            V_u = -E - V_\gamma
        \end{cases}
        \Rightarrow
        E = V_u - V_\gamma
        \)

        \begin{flalign*}
            &I > 0 &&\\\
            &\frac{V_u - V_i}{R} > 0\\
            &V_i < V_u = -E - V_\gamma
        \end{flalign*}
\end{itemize}

\begin{tikzpicture}
\draw [->](-1, 0) -- (5, 0);
\draw [->](0, -0.5) -- (0, 3);
\draw [thick, color=red](-2, -1) -- (-1, -1)
    -- (3, 3);
\end{tikzpicture}

\subsection{Esercizio}

\begin{circuitikz}
    \draw (0, 0) to[short, o-] (3, 0)
        to[american voltage source] (3, 1.5)
    to[diode, invert] (3, 3)
    to[R, -o, l_=$R$] (0, 3)
    to[open, v=$V_i$] (0, 0);

    \draw (3, 0) -- (5, 0)
    to[american voltage source, invert] (5, 1.5)
    to[diode] (5, 3)
    -- (3, 3);
    \draw(5.5, 3) to[open, v^=$V_u$] (5.5, 0);
\end{circuitikz}

\bigbreak
\(
    \begin{cases}
        \text{\textcircled{1}} \quad \frac{V_i - V_u}{R} = I\\
        \text{\textcircled{2}} \quad I + I_2 = I_1\\
        \text{\textcircled{3}} \quad V_u = E_1 + V_{D1}\\
        \text{\textcircled{4}} \quad V_u = - V_{D2} - E_2
    \end{cases}
\)

\begin{itemize}
    \item $D_1$ e $D_2$ \underline{OFF}

        \(
        \begin{cases}
            I_1 = 0\\
            I_2 = 0
        \end{cases} \quad \xrightarrow{\text{\textcircled{2}}}\quad
        I = 0 \quad\rightarrow\quad V_u = V_i
        \)
        \(
        V_{D1} < V_\gamma
        \)

        \textcircled{3}
        \(
        \begin{cases}
            V_{D1} = V_u - E_1\\
            V_u - E_1 < V_\gamma
        \end{cases}
        \Rightarrow
        \begin{cases}
            V_u < E_1 + V_\gamma\\
            V_i < E_1 + V_\gamma
        \end{cases}
        \)

        \( V_{D2} < V_\gamma \)

        \( V_{D2} = -V_U - E_2\)

        \( V_\gamma > - V_u - E_2 \)

        \( V_i > -V_\gamma - E_1\)

    \item $D_1$ Acceso e $D_2$ Spento


\end{itemize}

\begin{tikzpicture}
\draw [->](-1, 0) -- (5, 0);
\draw [->](0, -0.5) -- (0, 3);
\draw [thick, color=red](-2, -1) -- (-1, -1)
    -- (2, 2) -- (5, 2);
\end{tikzpicture}

\section{Appunti di Tue 21 May 2019 02:48:28 PM CEST}

Potenza
\begin{itemize}
    \item Statica = 0
    \item Dinamica

        $\rightarrow$ Cortocircuito

        $\rightarrow$ Carico
\end{itemize}


\begin{figure}[ht]
    \centering
    \includegraphics[width=1.8in]{placeholder.jpg}
    \caption{Grafico 1}
\end{figure}

\begin{samepage}
\[
    I_{D} = \frac{\beta}{2}{\left\{\frac{t}{t_R} V_{DD} - V_{T} \right\}}^2
\]

\[
    V_i(t) = \frac{t_1}{t_r} V_{DD} = V_T \rightarrow t_1 = \frac{V_t}{V_{DD}} t_R
\]

\[
    V_i(t_5) = \frac{t_5}{t_R} V_{DD} = \frac{V_{DD}}{2} \rightarrow t_5 = \frac{t_R}{2}
\]

\[
    \tilde{P} = \frac{1}{T}\int_{0}^{T} V_{DD}I_{D}dt =
    \frac{1}{T} 4 \int_{t_1}^{t_5} V_{DD} \frac{\beta}{2} \left( t \frac{V_{DD}}{t_R} - V_T\right) ^2 dt
\]

\[
    =\frac{4}{T} \frac{\beta V_{DD}}{2} \frac{t_R}{3}
    \left[ \left( t \frac{V_{DD}}{t_R} - V_T\right)^3\right]^{\frac{t_R}{2}}_{\frac{V_T}{V_{DD}} t_R }
\]

\framebox{$\tilde{P_{CC}} = \frac{\beta}{2} \frac{t_R}{T} \left( V_DD - 2 V_T\right)^3$}

% Riprende da ?
\end{samepage}

\begin{figure}[H]
    \centering
    \begin{circuitikz}
        \draw(0, 0) node[eground]{}
        to[Tnmos, n=tnm] ++(0, 1.5)
        to[Tpmos, n=tpm] ++(0, 1.5)
        -- ++(0, 0.2)
        node[vcc]{$V_{CC}$};

        \draw(tnm.G) -- (tpm.G);

        \draw (tnm.D -| tnm.G) to[short, -*] ++(-0.5, 0);
        \draw(tnm.D) -- ++(1, 0)
            to[C] ++(0, -1)
            node[eground]{};

    \end{circuitikz}
    \caption{Circuito 1}
\end{figure}

Applicando Kirkoff: $I_{DP} = I_{Dn} + I_C$

Ricostruita la situazione possiamo analizzare la potenza media associata al carico $\tilde{P_i} = \frac{1}{T}\int_0^T I_{DD} V_{DD} dt $

Sempre dalla relazione di prima, siccome $I_C = 0 \Rightarrow I_{DD} = I_{DP}$


P-mos \'e \textbf{saturo} quando \framebox{$ V_u < V_i + V_t $}

Istantaneamente posso avere un bilancio energetico non nullo, ma in un caso periodico somma dell' energia, per il principio di conservazione deve essere nulla

Sommando le tre potenze devo trovare la potenza complessiva, quindi:
\[
    \tilde{P_L} = \tilde{P_n} + \tilde{P_p} + \tilde{P_C}
\]

\[
    \tilde{P} = \frac{1}{T}\int_0^T  V_U I_C dt = \frac{1}{T} \int_0^T V_U C_L \frac{dV_U}{dt} dt =
    \frac{C_L}{T}\int_{V_U(0)=0}^{V_U(T) = 0} \cdots = 0
\]

Ovvio che \'e nullo perche l'energia di un condensatore \'e legata alla carica

\[
    \tilde{P_P} = \frac{1}{T} \int_0^T V_{BDP} I_{DP} dt = \frac{1}{T}\left[ \int_0^{\frac{T}{2}} V_{BD} I_{DP} dt + \int_{frac{T}{2}}^{2T} V_{SD}I_{DD}dt\right]
\]


\[
    \frac{1}{T} \int_0^{\frac{T}{2}} (V_{DD} - V_U) C_L \frac{dV_U}{dt}dt = -\frac{C_L}{T} \int (V_U - V_{DD}) dV_U = \cdots
\]

\[
    =\frac{C_L}{T} \frac{V_{DD}}{2}^2
\]

Calcolando la potenza consumata dal transitorio di Pull Down (N-mos)
Osservando che nel primo semiperiodo l'n-mos \'e spento, La corrente \'e nulla, quindi ci limitiamo ad integrare nel secondo semiperiodo
\[
    \tilde{P} = \frac{1}{T}\int_0^T  V_U I_C dt = \frac{1}{T} \int_{\frac{T}{2}}^T V_U C_L \frac{dV_U}{dt} dt =
\]

Sapendo che $I_{DD} = -C_L\frac{dV_U}{dt}$ Abbiamo che:

\[
    -\frac{C_L}{T} \int_{V_U(T/2) = V_{DD}} ^ {V_U(T) = 0} V_U \frac{dV_U}{dt} dt = \cdots = \framebox{$\frac{C_L}{T} \frac{V_{DD}^2}{2}$}
\]

Quindi: $\quad \tilde{P_L} = \frac{C_L}{T} {V_{DD}^2} $

Si pu\'o notare che la potenza dissipata non dipende dai parametri dei transitori ($\beta$ non compare nell' espressione)

Siccome quello che mi serve \'e caricare il condendatore, devo spendere un energia doppia rispetto a quella che $\cdots$

Se cambia $\beta$ cambia solo il tempo in cui si carica/scarica il condensatore, $\beta$ non influisce sull'energia

$ \frac{1}{T} = f  \Rightarrow  f C_L {V_{DD}}^2$

Questa che \'e una potenza dinamica, aumenta con la frequenza $\rightarrow$ devo caricare e scaricare il condensatore pi\'u volte

In $P_{CC}$ Compare il rapporto  $\frac{t_R}{T}$, supponendo di essere capaci di far andare pi\'u veloce il circuito, il rapporto tende a rimanere costante $\rightarrow$ Ridurre il periodo aumenta la potenza associata a carica-scarica, ma ha un effetto limitato sulla potenza di cortocircuito

La frequenza sicuramente cresce perche siamo sicuri di riuscire a fare dispositivi pi\'u veloci, ma fare dispositivi piu piccoli, lo scopo della riduzione delle geometrie non \'e fare lo stesso circuito pi\'u piccolo, ma per poter mettere pi\'u ``roba'' all'interno dello stesso chip.

Viene sfruttata la possibilit\'a di mettere pi\'u componenti nello stesso spazio, la frequenza di clock dipende anche dall'architettura e da come \'e fatto il circuito

\begin{figure}[ht]
    \centering
    \includegraphics[width=3in]{img/elettronica/esempio_clock.jpg}
    \caption{Esempio di clock}
\end{figure}

Rete sincrona \'e piu robusta e piu sicura al problema delle Alee, il periodo di clock deve garantire che tutta la rete combinatoria, abbia il tempo di completare il suo lavoro.

La possibilit\'a di integrare circuiti pi\'u complessi viene sfruttata per implementare architetture con maggiori prestazioni, es. Circuiti in Parallelo

Siccome abbiamo scoperto che la frequenza di clock impatta direttamente sulla frequenza associata al carico, e tende ad aumentare con la riduzione delle dimensioni, comporta al fatto che, se tutto il resto rimanesse costante, tutto va pi\'u veloce e consuma di pi\'u perche va piu veloce: Utilizzo architetture con maggiore prestazione a cui \'e associato un maggiore consumo

Circuiti c-mos caratterizzato da un basso consumo di potenza statico, ma un alto consumo dinamico

Differenza fondamentale con la logica RTL era che quella consumava sia che lavorasse, sia che non lavorasse

Logica di tipo \textbf{ratioless}: Le dimensioni del transistore non impattano sulla funzionalit\'a


\section{Lezione del Wed 22 May 2019 03:45:35 PM CEST}
\subsection{Immunit\'a ai disturbi in sistemi analogici}
Principale differenza tra elettronica analogica e digitale \'e  che \'e in grado di distinguere il segnale dal rumore.

\begin{figure}[ht]
    \centering
    \begin{circuitikz}
        \draw(-1, 3) node[nmos] (N1){};
        \draw(1, 3) node[nmos, xscale=-1] (N2){};
        \draw(N1.G) to[short, -o] ++ (-0.2, 0) node[above]{$V_1$};

        \draw (0, 0) node[eground] (G){}
            to[isource] ++ (0, 2) node(bisection){} -- ++ (-1, 0) -- (N1.S);
        \draw (N1.D) to[short, -o] ++ (0.5, 0)
            node[above]{$V_U$};

        \draw(1.2, 1) node[anchor=east]{$\big\uparrow I_0$};
        \draw(N1.D) to[R, -o] ++(0, 2)
            -- ++ (0, 1)
            -- ++ (1, 0)
            node(UB){}
            -- ++ (1, 0)
            -- ++ (0, -1)
            to[R, o-] (N2.D);

        \draw(bisection.center) -- ++ (1,0) -- (N2.S);
        \draw(N2.G) to[short, -o] ++ (0.2, 0) node[above]{$V_2$};

        \draw(UB.center) to[short, ->] ++ (0, 1) node[vdd]{$V_{DD}$};
    \end{circuitikz}
    \caption{Amplificatore Differenziale\label{circuito_1}}
\end{figure}


Osservazioni su Figura~\ref{circuito_1}:
\begin{itemize}
    \item Per Kirkoff $ I_1 + I_2 = I_0$

    \item Siccome la somma delle correnti \'e non nulla, i due transistori ($M_1$  e $M_2$), non possono essere spenti allo stesso tempo

        $I_{D\text{sat}} = \frac{\beta}{2}{(V_{GS}-V_T)}^2$

    \item La caratteristica del generatore di corrente sul piano Tensione-Corrente \'e che \'e costante $\Rightarrow$ la tensione $V_X$ \'e incognita

    \item
        Per Kirkoff:
        $V_1 - V_{GS1} + V_{GS2} - V_{2} = 0 \Rightarrow$
        $V_1 - V_2 = V_{GS1} - V_{GS2}$
\end{itemize}

Supponenedo che per ipotesi: $M_1$ se acceso \'e saturo e $M_2$ Se acceso \'e saturo, e, supponendo che $V_1 = V_2$, allora:

$V_1 - V_2 = 0 \Rightarrow V_{GS2} - V_{GS2} = 0 \Rightarrow V_{GS1} = V_{GS2} $

Siccome la corrente dipende solo da $V_{GS}$, se le tensini sono uguali, le correnti sono uguali $\xrightarrow{SAT} I_1 = I_2$

Siccome $I_1 + I_2 = 0 \Rightarrow I_1 = \frac{I_0}{2} , I_2 = \frac{I_0}{2}$

\[
    V_{U1} = V_{DD} - RI_1 = V_{DD} - R\frac{I_0 }{2}
\]

\[
    V_{U2} = V_{DD} - RI_2 = V_{DD} - R\frac{I_0 }{2}
\]
Il valore di uscita non dipende dall'ingresso, se $V_1$ e $V_2$ sono uguali tra di loro, l'uscita non dipende da loro (Se lavoriamo in regione di saturazione)

Questo tipo di circuito non vede il rumore, siccome entra uguale in entrambi gli ingressi

\vspace{20px}
Supponendo ora che $V_1$ e $V_2$ siano \textbf{diversi}, (es $V_1 > V_2$):

\[
    V_1 - V_2 > 0 \Rightarrow V_{GS1} - V_{GS2} > 0 \Rightarrow V_{GS1} > V_{GS2}
\]
\[
    V_{GS1} > V_{GS2} \Rightarrow I_1 > I_2
\]

Dato che la somma delle correnti \'e limitata ($I_1 + I_2 = I_0$), la corrente $I_1$ continua a crescere, mentre $I_2$ diminuisce, Fino a quando la corrente $I_0$ gira unicamente su un ramo. La corrente non pi\'o diventare negativa perch\'e andrebbe contro le condizioni del transistore.

\begin{figure}[ht]
    \centering
    \includegraphics[width=4in]{img/elettronica/grafico2.jpg}
    \caption{Grafico $I_0$, $V_{id}$\label{grafico_analog}}
\end{figure}

Chiamata $V_{iD}$ la tensione differenziale, posso tracciare l'andamento di $I_1$ (Figura~\ref{grafico_analog}), il quale satura al valore $I_0$

\begin{minipage}[t]{0.45\textwidth}
\[
    \begin{cases}
    V_{u1} = V_{DD} - RI_1\\
    V_{u2} = V_{DD} - RI_2
    \end{cases}
\]
\end{minipage}
\begin{minipage}[c]{0.5\textwidth}
    % Grafico
    \begin{center}
        \includegraphics[width=2.8in]{img/elettronica/grafico3.jpg}
    \end{center}
\end{minipage}

Comportamento radicalmente diverso se i segnali variano simultaneamente o non; Quelle non simultanee vengono amplificate

Segnale d'ingresso di modo differenziale
\[
    \begin{cases}
        V_{id} = V_1 - V_2\\
        V_{ic} = \frac{V_1 + V_2}{2}
    \end{cases} \Rightarrow
    \begin{cases}
        V_1 = V_{ic} + \frac{V_{id}}{2} \\
        V_2 = V_{ic} + \frac{V_{id}}{2} - V_{id} = V_{ic} - \frac{V_{id}}{2}
    \end{cases}
\]

Qualunque coppia di segnali \'e suddivisiile in un segnale di modo comune ed un segnale differenziale, questo circuito cancella il segnale di modo comune, mentre il segnale di modo differenziato viene amplificato.

\'E utile per distinguere quale segnale contribuisce all'uscita

\begin{figure}[ht]
\begin{circuitikz}
    \draw(0, 0) node[nmos](Q){};
    \draw(0, -1) node[eground](G){};
    \draw(Q.G) to[short, -o] ++(-0.5, 0) node[above]{$V_i$};

    \draw(Q.D) to[R, -o] ++(0, 2)
        (Q.S) to[-] (G.center);
    \draw(Q.D) to[short, -o] ++(1,0)
        node[above]{$V_u$};
\end{circuitikz}
\end{figure}

Se mettessi in ingresso due volte lo stesso segnale all'amplificatore differenziale (Figura~\ref{circuito_1}):
\[
    V_{ic} = \frac{V_{i0} + \cancel{V_m \sin wt} + V_{i0} - \cancel{V_m \sin wt} - V_{i0}}{2} = V_{i0}
\]

\[
    V_m = \cancel{V_{i0}} + V_m \sin wt - \cancel{V_{i0}} + V_m \sin wt = 2 V_m \sin wt
\]

\subsection{Generatore di corrente}

\begin{figure}[ht]
    \centering
    \includegraphics[width=4in]{img/elettronica/grafico1.jpg}
    \caption{Grafico con linee orizzontali}
\end{figure}

Un transistore che lavora in saturazione, genera un uscita costante

Applicando una tensione costante, il transisotre \'e saturo e la corrente vale $I_D = \frac{\beta}{2}(V_{GG} - V_T)^2$

Per tensioni sufficentemente grandi: $V_{GS} < V_{DS} + V_T \Rightarrow V_x > V_{GG} - V_T$ lavora in saturazione




\begin{figure}[H]
    \begin{circuitikz}
        \draw node[ground](G){}
            to[R, -*, l=$R_2$] ++(0, 2) node[anchor=east](VGG){$V_{GG}$}
            to[R, -*, l=$R_1$] ++(0, 3) node[anchor=east]{$V_{DD}$} ;
        \draw (VGG) -- +(1.5, 0) node[nmos, anchor=G](N){};
        \draw(N.center) node[anchor=west]{$\Big\downarrow I_D$};
        \draw (N.D) to[short, -*] ++(0, 0.5) node[anchor=west]{$V_x$};
        \draw (N.S) -- ++(0, -0.2) node[eground]{};

    \end{circuitikz}
    \centering
    %\includegraphics[width=1.8in]{placeholder.jpg}
    \caption{Disegno transitore allungato}
\end{figure}

Per partitore di tensione:
\[
    V_{GG} = V_{DD} \frac{R_1}{R_1+R_2}
    = V_{DD} \frac{1} {1+\frac{R_2}{R_1}} \leftarrow \text{Compare ancora il rapporto tra fattori di forma}
\]

\begin{minipage}{0.45\textwidth}
\begin{figure}[H]
    \centering
    \begin{circuitikz}
        \draw node[ground](G){}
            node[nmos, anchor=D, yscale=-1, xscale=-1](M2){}
            to[] (M2.S) to[R, -*, l=$I_D$] ++(0, 2);


        \draw (M2.G) -- +(1, 0) node[nmos, anchor=G](N){};
        \draw(N.center) node[anchor=west]{$\Big\downarrow I_D$};
        \draw (N.D) to[short, -*] ++(0, 0.5) node[anchor=west]{$V_x$};
        \draw (N.S) -- ++(0, -0.2) node[eground]{};

        \draw (M2.S) to[short, *-] ++(1.5, 0)
            coordinate(angle)
            to[short, -*] (angle |- N);
    \end{circuitikz}
    \centering
    \caption{Disegno transitore allungato 2\label{ta_2}}
\end{figure}
\end{minipage}
\begin{minipage}{0.5\textwidth}
    \begin{figure}[H]
        \begin{circuitikz}
            \draw(0, 0) node[eground](G){}
            node[nmos, anchor=S, xscale=-1](M2){};

            \draw(M2.D) node[nmos, anchor=S, xscale=-1](M3){};
            \draw(M3.D) to[short, -*] ++(0, 1)
            node[right]{$V_{DD}$};

            \draw(M3.D) to[short, *-] ++(1, 0)
            coordinate(angle1) to[short, -*] (angle1 |- M3.G) -- (M3.G);

            \draw(M2.D) to[short, *-] ++(1, 0)
            coordinate(angle2)
            to[short, -*] (angle2 |- M2.G);

            \draw(M2.G) -- ++(0.5, 0)
            node[nmos, anchor=G](M1){};
            \draw(M1.S) node[eground]{};
            \draw(M1.D) to[short, -*] ++(0, 0.5) node[right]{$V_u$};
        \end{circuitikz}
        \centering
        \caption{Figura Dubbia\label{ta_3}}
    \end{figure}
\end{minipage}

Avendo connesso il gate del transistore al Drain
$ \Rightarrow V_{GS2} = V_{DS2} \xrightarrow{V_T > 0} V_{GS2} < V_{DS2} + V_t$

$M_2$ SAT
$M_1$ SAT

\[
    \begin{rcases}
        I_{D2} = \frac{\beta}{2}{(V_{GS2} = V_t)}^ 2\\
        I_{D1} = \frac{\beta}{2}{(V_{GS1} = V_t)}^ 2
    \end{rcases}
    \Rightarrow I_{D1} = I_{D2}
\]


% sep
\[
    \begin{cases*}
        I_D = I_{D2}\\
        I_R = \frac{V_{DD} - V_{GS2}}{R}
    \end{cases*}
    \Rightarrow \frac{V_{DD}  - V_{GS2}}{R} = \frac{\beta}{2} (V_{GS2} - V_T)^2
\]

\[
    I_{D3} = \frac{\beta_3}{2}(v_{GS3} = V_t) ^ 2
\]

Da cui:
\[
    \begin{split}
    \frac{\beta_2}{2}(V_{GS2} - V_T) ^2 = \frac{\beta_3}{2}(V_{GS2} - V_T) ^2\\
    \sqrt{\frac{\beta_2}{\cancel{2}}(V_{GS2} - V_T) ^2}= \sqrt{\frac{\beta_3}{\cancel{2}}(V_{GS2} - V_T) ^2}\\
    \cdots
    \end{split}
\]
\[
    V_{GS2} = \frac{V_{DD} + V_T(\theta - 1)}{\theta + 1}\qquad \left(\theta\coloneqq \sqrt{\frac{\beta_2}{\beta_3}}\right)
\]

Ricordiamo che \'e vero solamente se entrambi i transitori lavorano in \textbf{saturazione}:

\begin{center}
$M_1$ Saturo $\Rightarrow \quad V_{GS1} < V_{DS1} + V_T$

$V_1 - \cancel{V_x} < V_{u1} - \cancel{V_x} + V_T \qquad \rightarrow \framebox{$V_{u1} > V_1-V_T$}$

$M_2$ Saturo $\cdots \quad \Rightarrow \framebox{$V_{U2} > V_2-V_T$}$
\end{center}

Siccome voglio che queste due condizioni siano verificate sempre, il prodotto $RI_0$ \'e costante, Fissato $V_{DD}$, se non devo scendere troppo, vuol dire che impone un vincolo sul valore massimo $RI_0$


\section{Appunti  Thu 23 May 2019 02:46:13 PM CEST}
\begin{figure}[ht]
    \centering
    \begin{circuitikz}
        \draw(-1, 3) node[nmos] (N1){};
        \draw(1, 3) node[nmos, xscale=-1] (N2){};
        \draw(N1.G) to[short, -o] ++ (-0.2, 0) node[above]{$V_1$};

        \draw (0, 0) node[eground] (G){}
        to[isource, -*] ++ (0, 2) node(bisection){$V_x$} -- ++ (-1, 0) -- (N1.S);
        \draw (N1.D) to[short, -o] ++ (0.5, 0)
            node[above]{$V_{U1}$};

        \draw(1.2, 1) node[anchor=east]{$\big\uparrow I_0$};
        \draw(N1.D) to[R, -o] ++(0, 2)
            -- ++ (0, 1)
            -- ++ (1, 0)
            node(UB){}
            -- ++ (1, 0)
            -- ++ (0, -1)
            to[R, o-*] (N2.D);

        \draw(bisection.center) -- ++ (1,0) -- (N2.S);
        \draw(N2.G) to[short, -o] ++ (0.2, 0) node[above]{$V_2$};
        \draw(N2.D) node[right]{$V_{U2}$};

        \draw(UB.center) to[short, ->] ++ (0, 1) node[vdd]{$V_{DD}$};
    \end{circuitikz}
    \caption{Amplificatore Differenziale\label{amp_diff}}
\end{figure}


\[
    \begin{cases}
        V_{id} = V_1 - V_2\\
        V_{ic} = \frac{V_1 + V_2}{2}
    \end{cases}
\]

\section{Amplificatore Operazionale}
\begin{minipage}{0.45\textwidth}
\begin{circuitikz}
    \draw(0, 0) node[above]{$V_{U2}$} to[short, *-] (1, 0) node[not port, anchor=in](np){};
    \draw(np.out) to[short, -*] ++(1, 0) node[above] {$V_{U2}^{1}$} ;
\end{circuitikz}
\end{minipage}
\begin{minipage}{0.45\textwidth}
\begin{figure}[H]
    \centering
    \includegraphics[width=1.8in, angle=90]{img/elettronica/grafico_invertitore.jpg}
    \caption{Grafico invertitore}
\end{figure}

\end{minipage}

Caratteristica a 3 rami:
\begin{itemize}
    \item un ramo estremamente ripido
    \item due rami costanti
\end{itemize}


Cambiando il modo comune $V_{ic}$l'uscita differenziale non cambia

Doppio ingresso ed uscita singola, posso  un segnale in ingresso differenziale, $V_id$ ed un segnale in ingresso in modo comune $V_ic$  (Grafico)


\begin{figure}[H]
    \centering
    \begin{circuitikz}
        \draw(0, 0) node[left]{$V_i$}
        to[short, i=$I^+{=}0$, *-] ++(1.5, 0)
        node[op amp, anchor=+, yscale=-1](amp){};
        \draw(0, |-amp.-) node[left]{$V_2$} to[short, *-, i=$I^- {=} 0$] (amp.-);
        \draw (amp.out) to[short, -*] ++(0.2, 0)
        node[above]{$V_u$};
    \end{circuitikz}
    \caption{Amplificatore Operazionale}
\end{figure}




Posso definire la pendenza della retta passante per l'origine come $A_d = \frac{dV_U}{dV_{id}} \rightarrow \infty $

Lo chiamo guadagno differenziale perche \'e il guadagno dell'uscita riferito all'ingresso differenziale

Allostesso modo posso definirea anche $A_c = \frac{dV_U}{dV_{ic}}\rightarrow 0$

Possiamo introdurre un parametro di qualit\'a: \textit{CCMR: Common Mode Rejection Ratio}
\[
    \text{CMMR} = \left| \frac{A_d}{A_c}\right|
\]
Idealmente $CMRR \rightarrow \infty$

Chiamata \textbf{Regione di alto guadagno}($AG$) La retta quasi verticale passante per l'origine abbiamo che

\underline{AG}
\[
    V_{id} = 0
\]
\[
    -V_n < V_n < +V_n
\]

Abbiamo poi due altre regioni, in cui il guadagno \'e nullo, chiamandole rispettivamente $\cdots$

SAT+:
\(
    Vu = V_M
\),
\(
    V_{id} > 0
\)

SAT-:
\(
    Vu = -V_M
\),
\(
    V_{id} < 0
\)

Aggiungendo un criterio di idealit\'a , suppongo che la tensione di uscita $V_u$ non \'e funzione della corrente di uscita $I_u$

Cio\'e $V_u \neq f(I_u)$

$\rightarrow$ Si comporta come generatore di tensione ideale

Le caratteristiche di questo amplificatore sono:
\begin{itemize}
\item corrente di ingresso nulla sui morsetti
\item tensione d'uscita indipendente dalla corrente
    \end{itemize}

    Per ricordare, tra i due ingressi posso immaginare una resistenza che li collega con $R \rightarrow \infty$

    poi  un generatore ideale collegato  a di corrente per $I_u$


\begin{circuitikz}
    \draw(0, 0) node[op amp](am) {};

    \draw(am.-) to[R, l_=$R\rightarrow\infty$, /tikz/circuitikz/bipoles/length=.8cm] (am.+);
    \draw(am.out) -- ++(-0.5, 0)
    to[R, l=$R\rightarrow0$] ++(0, -2)
    to[vsource] ++(0, -1)
    node[eground]{};
    \draw(am.-) to[short, -*] ++(-0.3, 0);
    \draw(am.+) to[short, -*] ++(-0.3, 0);
    \draw(am.out) to[short, -*] ++(+0.3, 0);


\end{circuitikz}

\subsection{Utilizzi}
% altro circuito
Quello che voglio \'e tracciare la caratteristica di trasferimento $V_u$ in funzione di $V_i$

Inizio dalla regione di Auotguadagno:
\[
    \begin{split}
        V_{id} = 0\\
        V_{id} = V^+ - V^- \rightarrow V^- = -V_{id}
    \end{split}
\]

Applicando Kirkoff
\[
    \begin{rcases}
    I_u = \cancel{I^-} + I_2\\
    I_1 = \frac{V_i - \cancel{V^-} }{R_1} = \frac{V_i}{R_i}\\
    I_2 = \frac{V^- - V_u}{R_2}
    \end{rcases}
    \Rightarrow \frac{V_i}{R} = -\frac{V_u}{R_2} \rightarrow V_u = - \frac{R_2}{R_1}V_i
\]

\begin{figure}[H]
    \centering
    \includegraphics[width=1.8in]{placeholder.jpg}
    \caption{Grafico}
\end{figure}
\[
    V_u = \cancel{-}\frac{R_2}{R_1} V_i = \cancel{-}V_M
\]

Nella zona di \underline{SAT+}:
\[
    \begin{split}
    V_u = +V_M\\
    V_{id} > 0 \\
    V_{id} = V^+ - V^-
    \end{split}
\]

\[
    \begin{rcases}
        I_1 = \frac{V_i - V^-}{R_1}\\
        I_2 = \frac{V^- - V_u}{R_2}
    \end{rcases} \Rightarrow \frac{V_i - V^-}{R_1} = \frac{V^- - V_M}{R_2} \Rightarrow \cdots \Rightarrow
V^- = \frac{R_2 V_i + R_1 V_M}{\cancel{R_1 + R_2}} < 0
\]
\[
    R_2 V_i + R_1 V_M < 0 \Rightarrow \framebox{$V_i < -\frac{R_1}{R_2} V_M$}
\]

La retta \'e lineare $\rightarrow$ \'e un buon amplificatore perch\'e non distorce il segnale

Un altro aspetto importante \'e che quella curva non dipende dai parametri operazionali, dipende unicamente dalle resistenze, \'e totalmente indipendente dalla qualit\'a dell amplificatore stesso

\begin{circuitikz}
    \draw (0, 0) node[eground]{}
        to[vsource] ++(0, 2)
        node[above]{$V_i$}
        to[R, -*, l=$R_1$] ++(3, 0)
        coordinate(ing1)
        -- ++(0, 1)
        to[R, l=$R_2$] ++(3, 0)
        coordinate(tip);
        \draw (ing1) node[op amp, anchor=-](am){};
        \draw(am.+) -- ++(0, -1) node[eground]{};
        \draw(tip) -- (tip |- am.out);
        \draw(am.out) to[short, -o] ++(2, 0) node[above]{$V_u$};
        \draw(am.-) to[open, v>=$V_{id}$] (am.+);
\end{circuitikz}

Principio di cortocircuito virtuale: non \'e cortocircuito dal punto di vista della corrente, ma la tensione risulta virtualmente a terra

\underline{AG}
\[
    \begin{rcases}
    \begin{rcases}
        V_{id} = 0\\
        V_{id} = V^+ - V^-
    \end{rcases}
    \Rightarrow V^+ = V^- \rightarrow V^- = V_i\\
    \begin{rcases}
    I_1 = \frac{0 - V^-}{R_1}\\
    I_2 = \frac{V^- - V_U}{R_2}\\
    I_1 = I_2 + \cancel{I^-}
    \end{rcases}
    \end{rcases}\Rightarrow
        \frac{-V_iR}{R_1} = \frac{V_i- V_u}{\cancel{R_2}} \Rightarrow
    V_u = V_i \left(1 + \frac{R_2}{R_1}\right)
\]
\[
    -V_M < V_U < +V_M
\]

\underline{SAT+}
\[
    V_u = +V_M
\]
\bigbreak
\[
    \begin{rcases}
        V_{id} > 0\\
        V_{id} = V^+ - V^-
    \end{rcases}\rightarrow V^+ > V^-
\]
\bigbreak
\[
    \begin{rcases}
    I_1 = \frac{0 - V^-}{R_1}\\
    I_2 = \frac{V^- - V_u}{R_2}
\end{rcases} \rightarrow -\frac{V^-}{R_1} = \frac{V^- - V_M}{R_2} \rightarrow V^-\left(\frac{1}{R_2} + \frac{1}{R_1}\right) = \frac{V_M}{\cancel{R_2}}
\]

\[
    V^- = \frac{R_1}{R_1 + R_2} V_M < V_i
    \Rightarrow \framebox{$V_M < V_i \frac{R_1 + R_2}{R_1}$}
\]
\begin{figure}[H]
    \centering
    \includegraphics[width=4in]{img/elettronica/grafico_amplificatore.jpg}
    \caption{Grafico}
\end{figure}

% \begin{tikzpicture}
%     \draw[thick, ->] (-5, 0) -- (5, 0);
%     \draw[thick, ->] (0, -3) -- (0, +3);
%     \draw (-4, 2) -- (-1, 2) -- (1, -2) -- (4, -2);
%     \draw (-1, 0) node[circle, fill, label =$-\frac{R_1}{R_2}V_{max}$]{};
%     \draw (+1, 0) node[circle, fill, label =$+\frac{R_1}{R_2}V_{max}$]{};
% \end{tikzpicture}

$\Rightarrow$ Tratto orizzontale

\begin{circuitikz}
    \draw (0, 0)
        node[above]{$V_i$}
        to[C, *-*, l=$C$] ++(2, 0)
        coordinate(ing1)
        -- ++(0, 1)
        to[R, l=R] ++(2.5, 0)
        coordinate(tip);
        \draw (ing1) node[op amp, anchor=-](am){};
        \draw(am.+) -- +(0, -1) node[eground]{};
        \draw(am.out) to[short, -*] ++(3, 0)
            node[right] {$V_u$};
        \draw(tip) -- (tip |- am.out);
        \draw(am.-) to[open, v>=$V_{id}$] (am.+);
\end{circuitikz}

\underline{AG}
\[
    V_{id} \rightarrow V^+ - V^- = 0 \rightarrow V^- = 0
\]
\[
    \begin{rcases}
        I_c = \cancel{I^-} + I_R\\
        I_c = C \frac{d(V_i- \cancel{V^-})}{dt}\\
    I_R = \frac{\cancel{V^-} - V_u}{R}
    \end{rcases}
    \rightarrow C\frac{dV_i}{dt} = - \frac{V_u}{R} \rightarrow V_u(t) = -RC \frac{dV_i}{dt}
\]


% Appunti Tue 28 May 2019 02:50:49 PM CEST
\section{Integrazione di un segnale}


\begin{figure}[H]
    \centering
    \begin{circuitikz}
        \draw (0, 0)
        node[above]{$V_i$}
        to[R, *-*, l=$R$, i=$I_R$] ++(2, 0)
        coordinate(ing1)
        -- ++(0, 1)
        to[C, l=$C$, i=$I_C$] ++(2.5, 0)
        coordinate(tip);
        \draw (ing1) node[op amp, anchor=-](am){};
        \draw(am.+) -- +(0, -1) node[eground]{};
        \draw(am.out) to[short, -*] ++(2, 0)
        node[right] {$V_u$};
        \draw(tip) -- (tip |- am.out);
        \draw(am.-) to[open, v>=$V_{id}$] (am.+);
    \end{circuitikz}
    \caption{Circuito integratore invertente}
\end{figure}
In \underline{AG}:

\[ V_{Id} = 0 \rightarrow V^- = V^+ = 0 \]
\[
    \begin{rcases}
        I_R = \frac{V_i - V^- }{R}\\
        I_C = C\frac{d(V^- - V_u}{dt}\\
        -I_r = \cancel{I^-} + I_c
    \end{rcases}\Rightarrow
    \frac{V_i}{R} = -C \frac{dV_u}{dt}
    \Rightarrow \frac{dV_u}{dt} = -\frac{V_i}{RC} \rightarrow \int_0^t \frac{dV_u}{dt} dt = \int_0^t -\frac{Vi(t)}{RC} dt
\]
\[ V_u(t) - V_u(0) = -\frac{1}{RC}\int_0^t V_i(t)dt \]

\begin{center}
    \framebox{$ V_u (t) = V_u(0) - \frac{1}{RC}\int_0^t V_i(t) dt $}
\end{center}

Tutto questo vale finche la tensione di uscita rimane compresa fra $\pm V_M$

A Differenza del derivatore, questo circuito ha memoria

\subsection{}

\begin{figure}[H]
    \centering
    \begin{circuitikz}
        \draw (0, 0)
        node[above]{$V_i$}
        to[R, *-*, l=$R_1$, i=$I_R$] ++(2, 0)
        coordinate(ing1)
        -- ++(0, 1)
        to[R, l=$R_2$, i=$I_R$] ++(2.5, 0)
        coordinate(tip);
        \draw (ing1) node[op amp, anchor=-](am){};
        \draw(am.+) -- +(0, -1) node[]{$V_i$};
        \draw(am.out) to[short, -*] ++(2, 0)
        node[right] {$V_u$};
        \draw(tip) -- (tip |- am.out);
        \draw(am.-) to[open, v>=$V_{id}$] (am.+);
    \end{circuitikz}
    \caption{Circuito integratore invertente}
\end{figure}

\underline{AG}
\[
    \begin{rcases}
        V^- = V_i \\
        I_1 = \frac{0-V^-}{R_1}\\
        I_2 = \frac{V^- - V_u}{R_1}\\
        I_2 = \cancel{I^-} + I_c\\
    \end{rcases}
    \frac{V_i}{R_1} = \frac{V_i - V_u}{R_2}
\]
\[
    -V_i \left(\frac{1}{R_1} + \frac{1}{R_2} \right) = - \frac{V_u}{R_2}
\]
\[ V_u = \left( \frac{1}{R_1} - \frac{1}{R_2} \right) R_2 V_i \]
\[ V_u = \left( 1 + \frac{R_2}{R_1} \right) V_i \xrightarrow{R_2 \to 0} V_u = V_i  \]


\begin{minipage}{0.45\textwidth}
    \begin{circuitikz}
        \draw (0, 0) node[eground]{} to[R, l=$R_1$] (0, 2) to[R, -*, l=$R_L$] (0, 4) node[right]{$V_{DD}$};
        \draw(0, 2) node[anchor=east]{$V_u$}-- (2, 2) to[R, l=$R_L$] (2, 0) node[eground]{};
    \end{circuitikz}
\end{minipage}
\begin{minipage}{0.5\textwidth}
    \begin{circuitikz}
        \draw (0, 0) node[eground](G){} to[R, l=$R_1$] (0, 2) to[R, -*, l=$R_L$] (0, 4) node[right]{$V_{DD}$};
        \draw(0, 2) node[anchor=east]{$V_u$} -- (1, 2) node[op amp, anchor=+](amp){} ;
        \draw (amp.out) to[R, l=$R_L$] (amp.out |- G) node[eground]{};
        \draw (amp.-) -- ++(0, 1) coordinate(angl)-- (angl-|amp.out) -- (amp.out);
    \end{circuitikz}
\end{minipage}

Da Partitore di tensione:
\[ V_u = \frac{R_1}{R_1 R_2} V_{DD}  = \frac{1}{1+\frac{R_2}{R_1}} V_{DD}\]

\[ V_u = \frac{1}{1 + \frac{R_2}{R_1 \parallel R_L}} V_{DD}  \qquad R_1 \parallel R_L < R_1\]


\subsection{Circuito con piu ingressi}


\begin{figure}[H]
    \centering
    \begin{circuitikz}
        \draw (-1, 0)
        node[above]{$V_i$}
        to[R, *-*, l=$R_1$, i=$I_R$] (2, 0)
        coordinate(ing1)
        -- ++(0, 1)
        to[R, l=$R$, i=$I_R$] ++(2.5, 0)
        coordinate(tip);
        \draw (-1, -1) node[above]{$V_2$} to[R, l=$R_2$, *-] (1.5, -1) -- (1.5, 0);
        \draw (ing1) node[op amp, anchor=-](am){};
        \draw(am.+) -- +(0, -1) node[eground]{};
        \draw(am.out) to[short, -*] ++(2, 0)
        node[right] {$V_u$};
        \draw(tip) -- (tip |- am.out);
    \end{circuitikz}
    \caption{Circuito sommatore analogico}
\end{figure}

\underline{AG}
\[ V_id = 0 \rightarrow V^- = V^+ = 0\]
Applicando Kirkoff

\[ \begin{rcases}
        I_1 + I_2 = I^- + I_R\\
        I_1 = \frac{V_1 - \cancel{V^-}}{R_1} \\
        I_2 = \frac{V_2 - \cancel{V^-}}{R_2} \\
        I_R = \frac{V^- - V_u}{R} = -\frac{V_u}{R}\\
    \end{rcases}
    \frac{V_1}{R_1} + \frac{V_2}{R_2} = -\frac{V_u}{R} \rightarrow V_u = -R\left( \frac{V_1}{R_1} + \frac{V_2}{R_2}\right)  \xrightarrow{R_1 = R_2 = R^*} V_u = -\frac{R}{R^*}(V_1 + V_2)
\]

Questo circuito permette di calcolare direttamente in maniera analogica una combinazione lineare di ingressi

Provando a progettare un sommatore di tipo digitale:

(X: Grafico con sommatoria)

\textbf{ADC}: Analog Digital converter

\textbf{DAC}: Digital Analog Converter

Blocco Sommatore: Es $3+6$

\begin{tabular}{c c c c c}
    1 & 1 & \\
    0 & 0 & 1 & 1 & + \\
    0 & 1 & 1 & 0 & = \\
    \hline
    1 & 0 & 0 & 1
\end{tabular}

(X: Grafico sommatore a propagazione di riporto)
(RCA: Ripple Carry Adder)

Dal circuito possiamo distinguere un HA (Half Adder) e 3 FA (Full Adder)

In questo caso aggiungere bit vuol dire aggiungere ritardo, siccome \'e tutto in cascata. Nel caso analogico \'e tutto in parallelo. Qundi aggiungere altri bit non comporta ritardo

\begin{tabular}{c c|c c}
    a & b & $c_H$ & $S_H$\\
    \hline
    0 & 0 & 0 & 1 \\
    0 & 1 & 0 & 1 \\
    1 & 0 & 0 & 1 \\
    1 & 1 & 1 & 0
\end{tabular}

% \begin{circuitikz}
%     \draw (0, 0) node[left](zero){$a$} to[short, *-]  (2, 0) node[and port, anchor=in 1](ch){}
%     \node at (1.5, |- ch.in 2)[jump crossing](X);
%     \draw (0, -2) node[left]{$b$} to[short, *-] (1, -2) -- (1, |- X.west) -- (X.west);
%     \draw (X.east) -- (ch.in 2);
%     \draw (X.north |- zero) -- (X.north);
%     \draw (X.south) -- (X.south |- ,-2) ;
% \end{circuitikz}

$1^a$ mappa k
\begin{center}
    \begin{tabular}{c|c c c c}
    & 00 & 01 & 11 & 10\\
    \hline
        0 & 0 & 0 & 1 & 0\\
        1 & 0 & 1 & 1 & 1\\
    \end{tabular}
\end{center}
\[ C_{out} = ab + C_{in} (a+b) = \]

ricordando che $a + b = ab + \bar{a}b + a\bar{b}$ (Dalla tabella dell'or)
\[ ab \cancel{(a + C_{in})} + C_{in}(a\bar{b} + \bar{a}b) = C_H ++ C_{in} \cdot S_H \]

$2^a$ mappa k
\begin{center}
    \begin{tabular}{c|c c c c}
    & 00 & 01 & 11 & 10\\
    \hline
        0 & 0 & 1 & 0 & 1\\
        1 & 1 & 0 & 1 & 0\\
    \end{tabular}
\end{center}
\[ S = \bar{C_{in}} \bar{a} b + \bar{C_{in}} a \bar{b} + C_{in} \bar{a}\bar{b} + C_{in} ab \]
\[ S = \bar{C_{in}}( a\bar{b} + \bar{a}b) + C_{in}(ab + \bar{a}\bar{b}) = \bar{C_{in}} S_H + C_{in} \bar{S_H} = C_{in} \oplus S_H\]

La porta analogica \'e piu veloce e meno ingombrante di quella digitale, ma non \'e immune al disturbo


% Appunti Wed 29 May 2019 03:40:33 PM CEST

\section{Circuito Sottrattore}
\begin{circuitikz}
    \draw (0, 0) node[left]{$V_i$} to[R, o-] (2, 0)
    node[op amp, anchor=-](amp){};
    \draw (amp.+) to[R] ++(0, -2) node[eground]{};
    \draw (amp.-) -- ++(0, 1) coordinate(ang)
    (ang) to[R]  (amp.out |- ang) -- (amp.out) to[short, -o] ++(1, 0) node[right]{$V_u$};
    \draw (0, |- amp.+) node[left]{$V_2$} to[R, o-] (amp.+);
\end{circuitikz}
\underline{AG}

\[
    \begin{rcases}
        V_{id} = 0 \rightarrow V^+ = V^-\\
        \begin{rcases}
            I_1 = I_2 + \cancel{I^+} \\
            I_1 = \frac{V_2 - V^+}{R} \\
            I_2 = \frac{V^+}{R}
        \end{rcases}
        \rightarrow \frac{V_2 + V^+}{\cancel{R}} = \frac{V^+}{\cancel{R}} \rightarrow V_2 = 2V^+ \rightarrow V^+ = \frac{V_2}{2}
    \end{rcases} \rightarrow
    \begin{rcases}
        V^- = \frac{V_2}{2}\\
        I_3 = I_4 + \cancel{I^-}\\
        I_3 = \frac{V_1 - V^-}{R}\\
        I_4 = \frac{V^- - V_u}{R}
    \end{rcases}
    \frac{V_1 - \frac{V_2}{2}}{\cancel{R}} = \frac{\frac{V_2}{2} - V_u}{\cancel{R}}
\]

\begin{center}
    \framebox{$V_u = V_2 - V_1$}
\end{center}


\section{Amplificatore logaritmico}
\begin{circuitikz}
    \draw (0, 0) node[left]{$V_i$} to[R, o-] (2, 0)
    node[op amp, anchor=-](amp){};
    \draw (amp.+) -- ++(0, -1) node[eground]{};
    \draw (amp.-) -- ++(0, 1) coordinate(ang)
    (ang) to[diode]  (amp.out |- ang) -- (amp.out) to[short, -o] ++(1, 0) node[right]{$V_u$};
\end{circuitikz}
\underline{AG}
\[
    \begin{rcases}
        V_{id} = 0 \rightarrow V^+ = V^- = 0\\
        I_R = I_D + \cancel{I^-}\\
        I_R = \frac{V_i - V^-}{R}
        I_D = IS \left(e^{\frac{(V^- - V_u)}{V_T}} - 1\right)
    \end{rcases}\rightarrow
    \frac{V_i}{RI_s} = \left(e^{\frac{(V^- - V_u)}{V_T}} - 1\right) \rightarrow e^{\frac{(V^- - V_u)}{V_T}} = \frac{V_i}{RI_S} + 1
    \xrightarrow{\ln} \ln\left(e^{-\frac{V_u}{V_T}}\right) = \ln\left( \frac{V_i}{RI_S} + 1\right)
\]
\[ V_u = -V_T \ln\left( \frac{V_i}{RI_S -1} \right) \]
(Grafico: Amplificatore logaritmico)

\section{Amplificatore esponenziale/Antilogaritmico}
\underline{AG}
\[
    \begin{rcases}
        V_{id} = 0 \rightarrow V^+ = V^- = 0\\
        I_D = I_R + \cancel{I^-}\\
        I_D = I_S \left(e^{\frac{V_i - V^-}{V_T}} -1 \right)\\
        I_R = \frac{V^- - V_u}{R}
    \end{rcases} \rightarrow
    I_S \left(e^{\frac{V_i}{V_T}}-1\right) = -\frac{V_u}{R} = -RI_S \left( e^{\frac{V_i}{V_t} -1}\right)
\]

Amplificatore antilogaritmico

Ricordando che
\[ \ln (A+B) = \ln(A) \cdot \ln(B) \]

(X: Circuito Sommatore)

Non \'e un caso che il ramoin retroazione sia sul morsetto negativo:

\begin{minipage}{0.45\textwidth}
    \begin{tikzpicture}
        \begin{axis}[axis lines=middle, ymax=3, ymin=-3, yticklabels=\empty, xticklabels=\empty, xlabel=$V_i$, ylabel=$V_u$]
            \addplot[color=red, domain=-4:-1]{2};
            \addplot[color=red, domain=1:4]{-2};
            \addplot+[mark=none, color=red] coordinates{(-1, 2) (1, -2)};
        \end{axis}
    \end{tikzpicture}
\end{minipage}
\begin{minipage}{0.5\textwidth}
    \begin{tikzpicture}
        \begin{axis}[axis lines=middle, ymax=3, ymin=-3, yticklabels=\empty, xticklabels=\empty, xlabel=$V_i$, ylabel=$V_u$]
            \addplot[color=red, domain=-1:4]{2};
            \addplot[color=red, domain=-4:1]{-2};
            \addplot+[mark=none, color=red] coordinates{(-1, 2) (1, -2)};
        \end{axis}
    \end{tikzpicture}
\end{minipage}

\underline{AG}
\[
    \begin{rcases}
        V_{id} = 0 \rightarrow V^+ = V^- = 0 \\
        I_1 = I_2 + \cancel{I^+}\\
    \end{rcases} \frac{V_i}{R_1} = -\frac{V_u}{R_2} \rightarrow V_u = -\frac{R_2}{R_1}V_i \qquad \text{Per} -V_M < V_u < V_M
\]

\underline{SAT +}

\[
    \begin{rcases}
        \begin{rcases}
            V_{id} > 0, \quad V_u = +V_M\\
            V_{id} = V^+ - V^-
        \end{rcases} \rightarrow
        V^+ > 0\\
        \frac{V_i - V^+ }{R_1} = \frac{V^+ - V_u}{R_2}
    \end{rcases}
    \frac{V_i - V^+}{R_1} = \frac{V^+ - V_M}{R_2}
\]
\[ V^+ \left( \frac{1}{R_2} + \frac{1}{R_1} \right) = \frac{V_i}{R_1} + \frac{V_M}{R_2} \]
\[V^+ \left(\frac{R_1 + R_2}{\cancel{R_1 R_2}}\right) = \frac{V_iR_2 + V_M R_1}{\cancel{R_1 R_2}} \]
\[ V^+ = \frac{R_2V_i + R_1 V_M}{\cancel{R_1 + R_2}} > 0\]
\[ \cancel{R_2} V_i > -\frac{R_1 V_M}{R_2} \]

\underline{SAT -}
\[
    \begin{rcases}
        V_{id} < 0 \rightarrow V^+ - V^- < 0 \rightarrow V^+ < 0\\
        V_u = -V_M
    \end{rcases} \ldots \rightarrow V^+ = \frac{R_2V_i - R_1V_M}{\cancel{R_1 + R_2}} < 0 \Rightarrow \cancel{R_2} V_i < \frac{R_1}{R_2} V_M
\]

Se il guadagno d'anello \'e maggiore di $1$ in modulo si \'e rischio di stabilit\'a

Guadagno amplificatore operazionale $\to \infty$ in alto guadagno
\section{Circuito Bistabile (Circuiti Schmitt trigger)}

\begin{circuitikz}
    \draw (0, 0) node[eground]{} to[R] (2, 0)
    node[op amp, anchor=-](amp){};
    \draw (amp.+) to[short, -o] ++(-0.5, 0) node[left]{$V_i$};
    \draw (amp.-) -- ++(0, 1) coordinate(ang)
    (ang) to[R]  (amp.out |- ang) -- (amp.out) to[short, -o] ++(1, 0) node[right]{$V_u$};
\end{circuitikz}


\underline{AG}
\[
    \begin{rcases}
        V_{id} = 0  \rightarrow V^+ = V^- = V_i\\
        I_1 = I_2 + \cancel{I^+}\\
        \frac{0 - V^+}{R_1} = \frac{V^+ - V_u}{R_L}
    \end{rcases}
    -\frac{V_i}{R_1} = \frac{V_i - V_u}{R_2} \rightarrow -V_i \left( \frac{R_2}{R_1} + 1\right) = V_u \Rightarrow V_u = V_i\left( 1 + \frac{R_1}{R_2}\right)
\]

\[ V_M = V_i^*\left( 1 + \frac{R_2}{R_1}\right) \rightarrow V_i^* = \frac{V_M}{1 + \frac{R_2}{R_1}} \]
\underline{SAT +}
\[
    \begin{rcases}
        \begin{rcases}
            V_u = +V_M\\
            V_{id} > 0 \rightarrow V^+ > V^- = V_i\\
            I_1 = I_2 + \cancel{I^+}\\
            \frac{0 - V^+}{R_1} = \frac{V^+ - V_u}{R_2}
        \end{rcases} \rightarrow -\frac{V^+}{R_1} = \frac{V^+ - V_M}{R_2} \rightarrow -V^+ \left( \frac{1}{R_1} + \frac{1}{R_2} \right)= - \frac{V_M}{R_2}\\
        V^+\left( \frac{R_1 + R_2}{R_1\cancel{R_2}}\right) = \frac{V_M}{R_2} \rightarrow
        V^+ =\frac{R_1}{R_1 + R_2} V_M = \frac{V_M}{1+\frac{R_2}{R_1}}
    \end{rcases}
    V_i < \frac{V_M}{1 + \frac{R_2}{R_1}}
\]
\subsection{Trigger invertente}
\begin{center}
    \begin{tikzpicture}
        \begin{axis}[axis lines=middle, ymax=3, ymin=-3, yticklabels=\empty, xticklabels=\empty, xlabel=$V_i$, ylabel=$V_u$]
            \addplot[color=red, domain=-4:1]{2};
            \addplot[color=red, domain=-1:4]{-2};
            \addplot+[mark=none, color=red] coordinates{(-1, -2) (1, 2)};
        \end{axis}
    \end{tikzpicture}
\end{center}


\subsection{Trigger non invertente}
\begin{center}
    \begin{tikzpicture}
        \begin{axis}[axis lines=middle, ymax=3, ymin=-3, yticklabels=\empty, xticklabels=\empty, xlabel=$V_i$, ylabel=$V_u$]
            \addplot[color=red, domain=-1:4]{2};
            \addplot[color=red, domain=-4:1]{-2};
            \addplot+[mark=none, color=red] coordinates{(-1, 2) (1, -2)};
            \addplot+[mark=none, color=blue] coordinates{(-4, -2) (1, -2) (1, 2) (4, 2)};
            \addplot+[mark=none, color=green] coordinates{(4, 2) (-1, 2) (-1, -2)};
        \end{axis}
    \end{tikzpicture}
\end{center}

Aumentando il valore della tensione di ingresso (da Va) continuo a rimanere nello stesso tratto di curva, senza salti

Se sono al valore basso $-V_M$, per passare al valore alto devo applicare una tensione di ingresso che \'e almeno $V^*$
Se sono al valore alto, per passare al valore basso devo applicare una tensione che \'e minore di $-V^*$

Nella fascia di ambiguit\'a se sono al valore basso rimango al valore basso, se sono a quallo alto rimango in quello alto

Viene percorso un \textbf{Ciclo di isteresi}

Questo circuito \'e Molto pi\'u resistente al rumore rispetto ad un circuito a soglia per la fascia di ambiguit\'a

Applicando prima un segnale molto positivo, e riportando il segnale a 0 l'uscita rimane positiva. L'uscita del valore in 0, dipende dalla `storia' del circuito. Questo circuito si ricorda se l'ultimo valore applicato \'e positivo o negativo (\'e una Cella di memoria).

\'E in grado di mantenere fin tanto che \'e acceso un valore binario in memoria $\rightarrow$ Passaggio da reti combinatorie a reti sequenziali

\newpage
%Appunti del  Thu 30 May 2019 03:01:57 PM CEST
\section{Circuito Astabile}

\begin{circuitikz}

\end{circuitikz}
(X: Circuito 1)

\[
    \begin{split}
        &V_i(0) = 0\\
        &V_u(0) = V_M \rightarrow \text{SAT +?} \rightarrow V_{id} > 0 \text{?}
    \end{split}
\]

La condizione di alto guadagno del circuito abbiamo gi\'a visto che \'e sospetta (instabile), Possiamo presupporre che sia piu facile trovarsi in una delle due altre regioni

Ipotizzo inizialmente che la tensione all'inizio sia $V_M$ (in saturazione positiva)

Posso dire, svolgendo calcoli analoghi ai precedenti:
\[
    \begin{rcases}
        V_i = V^- = 0\\
        V^+ = V_M \frac{R_1}{R_1 + R_2} = \frac{V_M}{1 + \frac{R_2}{R_1}} > 0
    \end{rcases} V_{id} = V^+ - V^- > 0 \rightarrow \text{\'E in saturazione positiva!}
\]

\[
    \begin{rcases}
        I_R =\frac{V_i - V_u}{R}\\
        V_u = V_M\\
        I_C = C\frac{dV_i}{dt}\\
        I_C = - I_R
    \end{rcases}
    C \frac{dV_i}{dt} = - \frac{V_i - V_M}{R}
\]

Siccome $V_i$ innizialmente \'e uguale a 0, $V_M$ \'e positivo, vuol dire che il condensatore si sta caricando
\[
    \frac{dV_i}{V_i - V_M} = -\frac{dt}{RC}
\]

\[
    \int_{V_i(0) = 0}^{V_i(t)} \frac{dV_i}{V_i - V_M} = \int_0^t - \frac{dt}{RC}
\]
\[ -\frac{t}{RC} = -\ln\frac{V_i(t) - V_M}{-V_M}\xrightarrow{\exp} e^{-\frac{t}{RC}} = \frac{V_i(t) - V_M}{-V_M} \]
\[ V_i(t) = V_M - V_M e^{-\frac{t}{RC}} = V_M \left(1 - e^{-\frac{t}{RC}}\right) \]

Questo vale fino a quendo $V_i < Vi^*$

Osservando il condensatore nel momento $V_i = Vi^*$ (Il momento in cui cado nel ramo di saturazione negativa):

\[
    \begin{rcases}
        V_i^* > 0\\
        V_u = -V_M\\
        V^- = V_i\\
        V^+ = -V_M \frac{R_1}{R_2} = -\frac{V_M}{1 + \frac{R_2}{R_1}}\\
    \end{rcases}
        V_{id} = V^+ - V^- < 0 \rightarrow \text{Saturazione negativa}\\
    \]

\[
    \begin{rcases}
        I_R = \frac{V_i - V_u}{R} > 0\\
        I_R = -I_C
    \end{rcases} I_C < 0
\]

\begin{tikzpicture}
    \begin{axis}[axis lines=middle,
        ymax=3, ymin=-3,
        xmax=10,
        yticklabels=\empty, xticklabels=\empty, xlabel=$t$, ylabel=$V_i$]
        \addplot[color=red, domain=0:2]{3*(1-exp(-x))};
    \end{axis}
\end{tikzpicture}
\begin{tikzpicture}
    \begin{axis}[axis lines=middle,
        ymax=3, ymin=-3,
        xmax=10,
        yticklabels=\empty, xticklabels=\empty, xlabel=$t$, ylabel=$V_i$]
        \addplot[color=red, domain=0:2]{2};
        \addplot+[mark=none] coordinates{(2, 2) (2, -2)};
        \addplot[color=red, domain=2:4]{-2};
        \addplot+[mark=none, color=red] coordinates{(4, -2) (4, 2)};
        \addplot[color=red, domain=4:6]{2};
        \addplot+[mark=none, color=red] coordinates{(6, 2) (6, -2)};
        \addplot[color=red, domain=6:8]{-2};
    \end{axis}
\end{tikzpicture}

Continua ad oscillare, la frequenza dipende da $RC$

\begin{center}
\begin{circuitikz}
    \draw (0, 0) node[not port](n1){};
    \draw (n1.out) to[C] ++(0, -1) node[eground]{};
    \draw (n1.out) -- ++(1, 0) node[not port, anchor=in](n2){};
    \draw (n2.out) to[C] ++(0, -1) node[eground]{};
    \draw (n2.out) -- ++(1, 0) node[not port, anchor=in](n3){};
    \draw (n3.out) -- ++(0, -2) -- (n1.in |-, -2) -- (n1.in);
\end{circuitikz}
\end{center}

Attraversando una serie di porte logiche di questo tipo posso rigenerare il segnale (Qualit\'a rigenerativa delle logiche digitali)

Per controllare il ritardo del segnale basta metter in cascata pi\'u invertitori

Per controllare la frequenza dell'onda quadra, basta che aumenti il numero di ritardi? (Domanda Felice)

Dopo un numero pari di inversioni, se in ingresso si ha un valore alto, in uscita avr\'o lo stesso valore $\rightarrow$ Circuito bistabile

(Grafico 2)
\begin{tikzpicture}
    \begin{axis}[axis lines=middle,
        ymax=3, ymin=-3,
        xmax=10,
        yticklabels=\empty, xticklabels=\empty, xlabel=$x$, ylabel=$y$]
        \addplot[color=red, domain=0:1]{2};
        \addplot[color=red, domain=2.5:6]{0};
        \addplot+[mark=none, color=red] (1, 2)  parabola (2.5, 0);
    \end{axis}
\end{tikzpicture}

Questo circuito ha memoria; Se sono in grado di scrivere un valore alto o basso su quel nodo, l'uscita in 0 rimane salvata

(X: Circuito porte 2)

\begin{center}
\begin{circuitikz}
    \draw (0, 0) node[not port](n1){};
    \draw (n1.out) to[C] ++(0, -1) node[eground]{};
    \draw (n1.out) -- ++(1, 0) node[not port, anchor=in](n2){};
    \draw (n2.out) to[C] ++(0, -1) node[eground]{};
    \draw (n2.out) -- ++(1, 0) node[not port, anchor=in](n3){};
    \draw (n3.out) -- ++(0, -2) -- (n1.in |-, -2) -- (n1.in);
\end{circuitikz}
\end{center}
Nel caso l'interruttore sia attaccato: $Q = D$, altrimenti $Q^1 = Q$

Se voglio cambiare il valore memorizzato basta aprire l'anello ed applicare un valore dall'esterno

Utilizzando un segnale di clock al posto del segnale di scrittura, questo circuito alterna nel tempo una fase di scrittura e di lettura.
Circuito \underline{Trasparente}
\[
    \begin{aligned}
        CK &= 1 \rightarrow Q = D \rightarrow \text{Fase di valutazione dell'ingresso, EVALUATION}\\
        CK &= 0 \rightarrow Q^1 = Q \rightarrow \text{Fase di mantenimento, HOLD}
    \end{aligned}
\]

Questo circuito non \'e ancora un \textit{Flip Flop}, ma un \textit{Latch}. Quando siamo a $0$ l'uscita \'e bloccata all'ultimo segnale

(X: Grafici clock)
\begin{tikzpicture}
    \begin{axis}[axis lines=middle,
        ymax=3, ymin=-3,
        xmax=10,
        yticklabels=\empty, xticklabels=\empty, xlabel=$CK$]
        \addplot[color=red, domain=0:2]{2};
        \addplot+[mark=none] coordinates{(2, 2) (2, -2)};
        \addplot[color=red, domain=2:4]{-2};
        \addplot+[mark=none, color=red] coordinates{(4, -2) (4, 2)};
        \addplot[color=red, domain=4:6]{2};
        \addplot+[mark=none, color=red] coordinates{(6, 2) (6, -2)};
        \addplot[color=red, domain=6:8]{-2};
    \end{axis}
\end{tikzpicture}

(X: Altro grafico)
Dall'ultimo grafico si pu\'o vedere che non \'e un comportamento da rete sincron. Quello che posso fare \'e aggiungere a queso oggetto un secondo stadio, passando ad un circuito di tipo Master-Slave

Chiamo questo circuito \textit{p-latch}

(X: Circuiti n-latch, caso duale dell' n-latch)

Per creare un circuito di tipo master-slave basta che metto in cascata un circuito\textit{n-latch} con uno \textit{p-latch}.

(X: Foto 1)
(X: Grafici ambigui dopo foto)

Lo slave legge solo un valore: Quello alla fine dello stato di valutazione del master $\rightarrow$ Il valore dell'uscita pu\'o essere aggiornato solo nei fronti di discesa del clock. $\rightarrow$ \textit{Flip flop} sincrono

(X: Immagine flip flop, (post grefici ambigui))

Per un flip flop Positive triggered (Campionato sul fronte positivo): Scambiare p ed n latch; Prima il P dopo N latch

\subsection{Multiplexer}

(X: Circuito Multiplexer)
\begin{tabular}{c|c c c c}
    s-ab & 00 & 01 & 11 & 00\\
    \hline
    0  & 0 & 1 & 1 & 0  \\
    1 & 0 & 0 & 1 & 1
\end{tabular}
$\Rightarrow y = as + b\bar{s}$


(X: Or ingressi negati)


$ z = \bar{x} + \bar{y} = \bar{xy}$ (NAND)

MUX mi costa 12 transistori +2. Per fare un latch utilizzoaltri 2 transistori $\rightarrow$ Latch utilizza 16 (+2) transistori
Per la configurazione master-slave (\textit{Flip Flop di tipo D}) ne servono altri 16


\newpage
\section{Appunti Tue 04 Jun 2019 02:57:14 PM CEST}

\textbf{Multiplexer} $y = sa + \bar{s}b \Rightarrow \bar{y} = sa + \bar{s}b$

(X: Figura con transistori)


Non \'e un problema se il segnale $y$ torna invertito, lo devo utilizzare invertito nel circuito master-slave ugualmente

Per fare il multiplexer, prima servivano 16 transistori, ora nella figura siamo passati a 10. prima il segnale per andare dall'ingresso all'uscita deve attraversare due strati di nand e due di not, ($\times 2$ con master slave), nel nuovo circuito deve attraversarne solo 2.

\`E possibile implementare il mux anche con due interruttori; L'idea diventa quindi sostituire ai due interruttori due transistori controllari da una corrente (Da 8 a 2 transistori). (\textit{Pass Transistor})

Questo \'e il primo caso dove la corrente d'ingresso $I \neq 0$

\begin{circuitikz}
    \draw (0, 0) node[eground] {}
    to[C] (0, 2);
\end{circuitikz}

\underline{Caso $t < 0$}

M off
\[
    \begin{rcases}
         I_D = 0\\
         I_D = I_c
     \end{rcases} \rightarrow I_c = 0 \rightarrow I_C = C\frac{dV_u}{dt} = 0 \rightarrow V_u = cost
\]

 Se l'ingresso \'e 0, l'uscita pu\'o trovarsi in un uscita in condizioni di alta imedenza.
Il valore si mantiente dentro al condensatore

DRAM: Memoria dinamica associata ad un condensatore
questo circuito ha 3 tipi d'uscita

\begin{enumerate}
    \item alta
    \item bassa
    \item alta impedenza
\end{enumerate}

Nel momento in cui si accende il transistore i casi possibili sono:

\begin{tabular}{c c c}
    $V_i$ & $V_u$ \\
    \hline
    $V_L$ & $V_L$ & -\\
    $V_L$ & $V_H$ & * $V_u : V_H\to 0$\\
    $V_H$ & $V_L$ & ** $V_u: 0 \to V_{DD} - V_T$\\
    $V_H$ & $V_H$ & -\\
\end{tabular}

Studio il primo dei due fenomeni (*):

il nodo $V_u$ ha sempre potenziale piu alto rispetto all'ingresso, di conseguenza \'e il Drain, mentre $V_I$ \'e il source

\[
    \begin{rcases}
        I_D = - I_C\\
        I_D \ge 0\\
        I_C = C\frac{dV_u}{dt}
    \end{rcases}
    C\frac{dV_u}{dt} < 0
\]

Derivata negativa, la cerica del condensatore si sta esaurendo\\
Esattamente quello scritto quando abbiamo calcolato il tempo di scarica del condensatore cmos

Il transitorio termina quando la derivata si annulla $\rightarrow$ la corrente si annulla $\rightarrow V_{DS} = V_u$ si annulla

Tende ad assumere asintoticamente il valore nullo


Studiando il caso (**)
Drain e Source sono invertiti
\[
    \begin{rcases}
    I_D = I_C\\
    I_D \ge 0\\
    I_C = C \frac{dV_u}{dt}
\end{rcases}
C \frac{d V_u}{dt} \ge 0
\]


\[ I_D = \frac{\beta}{2}\left( V_{GS} - V_T\right) ^2 = 0 \]
\[ V_{DD} - V_i \rightarrow V_u = V_{DD} - V_T \]

Questo pass transistor \'e capace di trasferire uno \textit{0 Forte} ed un un \textit{1 Debole}
%%
Supponendo ora di avere lo stesso caso di prima (Alto in ingresso e basso in uscita che voglio portare alto) $\rightarrow$ Transitorio di carica di un invertitore cmos, inizialmente saturo fino a quando la tensione di uscita $> V_{TP} $, poi va in lineare

Quindi sappiamo gi\'a che l'uscita si porta da al valore $V_{DD}$

SAT $\to I_D = \frac{\beta_D}{2}\left(V_{SG} - |V_{TP}|\right)^2 = 0 \rightarrow V_{SG} = V_u = |V_{TP}|$

La rete di pulldown non \'e capace di scaricare completamente il transistore, si ferma una soglia prima

\bigbreak{}
\begin{tabular}{c c}
    $V_u : V_H \to V_L$ & n 0 forte\\
                        & p 0 debole  ($|V_{TP}|)$\\
                        \hline
    $V_u: V_L \to V_H$ & n 1 debole ($V_{DD} - V_T$)\\
                       & p 1 forte\\
\end{tabular}


Metto insieme p ed n mos per migliorare le prestazioni

\[ V_u : 0 \xrightarrow{n SAT\\p SAT} |V_{TP}| \xrightarrow{n SAT\\ p LIN} V_{DD} - V_{TN} \xrightarrow{OFF, LIN} V_{DD} \]
\[ V_u : V_{DD} \xrightarrow{n SAT\\ p SAT} V_{DD} - V_{TN} \xrightarrow{LIN\\SAT} |V_{TP}| \xrightarrow{LIN, OFF} 0 \]

Un oggetto di questo tipo si chima \textit{transmission gate}

Passando attraverso due invertitori, il valore d'uscita viene rigenerato

Questo multiplexer non \'e immune ai disturbi, il meglio che posso sperare \'e che il segnale esce ridotto al massimo di quanto ci aspettiamo

\begin{tabular}{c c|c}
    a & b & y\\
    \hline
    0 & 0 & 0\\
    0 & 1 & 0\\
    1 & 0 & 0\\
    1 & 1 & 1\\
\end{tabular}

\[ y = \begin{cases}
    a \quad \text{se} \, =0\\
        b \quad \text{se} \,a = 1\\
    \end{cases}
\]

Segnale $a$ \'e usato come selettore

Difetto: non abbiamo margine con l'immunit\'a ai disturbi

(X : Circuito and con multiplexer e transistor)

\underline{OR}

\begin{tabular}{c c|c}
    a & b & y\\
    \hline
    0 & 0 & 0\\
    0 & 1 & 1\\
    1 & 0 & 1\\
    1 & 1 & 1\\
\end{tabular}

(X: Circuito or con multiplexer transistor)

Mentre connettendo stadi cmos, la corrente di gate \'e nulla, in questo caso (pass transistor) la corrente che circola non \'e nulla. Il tempo di propagazione dipende dal tempo di carica e scarica delle capacit\'a parassita

Nel pass transistor le capacit\'a parassita sono caricate dalla corrente che entra in ingresso. In questo caso devo caricare una maggiore capacit\'a con a disposizione una corrente piu debole. Auentare il numero degli stadi vuol dire aumentare la capacit\'a da caricare e la resistenza



% Appunti Thu 06 Jun 2019 02:42:48 PM CEST
\newpage
\section{Appunti Thu 06 Jun 2019 02:42:48 PM CEST}

% Scorsa lezione
Bit Line:

Wired line

Meccanismo di indirizzamento attraverso un decoder

Decoder si frammenta in una struttura gerarchica id pi\'u righe e decoder

Rimane da capire cosa c'e' in un incrocio

%

R/W segnale di controllo che specifica lettura e scrittura

Memorie a sola lettura: Piu semplice ed economica rispetto ad una memmoria RW.

Classificazione tra memorie:

\begin{itemize}
    \item ROM: Read Only Memory; Il contenuto rimane invariato, si pone il problema di definire una volta per tutte il contenuto:
        \begin{itemize}
            \item (MP) Rom (Mask Programmable); Programmate in fabbrica

            \item \textbf{FP ROM} (Field Programmable ROM); Programmata dall'utente Genericamente chiamata (PROM: Rom Programmabile)
        \end{itemize}
    \item RWM: Read Write Memory
        \begin{itemize}
            \item Volatile RWM: Perde i dati quando manca l'alimentazione (RAM: Random Access Memory, Memoria ad accesso Arbitrario). Il tempo d'accesso all'informazione non dipende dalla posizione del dato (indirizzo).
                \begin{itemize}
                    \item Dato Memorizzato in maniera Statica (SRAM: Static Ram)
                    \item Dato Memorizzato in maniera Dinamica (DRAM: Dynmic RAM)
                \end{itemize}
            \item Non Volatile RWM: Permanente una volta che \'e stato scritto un dato, esso rimane nel tempo indefinitamente
                \begin{itemize}
                    \item EPROM
                    \item $E^2$PROM
                    \item FLASH EPROM
                    \item $\ldots$
                \end{itemize}
                La sigla ROM indica che l'operazione di lettura e di scrittura sono radicalmente diverse (tempi, tensioni applicate, $ldots$), sono progettate per essere lette frequentemente e programmate o scritte non cosi spesso.

                Chiamate Memorie a Prevalente Lettura (\textit{Read Mostly Memory})
        \end{itemize}
\end{itemize}

Come progettare L'incrocio con la memoria ROM

Ogni colonna di questa matrice \'e implementabile come la tabella di verit\'a alimentata dal decoder.

Per avere l'uscita negata basta che inverta la logica del circuito
% Nor

\textbf{NOR-BASED}

Se sulla tabella di verti\'a leggo uno $\rightarrow$ transistore nella rispettiva \textit{word}? line.
Anzich\'e invertire l'uscita, basta invertire se mettere o no il transistore in ingresso

Per programmare la memoria:

\begin{itemize}
    \item
        Nel caso in cui la programmi il fabbricante, sceglie se mettere o no il transistore, non \'e molto pratico, siccome la costruzione dei transistori \'e una delle prime fasi.

        Quello che si pu\'o fare \'e invece che agire sulle fasi iniziali della progettazione, agiamo sulle fasi finali: Costruisco una matrice, dove i transistori ci sono tutti; Ma siccome ogni transistore \'e collegato alla linea di bit, vuol dire che li devo connettere fisicamente alla linea metallica del segnale della bitline.

        Ovverso scelgo se collegare o no il transistore alla bitline. Se non lo collego, si comporta come se non ci fosse.
        \'E molto piu costoso differenziare i progetti, piuttosto che produrre qualche transistore in pi\'u.

        Rimane vero che il costo di questa operazione \'e elevato, si giustiica solamente per volumi di produzione sufficentemente grandi (MPROM)

    \item Programmazione sul campo, connetto tutti i transistori, attraverso un interconnessione particolare, rappresentabile impropiamente come un fusibile (tratto di circuito a resistenza elevata, se si scalda a sufficenza interrompe il circuito).

        La pista del fusiblile non si fonde quando leggo, ma quando scrivo: Corrente di fusione di programmazione \'e pi\'u alta,

        Fondere dove ho bisogno di un 1 nella tabella di verit\'a

    \item  Un altro metodo \'e scegliere la tensione di soglia $V_T$ dei transistori.
        \'E necessario modificare la tensione di soglia dinamicamente. Per non riportare il problema alla fabbricazione
        Tensione di soglia: Tensione da applicare al gate per far si che si formi sotto al gate un canale conduttivo, che corrisponde ad una certa densit\'a di elettroni.

        Se all'interno dell'ossido \'e intrappolata una carica negativa: Neutralizza una parte della carica positiva di Gate: La tensione di soglia si alza: \'E necessaria pi\'u carica di gate per avere la stessa tensione di soglia

        \textbf{Transistore a Doppio Gate}
\end{itemize}

Transistore a doppio gate: Ha dimensioni estremamente ridotte, quindi per \textit{Effetto tunnel} \'e possibile con probabilit\'a $\neq 0$ che un elettrone entri dentro lo strato isolato. Accelerando i portatori di canale posso fare si che una parte degli elettroni di canale entrino dentor e rimangano intrappolati dentro a questo \textit{Floating Gate}

In questo modo posso modificare la tensione di soglia variando le correnti di canale

Voglio che la velocit\'a degli elettroni (Energia cinetica)  di quando leggo e quando scrivo siano diverse (Pi\'u basse durante la lettura, pi\'u alte durante la programmazione). Segue che la lettura \'e molto pi\'u rapida

La probabilit\'a di effetto tunnel \'e molto ridotta per fare in modo che i dati non vengano variati durante la lettura.

A lungo andare \'e molto probabile che elettroni entrino dentro all'ossido, rimanendo permanentemente intrappolati. \textit{Fenomeno di Invecchiamento del transistore}. Il numero di volte che posso cancellare e programmare la memoria, non \'e infiito.

Vuol dire che posso leggere la memoria, un numero infinito di volte, ma posso scriverla solo un numero finito di volte. Non \'e utilizzabile come memoria centrale per un Computer. (Se ci voglio scrivere con una velocit\'a di 1 $GHz$ sarebbe esaurita in 1 secondo)

Chiamiamo queste memorie EPROM e  EEPROM (\textit{Elecrically Eraseable Programmable Read Only Memory})

\subsection{Memorie Voltatili}

Se connetto la cella alla bit line, la bit line influisce la cella: necessita distinzione tra lettura e scrittura.

Bistabile + Faccio competere con una rete forte se voglio scrivere ed una rete debole se voglio leggere (dalla cella): SRAM 5T(ransistori)

Alternativa: Condensatore isolato (Problema con capacit\'a parassita della bitline) (DRAM 1T), le operazioni di lettura sono molto pi\'u complicate). Si dice che la lettura \'e distruttiva.

Un altro problema \'e che se carico la cella, il condensatore \'e imperfetto, e poco alla volta perde carica. Periodicamente devo rigenerare i valori nelle celle (Operazione di Refresh). Questo rallenta le operazioni

Per questo motivo, le memorie di tipo statico sono significativamente pi\'u veloci delle memorie di tipo dinamico.

\subsection{Conversione ADC/DAC}
Ci limitiamo ad enunciare i principi (come nelle memorie) senza entrare nel dettaglio.

Convertitore ADC, entra un segnale esce una n-upla di bit, \'e implicita una perdita di informazione.

Immaginando che il segnale d'ingresso $V_{in}$ vari da 0 a $V_{SF}$.
Posso rappresentare in funzione dell'ingresso $V_{in}$ la rappresentazione decimale equivalente.

\begin{center}
\begin{tikzpicture}
    \begin{axis}[xmin=0, ymin=0]
        \addplot[color=red]{x};
    \end{axis}
\end{tikzpicture}
\end{center}


Assegno i possibiili valori d'ingresso a rispettivi valori binari per gli $a_0 \ldots a_n$: Processo di Quantizzazione
\[\frac{V_{FS}}{2^n} = Q\]

Se scelgo questo tipo di quantizzazione l'errore massimo \'e $Q$

\begin{center}
\begin{tikzpicture}
    \begin{axis}[xmin=0, ymin=0]
        \addplot[color=red]{x};
    \end{axis}
\end{tikzpicture}
\end{center}

In questo caso, tralasciando, l'ultimo intervallo l'errore massimo che ho \'e $\frac{Q}{2}$. Tutto questo funziona fino a quando non entro negli intervalli segnati.

% op amp
\begin{circuitikz}
    \draw (0, 0) node[op amp](oa){};
    \draw (-1, |- oa.-) node[left]{$t_1$} to[short , *-] (oa.-);

\end{circuitikz}

Serve un circuito in grado di leggere le soglie d'ingresso (Comparatore elementare)
\[ V_id = V_{i0} - t_1 > 0 \to V_{in} > t_1 \to V_u = V_M \]

Siccome devo osservare 3 soglie: 3 Amplificatori operazionali

% 3 op amp
\begin{circuitikz}

\end{circuitikz}
Codice a 3 monitor

($V_{FS}$ Tensione di fondoscala)

\begin{itemize}
    \item Generare i valori di riferimento:
        Siccome entra una corrente nulla nell'operazionale basta un partitore. (Per riprodurre le tensioni basta mettere resistenze proporzionali)
    \item ABC \'e gia un uscita digitale: Convertire il codice termometro in un codice binario

        \begin{minipage}{0.3\textwidth}
            \begin{tabular}{c c c|c c c }
                A & B & C & $a_1$ & $a_0$\\
                \hline
                0 & 0 & 0 & 0 & 0\\
                0 & 0 & 1 & 0 & 1\\
                0 & 1 & 0 & - & -\\
                0 & 1 & 1 & 1 & 0\\
                1 & 0 & 0 & - & -\\
                1 & 0 & 1 & - & -\\
                1 & 1 & 0 & - & -\\
                1 & 1 & 1 & 1 & 1\\
            \end{tabular}
        \end{minipage}
        \begin{minipage}{0.6\textwidth}
        \begin{tabular}{c|c|c|c|c|}
            A/BC & 00 & 01 & 11 & 10\\
            \hline
            0 & 0 & 0& 1 & -\\
            \hline
            1 & - & - & 1 & -\\
            \hline
        \end{tabular} $\Rightarrow a_1 = B$
        \bigbreak
        \begin{tabular}{c|c|c|c|c|}
            A/BC & 00 & 01 & 11 & 10\\
            \hline
            0 & 0 & 1& 0 & -\\
            \hline
            1 & - & - & 1 & -\\
            \hline
        \end{tabular} $\Rightarrow a_0 = A + \bar{B}C$
        \end{minipage}

        Convertitore di tipo Flash
\end{itemize}

Numero decimale che rappresenta l'n-upla di bit posso scriverlo come
\[A = \sum\limits_{i=0}^{n-1} a_i2^i  = 2^n\sum\limits_{i=0}^{n-1} a_i2^{i-n} \]
\[ \sum\limits_{i=0}^{n-1} a_i2^{i-n} \approx 1 \]

\[\sum\limits_{i=0}^{n-1} a_i2^{i-n} V_{FS} = a_{n-1}\frac{V_{FS}}{2} + a_{n-2}\frac{V_FS}{4} \ldots a_0\frac{V_{FS}}{2^n}\]

Ho ricondotto il problema ad una somma di tensioni proporzionale a $V_{FS}$ : Basta un circuito sommatore
% sommatore
\[ V_ = - \left( \frac{R}{R_1} V_1 + \frac{R}{R_2}V_2\right) \]

\[ V_u = a_1 \frac{V_{FS}}{2} + a_0\frac{V_{FS}}{4} \]
Immaginando di applicare il principio di sovrapposizione degli effetti
\[  a_1a_0 = 01\qquad V_u = \cancel{a_0}\frac{V_{FS}}{4} = -\left(\frac{R}{R_1}V_1 + \frac{R}{R_2} V_2\right) \]
\[ R_2 = 4R\frac{V_H}{V_{FS}} \]

Altro caso $a_1a_0 = 10$
\[ V_u = \cancel{a_1} \frac{V_{FS}}{2} + \cancel{a_0 + \frac{V_{FS}}{4}}  = -\frac{R}{R_1} V_H \to
    R_1 = 2R \frac{V_H}{V_{FS}}
\]


\end{document}

