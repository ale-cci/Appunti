\documentclass[../template]{subfiles}

\begin{document}
\section{Amplificatore differenziale}
\begin{center}
    \begin{tikzpicture}
        \draw
            (0, 0) node[ground]{}
            to[european current source, i<=$I_D$] (0, 1.5)
            coordinate(x)
            -- ++(-1, 0)
            node[nmos, anchor=S] (nmos1){}

            (nmos1.D) to[R, l=R, i<=$I_{D1}$] ++(0, 2)
            node[vdd]{}
            (nmos1.G) node[left]{$V_{i1}$}
            (nmos1.D) -- ++(0.2, 0) node[right]{$V_{u1}$}

            (x) -- ++(1, 0)
            node[nmos, anchor=S, xscale=-1] (nmos2){}
            (nmos2.D) to[R, l=R, i<=$I_{D2}$] ++(0, 2)
            node[vdd]{}
            (nmos2.G) node[right]{$V_{i2}$}
            (nmos2.D) -- ++(0.2, 0) node[right]{$V_{u2}$}
            ;
    \end{tikzpicture}
\end{center}

Da Kirkoff $I_{D1} + I_{D2} = I_D$
\[
    V_{i1} - V_{i2} =  V_{GS1} - V_{GS2}
\]
Questo di porta a dire che se $V_{i1} > V_{i2}$, allora $V_{GS1} > V_{GS2}$ e di conseguenza $I_{D1} > I_{D2}$.
E siccome la tensione di uscita è scritta nella forma $V_u = V_{dd} - R I_D$, ne segue che $V_u1 < V_u2$.

Se le due tensioni di ingresso sono identiche, non c'è motivo di pensare che le correnti sui due rami non siano identiche, siccome il circuito è simmetrico. Diversamente se una delle due tensioni è maggiori ci si può aspettare che la corrente maggiore vada dal ramo con tensione maggiore.

Questo succede fino al limite in cui tutta la corrente circola su un unico ramo del circuito.

\begin{figure}[h]
    \begin{tikzpicture}
        \begin{axis}[ylabel={$I_{D1}, I_{D2}$}, xlabel={$V_{i1} - V_{i2}$}]
            \addplot {3 / (1 + e^(-x))};
            \addplot {3-3 / (1 + e^(-x))};
            \addplot[dashed, gray]{3};
            \draw (0, 3) node[circ]{} node[left]{$I_0$};
        \end{axis}
    \end{tikzpicture}
    \begin{tikzpicture}
        \begin{axis}[ylabel={$V_{U1}, V_{U2}$}, xlabel={$V_{i1} - V_{i2}$}]
            \addplot {.5 + 3 / (1 + e^(-x))};
            \addplot {.5 + 3-3 / (1 + e^(-x))};
            \addplot[dashed, gray]{.5};
            \draw (0, .5) node[circ]{} node[below]{$V_{dd} - R I_0$};
            \draw (0, 3.5) node[circ]{} node[above] {$V_{dd}$};
        \end{axis}
    \end{tikzpicture}
\end{figure}

Questo circuito è un amplificatore che varia con la differenza dei segnali d'ingresso.
Chiamando $V_{i1} - V{i2} = V_id$ e $\frac{V_{i1} + V_{i2}}{2} = V_{ic}$, posso sempre determinare i valori $V_{i1}$ e $V_{i2}$ attraverso differenza e valore medio. La media degli ingressi prende il nome di "componente di modo comune", ed è indicata con $V_{ic}$.
È possibile definire un guadagno $A_d = \frac{dV_u}{dV_{id}}$, calcolando lo stesso guadagno per le componenti di modo comune:
Tenendo i due segnali identici in ingresso, si ottiene:
$I_{D1} = I_{D2} = \frac{I_0}{2}$, indipendentemente dal valore di corrente $V_{u1} = V_{u2} = k$. Da cui $A_{c} = \frac{dV_u}{dV_{iC}} = 0$, quindi questo circuito è in grado di amplificare la componente differenziale, ignorando completamente le variazioni in modo comune.

\subsection{Considerazioni su opamp}
\begin{center}
    \begin{circuitikz}
        \draw
        (0,0) node[op amp] (opamp) {}
        (opamp.+) node[left] {$v_+$}
        (opamp.-) node[left] {$v_-$}
        (opamp.out) node[right] {$v_u$}
        (opamp.up) --++(0,0.5) node[vcc]{}
        (opamp.down) --++(0,-0.5) node[ground]{}
        ;
    \end{circuitikz}
\end{center}

È possibile descrivere la relazione in uscita della funzione a gradino in tre tratti differenti, una regione centrale, in cui il guadagno tende all'infinito, descritto dall'equazione di validità $V_{id = 0}$ per $-V_M < V_u < V_M$; un tratto di saturazione positiva, vero per $V_{id} > 0$, dove $V_u = V_M$, ed un tratto di saturazione negativa, $V_u = -V_M$, vero per $V_{id} < 0$.

Per trasformare la funzione in uscita dell'operatore operazionale in una funzione a gradino, occorre amplificare l'uscita corrente e traslarla sul nuovo intervallo. Infatti mettendo l'uscita in ingresso a due invertitori il valore, appena positivo, viene tradotto in un valore positivo alto.

Chiamando le corrente di ingresso sui rami $I^+$ e $I^-$, è facile capire che sono nulle, siccome sono correnti di gate dei transistori mos. Inoltre la tensione in uscita non dovrebbe essere influenzata dai circuiti connessi a carico, in modo da far funzionare questo componente come un generatore ideale di tensione, controllato dalla differenza di tensione in ingresso.

Analizzando il seguente circuito:

\begin{center}
\begin{circuitikz}
    \draw (0, 0) node[ground]{}
    to[R, i<=$I_1$, l=$R_1$] (0, 2)
    to[R, i<=$I_2$, l=$R_2$] (0, 4)
    node[vdd]{}
    (0, 2) to[short, i=$I_u$] (1, 2)
    node[circ]{} node[above]{$V_u$};
\end{circuitikz}
\end{center}

Otteniamo la formula della tensione in uscita $ V_u = \frac{R_1}{R_1 + R_2} (V_{dd} - R_2 I_u) $,
da cui possiamo osservare che se la corrente in uscita $I_u$ è 0, il circuito si comporta esattamente come un partitore di corrente. Per ottenere questo risultato, colleghiamo un transistore bipolare

\begin{center}
\begin{circuitikz}
    \draw (0, 0) node[ground]{}
    to[R, i<=$I_1$, l=$R_1$] (0, 2)
    to[R, i<=$I_2$, l=$R_2$] (0, 4)
    node[vdd]{}
    (0, 2) to[short, i=$I_u$] (1, 2)
    node[npn, anchor=B] (npn){}
    (npn.C) node[vdd]{}

    (npn.E) to[short, i=$I_u$] ++(.5, 0){}
    node[right]{$V_u$};
    ;
\end{circuitikz}
\end{center}

Il quale deve necessariamente funzionare in regione attiva diretta, in quanto la corrente di emettitore $I_u > 0$ e non può essere saturo, in quanto la giunzione della base collettore è negativa.
Quindi la corrente di collettore $I_c = \beta_F I_B$ e $I_E = (\beta_F + 1) I_B$, siccome $\beta_F$ è grande, allora la corrente $I_B$ è approssimativamente 100 volte minore della corrente $I_E$. Generando una tensione in uscita pressoché indipendente dalla corrente a carico.

Le tre caratteristiche ideali di questo circuito sono che dipende unicamente dalla differenza delle tensioni in ingresso, e la tensione e corrente a carico no influiscono sull'uscita.
\subsubsection{Amplificatore invertente}

\begin{center}
    \begin{circuitikz}
        \draw (0, 0)
        node[above]{$V_i$}
        to[R, *-*, l=$R_1$, i=$I_1$] ++(2.5, 0)
        coordinate(ing1)
        -- ++(0, 1)
        to[R, l=$R_2$, i=$I_2$] ++(2.5, 0)
        coordinate(tip);
        \draw (ing1) node[op amp, anchor=-](am){};
        \draw(am.+) -- +(0, -1) node[ground]{};
        \draw(am.out) to[short, -*] ++(1, 0)
        node[right] {$V_u$};
        \draw(tip) -- (tip |- am.out);
        \draw(am.-) to[open, v>=$V_{id}$] (am.+);
    \end{circuitikz}
\end{center}

Il ramo con la corrente $R_2$ prende il nome di ramo in retroazione, siccome collega l'uscita all'ingresso.
Studiamo il circuito nelle tre regioni di funzionamento, ricordando le relazioni:

\begin{minipage}{.3\textwidth}
    Alto guadagno:
    \[\begin{cases}
        &V_{id} = 0\\
        &-V_M < V_u < V_M
    \end{cases}\]
\end{minipage}
\begin{minipage}{.3\textwidth}
    Saturazione positiva
    \[\begin{cases}
        V_{id} > 0\\
        V_u = V_M
    \end{cases}\]
\end{minipage}
\begin{minipage}{.3\textwidth}
    Saturazione negativa
    \[\begin{cases}
        V_{id} < 0\\
        V_u = -V_M
    \end{cases}\]
\end{minipage}
Con $I^+ = I^- = 0$ e $V_{id} = V^+ - V^-$ sempre valide.

\begin{tcolorbox}
    In regione di alto guadagno, $V^+ = V^- = 0$, in quanto la tensione $V^-$ è vincolata dal potenziale di terra, portando un riferimento di terra virtuale, in quanto il nodo non è connesso veramente a terra.

    Applicando kirkoff al nodo, allora $I_1 = I_2$ implicando
    \[
        V_u = - \frac{R_2}{R_1} V_i
    \]
    Valida per $V_i > -\frac{R_1}{R_2}V_M$ e $V_u < \frac{R_1}{R_2} V_M$

    Come osservabile dalla relazione della tensione in uscita, in questa regione di alto guadagno, è possibile variare il valore del guadagno semplicemente cambiando le proporzioni tra le due resistenze.
\end{tcolorbox}
\begin{tcolorbox}
    In regime di saturazione positiva, allora $V^+ > V_-$ e la corrente in ingresso è nulla, implicando $I_1 = I_2$ e dalle equazioni di ohm:
    \[
        V^- = \frac{V_i R_2 + V_u R_1}{R_1 + R_2}
    \]
    È da controllare $V^- < 0$, rendendo vera l'equazione per $V_i < - \frac{R_1 V_M}{R_2}$, lo stesso accade in regione di saturazione negativa.
\end{tcolorbox}
\begin{center}
    \begin{tikzpicture}
        \begin{axis}
            \addplot[domain=-2:2]{-x};
            \addplot[domain=2:5]{-2};
            \addplot[domain=-5:-2]{2};
        \end{axis}
    \end{tikzpicture}
\end{center}


\subsection{Amplificatore lineare non invertente}
\begin{center}
    \begin{circuitikz}
        \draw (0, 0)
        node[ground]{}
        to[R, l=$R_1$, i=$I_1$] ++(2.5, 0)
        coordinate(ing1)
        -- ++(0, 1)
        to[R, l=$R_2$, i=$I_2$] ++(2.5, 0)
        coordinate(tip);
        \draw (ing1) node[op amp, anchor=-](am){};
        \draw(am.+) to[R, l=$R_1$] ++(0, -2) node[ground]{}
            (am.+) to[R, l=$R_2$] ++(-2, 0) node[circ]{} node[below]{$V_i$};
        \draw(am.out) to[short, -*] ++(1, 0)
        node[right] {$V_u$};
        \draw(tip) -- (tip |- am.out);
        \draw(am.-) to[open, v>=$V_{id}$] (am.+);
    \end{circuitikz}
\end{center}

\begin{tcolorbox}
    In regione di alto guadagno,
    \begin{align*}
        &V^+ = V_i \frac{R_2}{R_1 + R_2}
        &V^- = \frac{V^- - V_u}{R_2}
    \end{align*}

    Da cui la relazione: $V_u = \frac{R_2}{R_1} V_i$, che è la stessa relazione di prima, ma con segno opposto.

    Calcolando le condizioni di validità:
    \[
        \frac{R_2}{R_1} V_M < V_i < \frac{R_2}{R_1} V_M
    \]
\end{tcolorbox}
I calcoli sono gli stessi del precedente, risultando in un amplificatore lineare non invertente.

\begin{center}
    \begin{tikzpicture}
        \begin{axis}
            \addplot[domain=-2:2]{x};
            \addplot[domain=2:5]{2};
            \addplot[domain=-5:-2]{-2};
        \end{axis}
    \end{tikzpicture}
\end{center}

\subsection{Circuito sommatore analogico}
\begin{center}
    \begin{circuitikz}
        \draw (-1, 0)
        node[above]{$V_i$}
        to[R, *-*, l=$R_1$, i=$I_R$] (2, 0)
        coordinate(ing1)
        -- ++(0, 1)
        to[R, l=$R_2$, i=$I_R$] ++(2.5, 0)
        coordinate(tip);
        \draw (-1, -1) node[above]{$V_2$} to[R, l=$R_1$, *-] (1.5, -1) -- (1.5, 0);
        \draw (ing1) node[op amp, anchor=-](am){};
        \draw(am.+) -- +(0, -1) node[eground]{};
        \draw(am.out) to[short, -*] ++(2, 0)
        node[right] {$V_u$};
        \draw(tip) -- (tip |- am.out);
    \end{circuitikz}
\end{center}
\begin{tcolorbox}
    In regione di alto guadagno, unendo le equazioni delle singole correnti con l'equazione di kirkoff al nodo:
    \begin{align*}
        I_1 + I_1' = \cancel{I^-} + I_2
        \\
        \frac{V_u + V_{i2}}{R_1} = -\frac{V_u}{R_2}
    \end{align*}
    Da cui $V_u = - \frac{R_2}{R_2} (V_{i1} + V_{i2})$, risultato estremamente importante perché indica la funzione di una somma tra le due tensioni in ingresso. Per questo motivo, il circuito prende il nome di sommatore analogico.

    Questo risultato ovviamente è indipendente dal numero di ingressi, e variando il valore delle resistenze legate agli ingressi è possible fare una somma pesata dei segnali in ingresso.
\end{tcolorbox}

\subsection{Circuito derivatore}

\begin{center}
    \begin{circuitikz}
        \draw (0, 0)
        node[above]{$V_i$}
        to[C, *-*, l=$C$, i=$I_1$] ++(2.5, 0)
        coordinate(ing1)
        -- ++(0, 1)
        to[R, l=$R_2$, i=$I_2$] ++(2.5, 0)
        coordinate(tip);
        \draw (ing1) node[op amp, anchor=-](am){};
        \draw(am.+) -- +(0, -1) node[ground]{};
        \draw(am.out) to[short, -*] ++(1, 0)
        node[right] {$V_u$};
        \draw(tip) -- (tip |- am.out);
        \draw(am.-) to[open, v>=$V_{id}$] (am.+);
    \end{circuitikz}
\end{center}
\begin{tcolorbox}
    In regione attiva diretta $I_R = - \frac{V_u}{R}$ e $I_C = C \frac{dV_i}{dt}$, da cui:
    \[
        V_u = -RC \frac{dV_i}{dt}
    \]
\end{tcolorbox}
Il circuito è quindi in grado di calcolare la derivata del segnale in ingresso.
Mettendo un condensatore sul ramo di uscita ed una resistenza nel ramo in ingresso si ottiene un circuito integratore.

\subsubsection{Stadio separatore}
\begin{center}
    \begin{circuitikz}
        \draw (0, 0) node[op amp] (amp){}
        (amp.+) node[circ]{} node[left] {$V_i$}
        (amp.-) -- ++(0, 1) -| (amp.out)
        (amp.out) -- ++(0.5, 0) node[circ]{} node[above]{$V_u$}
        ;
    \end{circuitikz}
\end{center}
\begin{tcolorbox}
    In regione attiva diretta, è facile calcolare $V_u = V_i$ per $-V_M < V_i < V_M$.
\end{tcolorbox}

In questo modo è possibile leggere il valore di $V_i$ a corrente d'ingresso nulla, erogando una corrente arbitraria.
Questo principio è possibile utilizzarlo per creare un generatore di tensione
\begin{center}
    \begin{circuitikz}
    \draw (0, 0) node[ground]{}
        to[R, i<=$I_1$, l=$R_1$] (0, 2)
        to[R, i<=$I_2$, l=$R_2$] (0, 4)
        node[vdd]{}
        (0, 2) to[short, i=$I_u$] (1, 2)
        node[op amp, anchor=+] (amp){}
        (amp.-) -- ++(0, 1) -| (amp.out)

        (amp.out) to[short, i=$I_u$] ++(.5, 0){}
        node[right]{$V_u$};
    \end{circuitikz}
\end{center}
L'applicazione più importante è in strumenti di misura, essenziale per assorbire una corrente nulla dal circuito in oggetto.


\subsubsection{Test}
Ipotizzando di invertire la polarità dell'amplificatore operazionale, analizziamo il seguente circuito:
\begin{center}
    \begin{circuitikz}
        \draw (0, 0)
        node[above]{$V_i$}
        to[R, *-*, l=$R_1$, i=$I_1$] ++(2.5, 0)
        coordinate(ing1)
        -- ++(0, 1)
        to[R, l=$R_2$, i=$I_2$] ++(2.5, 0)
        coordinate(tip);
        \draw (ing1) node[op amp, anchor=+, yscale=-1](am){};
        \draw(am.-) -- +(0, -1) node[ground]{};
        \draw(am.out) to[short, -*] ++(1, 0)
        node[right] {$V_u$};
        \draw(tip) -- (tip |- am.out);
        \draw(am.+) to[open, v>=$V_{id}$] (am.-);
    \end{circuitikz}
\end{center}

\begin{tcolorbox}
    In regione di alto guadagno, $V^+ = V^- = 0$ e $I_1 = I_2$, $I_1 = V_i / R$ e $I_2 = - V^u/R_2$ che ci porta a dire:
    \[
        V_u = -\frac{R_2}{R_1} V_i
    \]
    e le condizioni di esistenza di rimangono le stesse:
    \[
        -\frac{R_2}{R_1} V_M < V_i < \frac{R_2}{R_1} V_M
    \]
\end{tcolorbox}
\begin{tcolorbox}
    In regione di saturazione positiva $V_u = V_M$ e $V_{id} > 0$, quindi $V^+ > 0$
    Ma calcolando le condizioni di esistenza otteniamo:
    \[
        V_i > -\frac{R_1 V_M}{R_2}
    \]
\end{tcolorbox}
Mentre la regione di alto guadagno non cambia aspetto, le regioni di saturazione cambiano funzionamento.

\begin{center}
\begin{tikzpicture}
    \begin{axis}
        \addplot[domain=-2:2]{-x};
        \addplot[domain=-5:2]{-2};
        \addplot[domain=-2:5]{2};
        \addplot[dashed, gray, thin] coordinates {(2,-5) (2, 5)};
        \addplot[dashed, gray, thin] coordinates {(-2,-5) (-2, 5)};
    \end{axis}
\end{tikzpicture}
\end{center}

La relazione tra ingresso ed uscita non è più funzionale, richiede un'ulteriore ragionamento per comprendere quale tra i tre valori vengono restituiti dal circuito nella regione compresa tra $-V*$ e $V*$.

Ragioniamo quindi in regime dinamico, partendo da $V_a < -V*$ spostandoci a $V_b > V*$. Fintanto che $V_a < -V*$, il valore è costante raggiunto il valore critico, per continuità l'uscita si tiene costante vino al punto $V*$. Appena raggiunto il punto $V*$ è ritorna ad essere presente una sola soluzione, portando l'uscita al valore positivo.

\begin{center}
    \centering
    \begin{tikzpicture}
        \begin{axis}
            \addplot[thin, domain=-2:2]{-x};
            \addplot[thin, domain=-5:2]{-2};
            \addplot[thin, domain=-2:5]{2};
            \addplot[dashed, gray, thin] coordinates {(2,-5) (2, 5)};
            \addplot[dashed, gray, thin] coordinates {(-2,-5) (-2, 5)};

            \addplot[red, domain=-5:2]{-2};
            \addplot[red] coordinates {(2, -2) (2, 2)};
            \addplot[red, domain=2:5]{2};
            \draw (-4.5, 0) node[circ]{} node[above]{$V_a$};
            \draw (4.5, 0) node[circ]{} node[above]{$V_b$};
        \end{axis}
    \end{tikzpicture}
    \begin{tikzpicture}
        \begin{axis}
            \addplot[thin, domain=-2:2]{-x};
            \addplot[thin, domain=-5:2]{-2};
            \addplot[thin, domain=-2:5]{2};
            \addplot[dashed, gray, thin] coordinates {(2,-5) (2, 5)};
            \addplot[dashed, gray, thin] coordinates {(-2,-5) (-2, 5)};

            \addplot[green, domain=-5:-2]{-2};
            \addplot[green] coordinates {(-2, -2) (-2, 2)};
            \addplot[green, domain=-2:5]{2};
            \draw (-4.5, 0) node[circ]{} node[above]{$V_a$};
            \draw (4.5, 0) node[circ]{} node[above]{$V_b$};
        \end{axis}
    \end{tikzpicture}
\end{center}

Seguendo analogo ragionamento in verso opposto, si nota che per andare da $b$ ad $a$ e da $a$ a $b$, si effettuando due percorsi diversi, indicati in rosso ed in verde. Questo andamento prende il nome di ciclo di isteresi, ed il circuito prende il nome di circuito "trigger" di Shmitt.

Per forzare l'uscita alta serve forzare un valore sufficientemente positivo. La caratteristica di questo circuito è che quando si triggera ed il valore si porta positivo, è difficile che si spenga a causa di rumore. Questo è utile ad esempio in un circuito dove una lampada si spegne quando raggiunge un certo livello di luce.

Questo è il primo circuito che presenta un elemento di memoria.

\subsection{}

\begin{center}
    \begin{circuitikz}
        \draw (0, 0)
        node[ground]{}
        to[R, l=$R_1$, i=$I_1$] ++(2.5, 0)
        coordinate(ing1)
        -- ++(0, 1)
        to[R, l=$R_2$, i=$I_2$] ++(2.5, 0)
        coordinate(tip);
        \draw (ing1) node[op amp, anchor=+, yscale=-1](am){};
        \draw(am.-) -- ++(0, -.7)
        coordinate(x)
        to[C] ++(0, -1) node[ground]{}
        (x) to[R]  (x -| am.out)
        -- (am.out);
        \draw(am.out) to[short, -*] ++(1, 0)
        node[right] {$V_u$};
        \draw(tip) -- (tip |- am.out);
        \draw(am.+) to[open, v>=$V_{id}$] (am.-);
    \end{circuitikz}
\end{center}

Immaginiamo che inizialmente la capacità sia scarica, e che per un qualunque motivo, ci troviamo nella condizione di alto guadagno, con $V_u = V_M$
Ricordando che la corrente in ingresso è comunque nulla, possiamo separare il ramo e vederlo come uno risolvibile attraverso la forumla del partitore:
\begin{center}
    \begin{circuitikz}
        \draw (0, 0) node[ground]{}
        to[R] (2, 0)
        to[R] (4, 0)
        node[circ]{}
        node[right]{$V_u$}

        (2, 0) -- (2, -.5) node[below]{$V^+$};
    \end{circuitikz}
\end{center}
Quindi fintanto che $V_u = V_M$ allora
\[
    V^+ = \frac{R_q} {R_1 + R_2} V_M > 0
\]
Quindi $V_{id} > 0$, in accordo con le ipotesi. Analizzando l'altro ramo con il condensatore, siccome la tensione $V_u$ è positiva, e sul ramo circola una corrente, questa ha l'effetto di caricare il condensatore $C$, aumentando la tensione $V^-$.
Fintanto che $V_{id} > 0$, ovvero $V^+ > V^-$ allora l'amplificatore lavora in alto guadagno. Nel momento in cui $V_{id} < 0$, allora $V^- > V^+$ e l'uscita si porta con un ritardo al valore $-V_M$.

Il condensatore vedendo variato il valore di $V_u$ da $V_M$ a $-V_M$ inizia a scaricarsi, fino a quando $V^- > V_u$.
Raggiunto quel valore l'uscita si riporta a $+V_M$ e si ripete il ciclo.

Analizzando la stabilità di questo circuito infatti, il valore di $V_i = 0$, otteniamo che il circuito è instabile. In questo modo si ottiene un generatore di un segnale periodico, (onda quadra) dipendente dal valore delle resistenze e dalla capacità del condensatore.

Analogamente, mettendo a cascata un numero dispari di invertitori, e chiudendoli ad anello, si ottiene un oscillatore ad anello.

\end{document}

