\documentclass[../template]{subfiles}

\begin{document}
\section{Tecnologia cmos}
Facendo riferimento all'invertitore nmos citato al capitolo precedente:
\begin{figure}[h]
    \centering
    \begin{circuitikz}
        \draw
        (0, 0) node[ground]{}
        node[nmos, anchor=S] (npn){}

        (npn.G)
        -- ++(-.5, 0)
        node[circ]{} node[above]{$V_i$}
        (npn.D) node[pmos, anchor=D] (npn2){}

        (npn2.G) -| (npn.G)
        (npn2.S) node[vcc]{}
        node[right] {$V_{cc}$}

        (npn.D) to[short] ++(0.5, 0)
        node[circ]{}
        node[right]{$V_u$}    ;
    \end{circuitikz}
\end{figure}

Il transistore a canale $n$ ke` spento per $V_i < V_{Tn}$ e saturo quando $V_i < V_u + V_{Tn}$

\begin{tikzpicture}
    \begin{axis}
        \addplot[dashed] coordinates {(1, 0) (1, 5)};
        \addplot[dashed, domain=0:5] {x -1};
        \addplot[dashed, domain=0:5] {x + 2};
        \addplot[dashed] coordinates {(3, 0) (3, 5)};
    \end{axis}
\end{tikzpicture}

In regione di saturazione: $V_{SGp} < V_{SDp} + |V_{Tp}|$ da cui $V_n < V_i + |V_{Tp}$
\begin{tcolorbox}
    In caso di nmos spento, $I_{DP} = 0$, quindi il pmos funziona in regione lineare o è spento. Avendo ipotizzato che $V_{cc} - |V_{Tp}| > V_{Tn}$, allora non esiste regione del piano in cui entrambi i transistor sono simultaneamente spenti. Quindi il transistore p deve essere in lineare.

    \[
    I_{DP} = \beta_p \big\{ V_{SGp} - |V_{Tp}|) V_{SDp} - \frac{V_{SDp}^2}{2} \big\} = \beta_p V_{SDp} (V_{SGp} - |V_{Tp}| - \frac{V_{SDp}}{2}) \big\}
\]
Quindi $V_{SDp} =0$ e $V_u = V_{cc}$.
\end{tcolorbox}
\begin{tcolorbox}
    in caso di pmos spento: $I_{Dp} = 0$, quindi necessariamente l'nmos è spento o lineare, ma siccome è già stato detto che entrambi i transistori non possono essere spenti allo stesso tempo, allora n è in lineare.
    Quindi
\[
    \beta_n V_u \big\{ V_i - V_{Tn} - \frac{V_u}{2}\big\} = 0
\]
Da cui $V_u = 0$
\end{tcolorbox}
Allora il valore massimo è il massimo possibile e non c'è corrente statica, ed il valore basso è il minimo possibile ed ancora non c'è corrente statica.
\begin{tcolorbox}
    nella regine in cui entrambi i transistori sono in regione di saturazione: (prese solo le radici positive per ipotesi)
    \[
        \sqrt{\frac{\beta_n}{\beta_p}} (V_i - V_{Tn}) = V_{cc} - V_i - |V_{Tp}|
    \]
    In questa espressione non compare $V_u$, quindi l'unica soluzione possibile è un $V_i$ costante
    \[
        V_i = \frac{V_{cc} - |V_{Tp}| + \theta V_{Tn}}{1 + \theta}
    \]
    Ottenendo un guadagno $|A_V| \to \infty$! Questo è il motivo per cui prende il nome di ratio-less.
\end{tcolorbox}
Indipendentemente dalla dimensione dei dispositivi si comporta comunque come un invertitore.

\subsubsection{Caso particolare}
Supponiamo la condizione di complementarietà perfetta: $\beta_n = \beta_p$ e $V_{Tn} = |V_{Tp}|$, quindi $\theta = 1$
\[
    \bar{V_i} = \frac{V_{dd}}{2}
\]
Ottenendo il massimo valore di margine di immunità ai disturbi.


\subsubsection{Calcolo del margine di immunità ai disturbi}
Per calcolare il guadagno è necessario calcolarlo nella curva in cui si passa da guadagno 0 a guadagno infinito.
Utilizzando il caso particolare, calcoliamo le coordinate dei punti della caratteristica in cui la derivata è -1.

Siccome n è in regione lineare e p è in saturazione, allora:
\begin{align*}
    \beta \big\{ (V_i - V_T) V_u - \frac{V_u^2}{2}\big\} = \frac{\beta_2}(V_{cc} - V_i - V_T)^2
    \\
    V_u + (V_i - V_T) \frac{dV_u}{dV_i} - V_u \frac{dV_u}{dV_i} = (V_{cc} - V_i - V_T)
    \\
    2V_u - V_i + V_T = - V_{cc} + V_i + V_T
    \\
    V_u = V_i - \frac{V_{cc}}{2}
\end{align*}
Intersecando la condizione della retta ottenuta con la curva iniziale, otteniamo il valore di \vihmin e \volmax.


\subsubsection{Porta NOR}
Dall'espressione logica $y = \overline{a + b}$ sappiamo che la rete di pull-up deve portare l'uscita ad 1, quando sia $a$ che $b$ sono a 0, mentre la rete di pull-down deve portare l'uscita a 0 quando almeno uno dei due ingressi è al valore alto.
\begin{figure}[h]
    \centering
\begin{circuitikz}
    \draw (0, 0) node[ground]{}
    -- ++ (-.5, 0) node[nmos, anchor=S] (nmos1) {}
    (nmos1.D) -- ++(.5, 0)

    (0, 0) -- ++ (.5, 0) node[nmos, anchor=S, xscale=-1] (nmos2) {}
    (nmos2.D) -- ++(-.5, 0)
    -- ++(0, .5)
    coordinate(x)

    -- ++(.5, 0) node[circ]{} node[right]{$V_u$}

    (x) node[pmos, anchor=D, xscale=-1] (pmos1){}
    (pmos1.S) node[pmos, anchor=D] (pmos2) {}

    (pmos2.S) node[vcc]{}

    (nmos1.G) |- (pmos2.G)
    (nmos2.G) |- (pmos1.G)

    (nmos1.G) -- ++(-.2, 0) node[circ]{} node[above] {$V_a$}
    (nmos2.G) -- ++(.2, 0) node[circ]{} node[above] {$V_b$}
    ;
\end{circuitikz}
\end{figure}

\begin{center}
    \begin{tabular}{|c c|c c c c|c c|c|}
        a & b & M1  & M2  & M3 & M4 & PD  & PU  & y\\
        \hline
        0 & 0 & OFF & OFF & ON & ON & OFF & ON  & 1\\
        0 & 1 & OFF & ON  & OFF& ON & ON  & OFF & 0 \\
        1 & 0 & ON  & OFF & ON & OFF& ON  & OFF & 0\\
        1 & 1 & ON  & ON  & OFF& OFF& ON  & OFF & 0\\
    \end{tabular}
\end{center}

\subsubsection{Porta nand}
Analogamente, dall'espressione, logica $y = \overline{a \cdot b}$, il pull-up deve portare l'uscita ad 1 quando almeno 1 tra $a$ e $b$ è a 0, mentre la rete di pull down porta l'uscita a 0 quando entrambi gli ingressi sono ad 1

\begin{figure}[h]
    \centering
    \begin{circuitikz}
        \draw (0, 0) node[ground]{}
            node[nmos, anchor=S] (nmos1){}

            (nmos1.D) node[nmos, anchor=S, xscale=-1] (nmos2){}

            (nmos2.D) -- ++(.5, 0) node[right]{$V_u$}

            (nmos2.D) -- ++(0, .2)  coordinate(x)
            -- ++(-.5, 0) node[pmos, anchor=D] (pmos1){}
            (pmos1.S) -- ++(.5, 0)
            node[vcc]{}
            (x) -- ++(.5, 0)
            node[pmos, anchor=D, xscale=-1] (pmos2) {}
            (pmos2.S) -- ++(-.5, 0)

            (pmos1.G) |- (nmos1.G)
            (pmos2.G) |- (nmos2.G)

            (nmos1.G) -- ++(-.7, 0) node[left]{$V_a$}
            (nmos2.G) -- ++(.7, 0) node[right]{$V_b$}

            ;
    \end{circuitikz}
\end{figure}

Inoltre bisogna notare che non sempre è ottimale esprimere le espressioni in funzione di nand o nor, ma è possibile fare la funzione direttamente con i transistori, risparmiando componenti, ad esempio $y = \overline{(a + b) c}$:
\begin{figure}[h]
    \centering
    \begin{circuitikz}
        \draw (0, 0) node[ground]{}
        -- ++ (-.5, 0) node[nmos, anchor=S] (nmos1) {}
        (nmos1.D) -- ++(.5, 0)

        (0, 0) -- ++ (.5, 0) node[nmos, anchor=S, xscale=-1] (nmos2) {}
        (nmos2.D) -- ++(-.5, 0)
        node[nmos, anchor=S] (nmos3){}

        (nmos3.D)
        -- ++(0, .5)
        -- ++(-.5, 0)
        node[pmos, anchor=D](pmos1){}
        (pmos1.S) node[pmos, anchor=D](pmos2){}
        (pmos2.S) -- ++(.5, 0)
        coordinate(x)
        node[vcc]{}


        (nmos3.D) ++ (0, .5) -- ++(.5, 0)
        node[pmos, anchor=D, xscale=-1] (pmos3){}
        (pmos3.S) |- (x)

        (nmos1.G) node[left]{$V_a$}
        (nmos2.G) node[right]{$V_b$}
        (nmos3.G) node[left]{$V_c$}

        (pmos1.G) node[left]{$V_b$}
        (pmos2.G) node[left]{$V_a$}
        (pmos3.G) node[right]{$V_c$}

        (nmos3.D) -- ++(.2, 0)
        node[right]{$y$}
        ;
\end{circuitikz}
\end{figure}

Risparmiando un numero di transistori inferiore alla metà.
Con questo tipo di logica, per $n$ ingressi, sono richiesti $2n$ transistori.
\subsubsection{Transistori in parallelo}
\begin{center}
    \begin{circuitikz}
            \draw (0, 0)
            -- ++ (-.5, 0) node[nmos, anchor=S] (nmos1) {}
            (nmos1.D) -- ++(.5, 0)

            (0, 0) -- ++ (.5, 0) node[nmos, anchor=S, xscale=-1] (nmos2) {}
            (nmos2.D) -- ++(-.5, 0)
            -- ++(0, .5)

            (0, 0) -- ++(0, -.5)
            ;
    \end{circuitikz}
\end{center}
Ai due transistori connessi in parallelo, è possibile sostituire un'unico transistore con stessa tensione di soglia $V_T$ ed un fattore $\beta_{ed}$ calcolabile come:
\begin{align*}
    I_{Deq} = (\beta_1 + \beta_2) \{(V_{GS} - V_T) V_{DS} \frac{V_{DS}^2}{2}\}
    \\
    \beta_{eq} = \beta_1 + \beta_2
\end{align*}

Analogamente se si dispongono due transistori, caratterizzati dalla stessa tensione di soglia $V_T$, se la tensione in ingresso è equivalente, allora è possibile sostituire ai due transistori un singolo transistore con
\[
    \beta_{eq} = \frac{\beta_1 \beta_2}{\beta_1 + \beta_2}
\]
Nel caso $\beta_1 = \beta_2$: $\beta_{eq} = \beta_2$

Una qualsiasi rete cmos è quindi sempre riconducibile ad un invertitore cmos equivalente.

In una rete cmos se un transistore è spento, quello conta come un circuito staccato.
\subsection{Prestazioni e qualità dell'invertitore}
Il circuito non presenta potenza statica, ma una dinamica, in quanto la corrente è diversa da 0 al momento del passaggio di stato.
Inoltre in quanto è presente uno spostamento di carica è presente anche un ritardo.

Per calcolare questi valori, prendiamo la capacità peggiore $C_{o}$ del transistore.
Il modello del transistore MOS deve tenere conto anche di altre complicazioni, come la presenza di una capacità parassita non lienare tra source e bulk e tra bulk e drain. Una ctra source e gate ed una tra gate e drain. In altre parole ogni collegamento interno ad un circuito contiene una rispettiva capacità parassita tra i due nodi.

Per semplificare i calcoli, diciamo che queste capacità sono correlate alla capacità del condensatore $C_o$.


\subsubsection{Calcolo del ritardo tra due invertitori in cascata}
\begin{center}
\begin{circuitikz}
    \draw
    (0, 0)
    node[ground] {}
    node[nmos, anchor=S] (nmos1){}
    (nmos1.D) node[pmos, anchor=D] (pmos1){}
    (pmos1.S) node[vcc]{}
    (nmos1.G) |- (pmos1.G)

    (nmos1.D) -- ++(1, 0)
    coordinate(x)
    to[C] ++(0, -1)
    node[ground]{}

    (x) -- ++(1, 0)

    (nmos1.G) ++ (3, 0)
    node[nmos, anchor=G] (nmos2){}
    (nmos2.S) node[ground]{}
    (nmos2.D) node[pmos, anchor=D](pmos2){}
    (pmos2.S) node[vcc]{}
    (nmos2.G) |- (pmos2.G)
    ;
\end{circuitikz}
\end{center}

È possibile concentrare tutte le capacità parassita dei condensatori, in un unica capacità, situata nella connessione tra i due invertitori.
La capacità semplificata dipende da $C_L = K C_0 + C_{wire}$, ovvero parte delle capacità parassite dei transistori e la capacità parassita del filo. Il termine prevalente è $k C_0$

Presupponiamo che la tensione in ingresso abbia un andamento del tipo:

\begin{center}
    \begin{tikzpicture}
        \begin{axis}[xlabel=$t$, ylabel=$V_i$]
            \addplot[domain=-5:0]{0};
            \addplot[domain=0:5]{3};
        \end{axis}
    \end{tikzpicture}
\end{center}

\begin{tcolorbox}
    per $t < 0$  e $V_i = 0$ siamo in condizione statica, ed è già nota dalla caratteristica: $V_u = V_{cc}$.
\end{tcolorbox}
\begin{tcolorbox}
    Per $t > 0$ $V_i = V_{cc}$, presupponendo un tempo infinito, in cui si annullano gli effetti del transitorio, conosciamo già i valori della caratteristica: $V_u = 0$.

\end{tcolorbox}

Dato che serve un tempo infinito per raggiungere il valore $V_u = 0$, consideriamo il tempo di propagazione, il tempo necessario per portare l'uscita $V_u = V_{cc}/2$.

\begin{tcolorbox}
    Per $t = 0^+$, $V_{GS} = V_i = V_{cc} > V_T$, quindi l'nmos è necessariamente acceso, mentre
    $V_{SGp} = V_{cc} - V_i = 0 < |V_{Tp}|$ il pmos si spegne immediatamente.

    Quindi la corrente dell'nmos $I_D$ è uguale alla corrente che passa dal condensatore.
    \[
        I_{DN} = - I_C = - C_L \frac{dV_u}{dt}
    \]
    Siccome la formula della corrente varia in base al regime di funzionamento dell'nmos, il transitorio può essere scomposto in due tratti:
    da $V_{cc}$ a $V_{cc} - V_T$ in regione satura, ed un secondo tratto il regione lineare fino a raggiunere il valore $V_{cc} /2$


    Quindi nel tratto in regione satura:
    \begin{align*}
        I_D = \frac{\beta_n}{2}(V_{GS} - V_T)^2 = - C_L \frac{dV_u}{dt}
        \\
        \int^t_0 \frac{\beta_n}{2}(V_{GS} - V_T)^2 dt = \int^{V_u(t)}_{V_u(0) = V_{cc}} - C_L \frac{dV_u}{dt} dt
    \end{align*}

    Ottenendo un'andamento lineare decrescente
    \[
        V_u(t) = V_{cc} - \frac{\beta_n}{2C}(V_{cc} - V_T)^2t
    \]
    Ponendo la formula uguale a $V_{cc} - V_T$ ricavo la durata del primo transitorio:
    \[
        t_1 = \frac{2 C_L}{\beta_n} \frac{V_T}{(V_{cc} - V_T)^2}
    \]

    Il secondo tratto del transitorio è caratterizzato dalla regione lineare, valido da $V_{cc} - V_T$ a $\frac{V_{cc}}{2}$.
    Risolvendo l'equazione utilizzando la diversa espressione della corrente ottengo ($m = 2(V_{cc} - V_T)$)
    \begin{align*}
        \frac{2 C_L}{\beta} \frac{dV_u}{dt} = V_u (V_u - m)
        \\
        \frac{2 C_L}{\beta} \frac{1}{V_u (V_u - m)}\frac{dV_u}{dt} = 1
        \\
        \int_{t_1}^{t_{phl}}\frac{2 C_L}{\beta} \frac{1}{V_u (V_u - m)}\frac{dV_u}{dt} dt= \int_{t_1}^{t_{phl}} dt
        \\
        \frac{2 C_L}{\beta_n} \frac{1}{2(V_{cc} - V_T)} \ln \big(3 - 4\frac{V_T}{V_{cc}}\big) = t_{phl} - t_1
    \end{align*}

    Mettendo a sistema l'equazione appena ottenuta con $t_1$ calcolato in precedenza, otteniamo:
    \[
        t_{phl} = \frac{2 C_L}{\beta_n (V_{cc} - V_T)} \big\{
        \frac{V_T}{(V_{cc} - V_T)} + \frac{1}{2} \ln \big( 3 - 4 \frac{V_T}{V_{cc}}\big)\big
        \}
    \]

\end{tcolorbox}
Ipotizzando che $V_{cc} \gg V_T$ l'espressione si semplifica notevolmente in
\[
    \frac{C_L}{\beta_n V_{cc}} \ln(3) \approx \frac{C_L}{\beta_n V_{cc}}
\]
% 20200609

La capacità $C_L$ che è una tra le componenti principali del ritardo è composta da $C_{MOS}$, capacità parassita legata al transistore e da $C_{wire}$, capacità parassita del filo.
Prendendo come presupposto che $C_{MOS} \gg C_{wire}$, allora possiamo riscrivere l'espressione del tempo di propagazione come:
\[
    t_p = \frac{k C_{ox} w L}{c_{ox} \mu_n \frac{w}{L} V_{cc}} = \frac{k}{\mu_n}\frac{L^2}{V_{cc}}
\]
Da cui possiamo osservare come sia la capacità, sia il fattore $w$ non contino nell'espressione del tempo di reazione. Si osserva quindi che la componente principale è la lunghezza del canale $L$.
Una volta ridotta il più possibile la lunghezza $L$, l'unico parametro rimanente è la tensione $V_{cc}$, aumentando la tensione però ci si può aspettare un aumento di potenza.

Diversamente se la capacità parassita del filo prevale sulla capacità interna si sviluppano altri termini progettuali attraverso circuiti di buffer (spiegato dopo).

\subsubsection{Consumo di potenza}
Abbiamo già discusso il fatto che la potenza statica dell'invertitore è nulla. Ma la corrente di polarizzazione inversa, interna al transistor, seppur piccola, può essere significativa in circuiti "fermi" risultando, in circuiti complessi come componente fondamentale della potenza statica.

La dispersione principle dei dispositivi che analizziamo in questo corso è la potenza dinamica. Siccome non ha senso calcolare la potenza in un determinato istante, faremo riferimento ad una potenza media:
\[
    \overline{P} = \frac{1}{T} \int^T_0 P_{dd} dt
\]
$P_{dd}$ è la potenza istantanea erogata dal generatore e corrisponde a $P_{dd} = V_{dd} I_D$
\begin{center}
    \begin{circuitikz}
        \draw
        (0, 0)
        node[ground] {}
        node[nmos, anchor=S] (nmos1){}
        (nmos1.D) node[pmos, anchor=D] (pmos1){}
        (pmos1.S) node[vdd]{}
        (nmos1.G) |- (pmos1.G)

        (nmos1.D) -- ++(1, 0)
        coordinate(x)
        to[C] ++(0, -1)
        node[ground]{}

        ;
    \end{circuitikz}
\end{center}
Dal circuito possiamo osservare come la corrente di pull-up $I_D = I_C + I_{Dn}$. Passando da un valore di uscita basso, ad un valore di uscita alto il valore della corrente assume un valore diverso da $0$, quindi anche in assenza del condensatore si dissipa potenza solamente per il fatto di avere per un'istante entrambi i condensatori accesi.
La potenza dissipata in questo modo prende il nome di potenza di cortocircuito $P_{cc}$, con rispettiva corrente $I_{cc}$.
L'altra componente della potenza dissipata è quella associata al carico, utilizzata per caricare il condensatore parassita $C_L$, prendendo il nome di potenza di carico $P_L$.

Si può dimostrare che $P_{cc}$ dipende solo da quando il segnale d'ingresso è compreso tra $V_T < V_i < V_{dd} - V_T$, assumendo la forma di
\[
    \overline{P_{cc}} = \frac{\beta}{12} \frac{t_R}{T} (V_{dd} - 2 V_T)^3
\]

Di conseguenza se il tempo impiegato dal segnale d'ingresso per portarsi al valore alto tende a $0$, allora anche $P_{cc} \to 0$.

Per calcolare solamente la componente di potenza di carico, basta annullare la potenza di cortocircuito, assumendo che il segnale d'ingresso abbia un andamento a gradino.

Prendendo come riferimento un segnale d'ingresso ad'onda quadrata con periodo $T$, nel primo periodo da $0$ a $T/2$, il segnale d'ingresso è alto, quindi $V_i = V_{cc}$, di conseguenza l'nmos è acceso ad il pmos è spento. Nel secondo periodo il segnale d'ingresso è basso, quindi l'nmos è spento ed il pmos è acceso.

Quindi
\[
    \overline{P_L} = \frac{1}{T} \int_0^T V_{dd} (I_{dn} + I_C) dt = \overline{P_n} + \overline{P_p} + \overline{P_C}
\]
osservando che il sistema è periodico, tutta la potenza assorbita dal circuito in un periodo, deve essere dissipata dal circuito nel periodo successivo, quindi la potenza media è composta dalla potenza dei singoli componenti del circuito: nmos, pmos e condensatore

\[
    \overline{P_C} = \frac{1}{T} \int_0^T V_u I_C dt = \frac{1}{T} \int_0^T V_u C \frac{dV_u}{dt} dt = \frac{C}{T} \big\{\frac{V_{dd}^2}{2} - \frac{V_{dd}^2}{2}\big\} = 0
\]
Nella fase di pull-up il condensatore si carica e nella fase di pulldown rilascia energia, risultando in un bilancio energetico nullo.
\[
    \overline{P_n} = \frac{1}{T} \int_0^T V_{DS} I_D dt = \frac{1}{T} \int_0 ^{T/2} V_u (-C) \frac{dV_u}{dt} dt = -\frac{C}{T} \big\{ \frac{0^2}{2} - \frac{V_{dd}^2}{2}\big\} = \frac{C}{T} \frac{V_{dd}^2}{2}
\]
\[
    \overline{P_p} = \frac{1}{T} \int_0^T V_{SD} I_D dt = \frac{1}{T} \int_{T/2}^T (V_{dd} - V_u) I_D dt = \frac{C}{T} \frac{V_{dd}^2}{2}
\]
Concludendo che l'energia nella fase di pull-up è per metà dissipata e per metà immagazzinata nel condensatore, nella fase di pull-down l'energia immagazzinata viene dissipata attraverso la potenza richiesta dalla rete di pull-down nell'altra fase del transitorio.

\[
    \overline{P_L} = \frac{C_L V_{dd}^2}{T} = C_L V_{dd}^2 f
\]
aumentando $V_{dd}$ il circuito è quindi più veloce ma spende molta più potenza. Inoltre maggiore è la frequenza, maggiore è l'energia dissipata. Per questo motivo a bassi livelli di energia le prestazioni dei calcolatori diminuiscono.

Nel confronto tra le due componenti di potenza, $P_{cc}$ e $P_L$, quello che si vede facilmente è che aumentando la frequenza, tipicamente il termine $P_L$ prevale su $P_{cc}$.

Il prodotto tra il tempo richiesto per effettuare un'operazione e la potenza media richiesta, può essere interpretato come l'energia richiesta da una signola operazione:
\[
    t_p \overline{P_L} = \frac{C_L}{\beta V_{dd}}\frac{C_L V_{dd}^2}{T} = \frac{C_L^2 V_{dd}}{\beta T}
\]
Diminuire la capacità ha un effetto benefico su entrambi i parametri.

\end{document}
