\documentclass[../template]{subfiles}

\newcommand{\idiff}{\ensuremath{I = \frac{dQ}{dt}}}

\begin{document}
\section{Dispositivi e Circuiti Elettronici}
La corrente elettrica $\idiff$ è generata da movimenti di carica. Uno tra i diversi in moto per spostare delle cariche è attraverso un campo elettrico.

La densità di corrente elettrica $J$ è legata al campo elettrico $E$ dalla conducibilità elettrica $\sigma$ attraverso la legge di ohm $\bar{J} = \sigma \bar{E}$.
I materiali sono classificabili dal loro valore di $\sigma$, i materiali isolanti sono caratterizzati da bassi valori di $\sigma$, corrispondenti a bassi valori di corrente.
Valori di $\sigma$ alti comportano un'alta corrente, caratteristica dei materiali conduttori.

I semiconduttori sono materiali con conducibilità elettrica $\sigma$ variabile (ad esempio in funzione della temperatura).
Guardando il caso del componente resistenza, la corrente ai capi di essa dipende dall'inverso della costante $R$, quindi, a tensione costante, maggiore è la resistenza, minore è la corrente misurata ai due capi del componente.

\begin{figure}[h]
    \centering
    \includegraphics[width=.30\textwidth]{img/resistence-current}
    \includegraphics[width=.30\textwidth]{img/diode-current-graph}
\end{figure}
In figura sono riportate le relazioni corrente-tensione di una resistenza ed un diodo. È possibile osservare come il comportamento della corrente del diodo è completamente differente.

\subsubsection{Modello fisico}
Gli elettroni di ciascun materiale orbitano attorno ai rispettivi nuclei, in stato di equilibrio.
In stato equilibrio, gli elettroni di un materiale orbitano attorno al nucleo, la complessiva somma delle cariche risulta nulla.

Gli elettroni attratti dalla forza coulombiana, orbitano attorno al nucleo. Ad ogni orbita corrisponde una velocità di percorrenza,
quindi un'energia cinetica. L'ampiezza di ogni orbita dipende inversamente dal quadrato della distanza dal nucleo (formula forza di coulomb).
\\
In definitiva, possiamo

% TODO

Maggiore è la grandezza dell'orbita degli elettroni, maggiore è la veloci
Inoltre maggiore è la grandezza dell'orbita degli elettroni, maggiore è la velocità degli elettroni, quindi la loro energia cinetica.

All'ampiezza dell'orbita è quindi associato inversamente la forza attrattiva del nucleo, ed un'energia cinetica legata alla velocità di percorrenza.
\\
Possiamo quindi, gli elettroni interni ad un atomo generico in funzione dell'energia da essi posseduta, ottenendo un diagramma simile a quello in figura ..., distinta da diversi livelli

\begin{figure}[h]
    \centering
    \begin{tikzpicture}
        \draw[->] (0, 0) -- (0, 3)
            node[right]{$E$};

        \foreach \y in {.5, 1, ..., 2.5}
        {
            \draw (-3pt, \y) -- (+3pt, \y);
        }
    \end{tikzpicture}
\end{figure}

Ad ogni elettrone è dunque associato un livello sull'asse delle energie.
Per il principio di quantizzazione, dell'energia, non tutti i possibili livelli di energia sono possibili. L'asse quindi non è continua, ma quantizzata.

È importante notare che le orbite degli elettroni attorno agli atomi non sono descritte dalla meccanica classica, ma dalla meccanica quantistica, dove gli elettroni sono descritti attraverso forme sinusoidali.

Per questo motivo è possibile associare agli elettroni una lunghezza d'onda $\lambda$, dove nel caso di ipotesi stazionaria, è necessario che la circonferenza dell'orbita sia multiplo di $\lambda$.
\\
Questo spiega a grandi linee la presenza di valori permessi e proibiti sull'asse delle energie.

\subsubsection{Principio di esclusione di Pauli}
Ciascun livello energetico è occupabile al più da due elettroni (con spin opposto). In altre parole, il numero di elettroni che possono avere una certa distanza dal nucleo è finito.

Da questo è possibile dedurre che è possibile occupare interamente i livelli di energia.

Spostare corrente vuol dire spostare elettroni, cambiandone la velocità, quindi aumentandone l'energia cinetica, vincolata dai livelli di energia e dal principio di esclusione.

Per muovere un'elettrone quindi serve almeno l'energia per raggiungere il livello energetico libero più vicino.

Nel momento in cui due atomi diventino abbastanza vicini da interagire, quello che mi posso aspettare è che gli elettroni dei rispettivi atomi esercitino una forza repulsiva, con l'effetto di modificare l'orbita dei due elettroni.

Considerando quindi un unico diagramma energetico per il sistema dei due atomi, quello che ottengo è che i livelli non si sovrappongono ma si scostano leggermente, mantenendo comunque i principi di quantizzazione ed esclusione.

Generalizzando il discorso con un'interazione di $n$ atomi, otteniamo che ai precedenti livelli energetici, corrisponderà una moltitudine di livelli permessi, tra loro poco differenti, chiamata \textit{banda permessa}.
\\
I valori di energia tra due bande permesse, prendono il nome di \textit{banda proibita}.

È importante sottolineare che i valori interni alla banda permessa rispettano ancora il principio di esclusione e di quantizzazione, quindi è possibile che un'intera banda sia occupata da elettroni.

In condizione di quiete, gli elettroni tendono spontaneamente ad occupare i livelli con minore energia, quindi quelli più bassi.

La statistica di fermi indica un valore limite (\textit{Livello di Fermi}) che indica, in assenza di perturbazione, in termini probabilistici, tutti i livelli che sotto tale valore risultano occupati.

La posizione di questo livello diventa quindi fondamentale per indicare le condizioni di trasporto di carica del materiale.
Nel caso di materiali conduttori, il livello di fermi ricade internamente ad una banda permessa,
l'energia richiesta per spostare elettroni da livelli energetici occupati a livelli energetici liberi, è dipendente dalla loro distanza, quindi molto bassa.

Nel caso di materiali isolanti, il livello di fermi ricade all'interno di una banda proibita, l'energia sufficiente richiesta (energy gap) è quella per scavalcare l'intera banda proibita, quindi molto superiore al caso precedente.

I materiali isolanti sono dunque anch'essi soggetti a fenomeni di scarica elettrica.

\subsubsection{I semiconduttori}
Il semiconduttore ha una struttura simile a quella dell'isolante, quello che differisce è l'ampiezza del gap, richiedendo una quantità di energia bassa per effettuare il "salto" della banda proibita.
Se la quantità di energia ricevibile dalla temperatura dall'ambiente è pari o superiore al gap, diventa facile che elettroni passino da una banda energetica superiore.
Elettroni quindi nella banda permessa superiore, trovandola completamente vuota, richiedono a loro volta poca energia per muoversi da un livello energetico all'altro, fornendo al materiale caratteristiche di un conduttore.
Chiameremo quindi questa nuova banda, banda di conduzione, mentre chiameremo la vecchia banda, banda di valenza.

La banda di valenza, avendo anch'essa livelli svuotati da elettroni spostati in banda di conduzione, richiederà anch'essa poca energia per effettuare salti di gap interni alla banda.
La conducibilità elettrica aumenta quindi con la temperatura del circuito.

Riassumendo quindi, un semiconduttore il livello di fermi interseca una banda proibita, la cui ampiezza è sufficientemente piccola da essere probabile l'effetto di scavalcamento con la sola energia termica.
Il semiconduttore ha quindi "due" bande di conduzione.


% seconda lezione
\subsubsection{Seconda lezione}
la distanza dal nucleo cresce col crescere dell'energia, andando a vedere gli elettroni sullo strato di valenza, sono quelli più distanti dal nucleo.
La forma del reticolo di atomi è determinata dal numero di elettroni disponibili a collegarsi agli atomi vicini.
Tutti i materiali semiconduttori sono materiali della quarta colonna della tavola periodica, i quali possiedono quattro elementi nell'orbita di legame, tra essi prevale il silicio.

In alternativa agli elementi della tavola periodica è possibile formare delle leghe tra elementi della terza e quarta colonna o quarta e quinta, come arseniuro di gallio.

Schematizzando il reticolo del silicio su una mappa in due dimensioni (figura ?)
La promozione da banda di valenza a banda di conduzione, significa che esso è meno legato al nucleo originario. È talmente poco legato al nucleo che attraverso una forza di un campo magnetico può spostarsi internamente al reticolo.
Tali elettroni vengono definiti come elettroni liberi.

Il materiale in condizione di quiete è neutro, ovvero ha tanta carica positiva quanta negativa. Al momento in cui un'elettrone libero esce da una regione, quella regione non ha più carica nulla ma leggermente positiva. Possiamo quindi immaginare il moto dell'elettrone come uno spostamento di carica nello spazio, quindi una corrente.
\\
Lo spazio lasciato vuoto dall'elettrone, è successivamente occupato da altri elettroni liberi. Creando uno spostamento a catena degli elettroni.

Alla banda di conduzione è associato il movimento di una carica negativa, alla banda di valenza è associato il movimento della lacuna, carica positiva generata dallo spostamento dell'elettrone.

In un conduttore esiste solo un tipo di portatore di carica (elettroni), mentre nei semiconduttori esistono portatori di carica positiva e portatori di carica negativa.

È possibile modellare il movimento della lacuna come il movimento di una fittizia particella fisica, dotata di una massa (maggiore di quella dell'elettrone perché si muove più lentamente)

Per misurare la corrente è necessario conoscere la quantità di elettroni e lacune in un determinato volume.
Indicheremo quindi con $n$ il numero di elettroni per unità di volume \footnote{Misurata per numero di elettroni per centimetro cubo}.
Analogamente definiamo la concentrazione di lacune con $p$ come il numero di lacune per unità di volume.

Per densità di carica degli elettroni si calcola con $-qn$, mentre per lacune $qn$.
Ad ogni elettrone libero in banda di valenza, corrisponde una lacuna in banda di conduzione, quindi necessariamente all'equilibrio $p = n$.

Possiamo quindi definire come evento di generazione il momento in cui un'elettrone abbandona la banda di valenza generando una lacuna, mentre il fenomeno duale, il passaggio da banda di conduzione a bassa di valenza viene chiamato evento di ricombinazione.

Indichiamo quindi con $G$ il numero di coppie elettrone-lacune generate nell'unità di volume e nell'unità di tempo. Il tasso di ricombinazione è indicato con $R$.

Alla equilibrio quindi $n = p = n_i(T)$ e $R = G\ge 0$ il numero di elettroni e lacune costante prende il nome di $n_i$ (concentrazione intrinseca del materiale).

Sostituendo ad alcuni atomi di silicio, con atomi della 5a colonna, as es fosforo. In questo modo avendo un'elettrone di legame in più (ed un protone in più), siccome i legami sono tutti occupati con gli atomi di silicio adiacenti, otteniamo un'elettrone che non contribuisce ad alcun legame nel reticolo.
\\
La sostituzione di alcuni atomi di silicio con atomi della 3a o 5a colonna prende il nome di drogaggio.

La rara sostituzione di atomi di silicio, non varia il reticolo cristallino originario.

L'elettrone nella fascia più esterna non essendo legato agli altri se riceve energia sufficiente può liberarsi e comportarsi come una carica negativa mobile. È ancora vero che lascia alle sue spalle una carenza di carica negativa, ma la carica positiva è associata alla presenza del protone nel nucleo e non è in grado di spostarsi.
In questo caso si genera un'elettrone mobile, senza generare lacune mobili. In questo caso $p \neq n$. Gli atomi della quinta colonna prendono quindi il nome di atomi droganti di tipo donatore.

Tutto questo è rappresentabile nel diagramma delle energie inserendo un livello "donatore" interno alla banda proibita permettendo che l'evento di liberazione dell'elettrone richieda meno energia, rendendolo ancora più probabile a temperatura ambiente.
\\
Chiameremo $N_D$ la concentrazione di atomi donatori, e per mantenere la struttura del cristallino $N_D \ll 10^{22}$.

Analogamente prendendo un' elemento della terza colonna (es. boro) nel reticolo viene rimosso un legame, fornendo energia al reticolo e spostando elettroni, essi andranno ad occupare del legame mancante, generando lo spostamento di una carica positiva.
La carica negativa è fissa perché legata alla struttura del boro.

Chiameremo il boro atomo accettore, perché capace di ionizzare il reticolo negativamente. Con $N_A$ indicheremo la concentrazione degli atomi accettori per unità di volume. Interpretando l'evento sul diagramma energetico, sarebbe come un livello energetico che ricade nella banda proibita molto vicino alla banda di valenza. Generando una lacuna mobile in tale banda.

In questo caso $n < p$.
\end{document}
