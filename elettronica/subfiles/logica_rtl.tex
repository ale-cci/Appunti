\documentclass[../template]{subfiles}

\begin{document}
\section{Logica RTL: Transistor Resistor Logic}
Se associamo a valori di tensione alti e bassi (\V{H}, \V{L}) una codifica logica 1 e 0, il circuito precedente si comporta come un invertitore
logico.

Bisogna ricordare che anche il trasporto di carica (corrente) non è uniforme, ed è soggetto anch'esso a rumore.
Il problema che ha questo circuito visto come amplificatore è che amplifica il rumore in ingresso ed il segnale in ingresso dello stesso
fattore $A_V$.
Inoltre dato che il segnale in ingresso varia di poco, la qualità del segnale d'uscita risente molto delle variazioni di rumore.

Diversamente lo stesso circuito visto come invertitore logico, anche a rumore elevato fa il suo lavoro.

L'immunità al rumore è caratterizzato dalle due caratteristiche a guadagno 0 (off e sat). Solo a fronte di valori estremamente alte di rumore, in
grado di entrare nella fascia intermedia, si manifestano sul segnale in uscita.
% 33:00

\def\vilmax{\ensuremath{{\V{IL}}_\text{MAX}}}
\def\vihmin{\ensuremath{{\V{IH}}_\text{MIN}}}
\def\nm{\ensuremath{{N}_\text{M}}}

\subsection{Margine di immunità al rumore}
\begin{figure}[h]
    \centering
    \begin{tikzpicture}[scale=.8]
        \begin{axis}[ymin=-1, xmin=-1, ymax=4]
            \addplot[dashed, clr-gray]{\vcesatval};

            \addplot[domain=-5:\vgamma]{3};
            \addplot[domain=\vgamma:2]{(3 - \vcesatval) / (\vgamma - 2) * x +3 -(3 - \vcesatval) / (\vgamma - 2) * \vgamma };
            \addplot[domain=2:5]{\vcesatval};

            \addplot[dashed, clr-gray] coordinates {(\vgamma, 3) (\vgamma, 0)};
            \draw
                (-0.5, \vcesatval) node[above]{\V{L}}
                (-0.5, 3) node[above] {\V{H}}
                (2, 0) node[below]{\vihmin}
                (3, 0) node[circ]{} node[below]{\V{H}}
                (1.9, 1.9) node[right]{AD}
                (4, \vcesatval) node[above]{SAT}
                (0.4, 3) node[below]{OFF}
                (\vgamma, 0) node[below]{\vilmax}
                ;

        \end{axis}
    \end{tikzpicture}
\end{figure}
\noindent
Nel grafico del segnale d'uscita del circuito possiamo riconoscere il valore basso $\V{L} \equiv \vcesat$ ed allo stesso modo un valore
alto $\V{H} \equiv \V{cc}$.
La caratteristica da invertitore è esprimibile quindi come $\vu(\V{H}) = \V{L}$ e $\vu(\V{L}) = \V{H}$.
Inoltre siccome questo circuito tollera il rumore, per rumore minore di $\delta_1$ abbiamo ugualmente che $\vu(\V{L} + \delta_1) = \V{L}$.
Chiamiamo quindi $\vilmax = \V{L} + \delta_1$ il massimo valore della tensione d'ingresso che produce in uscita un $\V{H}$.
Analogamente chiamo $\vihmin = \V{H} - \delta_2$ il massimo valore della tensione d'ingresso che produce ancora un'uscita $\V{L}$.
Il margine di immunità al rumore \nm è il minimo tra $\delta_1$ e $\delta_2$.
\\
Dato che sappiamo già i valori di \V{H} e \V{L}, calcolando \vihmin come punto di intersezione tra la zona attiva diretta e di saturazione,
otteniamo:
\begin{align*}
    \delta_1 = \vilmax - \V{L} = \vgamma - \vcesatval = 0.55 V\\
    \delta_2 = \V{H} - \vihmin = 5 - 1.23 = 3.77V
\end{align*}
Da cui $\nm = 0.55 V$. Per avere immunità al rumore è necessario che $\delta_1 = \vilmax - \V{L}$ e $\delta_2 = \V{H} - \vihmin$ siano positivi
sarà positiva anche la loro somma, per questo possiamo dire che
\begin{align*}
    \V{H} - \V{L} > \vihmin - \vilmax\\
    \frac{\V{H} - \V{L}}{ \vihmin - \vilmax} > 1
\end{align*}
Facendo sempre riferimento al grafico è facile capire che rapporto tra $\V{H} - \V{L}$ e $\vihmin - \vilmax$ è $|A_V|$.
Per cui $|A_V| > 1$. Inoltre maggiore è tale valore, maggiore sarà il margine di immunità al rumore.

\end{document}
