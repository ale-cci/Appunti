\documentclass[../template]{subfiles}

\begin{document}
\section{Logica RTL: Transistor Resistor Logic}
Se associamo a valori di tensione alti e bassi (\V{H}, \V{L}) una codifica logica 1 e 0, il circuito precedente si comporta come un invertitore
logico.

Bisogna ricordare che anche il trasporto di carica (corrente) non è uniforme, ed è soggetto anch'esso a rumore.
Il problema che ha questo circuito visto come amplificatore è che amplifica il rumore in ingresso ed il segnale in ingresso dello stesso
fattore $A_V$.
Inoltre dato che il segnale in ingresso varia di poco, la qualità del segnale d'uscita risente molto delle variazioni di rumore.

Diversamente lo stesso circuito visto come invertitore logico, anche a rumore elevato fa il suo lavoro.

L'immunità al rumore è caratterizzato dalle due caratteristiche a guadagno 0 (off e sat). Solo a fronte di valori estremamente alte di rumore, in
grado di entrare nella fascia intermedia, si manifestano sul segnale in uscita.
% 33:00

\subsubsection{Soglia di rumore}

\end{document}
