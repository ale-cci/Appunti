\documentclass[../template]{subfiles}

\def\xbe{\ensuremath{X_\text{BE}}}
\def\xbc{\ensuremath{X_\text{BC}}}

\begin{document}
\section{Diodo in regione dinamica}
\begin{figure}[h]
    \centering
    \begin{circuitikz}[scale=1.2]
        \draw
        (0, 0)
        to[short, i=$\I{BC}$] (0.7, 0)
        to[diode] (2, 0)
        to[short, i<=$\I{C}$] ++(0.5, 0) node[below]{$C$}
        (0, 0)
        to[short, i=$\I{BE}$] (-0.7, 0)
        to[diode] (-2, 0)
        to[short, i=$\I{E}$] ++(-0.5, 0) node[above]{$E$}
        (0, 0) -- (0, -1) to[short, i<=$\I{B}$] (0, -2) node[right]{$B$}

        (0, -1) to[polar capacitor, l_=$Q_R$]
        ++(2, 0) -- ++(0, 1)
        (0, -1) to[polar capacitor, label=$Q_C$] ++(-2, 0) -- ++(0, 1)
        (0, 0) node[circ]{}
        ;
        \draw (2, 0)
        to[short, i=${\I{t}}$, color=clr-main-red] (2, 1)
        to[american controlled current source, color=clr-main-red] (-2, 1)
        -- (-2, 0)
        ;
    \end{circuitikz}
\end{figure}
A pagina \pageref{sec:regime_dinamico} è stato analizzato il diodo in regime dinamico.
\\
Effettuiamo quindi un'analisi analoga al transistore bipolare, utilizzando come formule, quelle del modello di Ebers e Moll.

Dato che i termini esponenziali compaiono spesso, introduciamo per semplicità le seguenti notazioni:
\begin{align*}
    \xbe = \big(e^{\vbe/ \V{T}} -1 \big) = \frac{\I{F}}{\I{S}}\\
    \xbc = \big(e^{\vbc/ \V{T}} -1 \big) = \frac{\I{R}}{\I{S}}
\end{align*}
Inoltre indichiamo con $\I{F} = \I{S} \xbe$ e $\I{R} = \I{S} \xbc$ le correnti forward e backward che si manifestano.
Ricordandoci che $\alpha_R = \frac{\I{S}} {\I{S} + \I{BCS}}$ e $\alpha_R = \frac{\I{S}}{\I{S} + \I{BES}}$, le formule del modello sono esprimibili come:
\begin{align*}
    &\I{C} = \I{F} - \frac{\I{R}}{\alpha_R}\\
    &\I{E} = \frac{\I{F}}{\alpha_F} - \I{R}\\
    &\I{B} = \frac{\I{F}}{\beta_F} + \frac{\I{R}}{\beta_R}
\end{align*}
Queste equazioni descrivono le tre correnti statiche del transistore, alle quali vogliamo aggiungere le correnti dinamiche, come nel caso del diodo.
\\
Aggiungiamo quindi due capacità parassita alla giunzione, chiamate rispettivamente $Q_F$ e $Q_R$.

Introduciamo quindi relazioni analoghe a descrivere le due cariche:
\begin{align*}
    &Q_F = Q_{Fs} \xbe = Q_{Fs} \frac{\I{F}}{\I{S}}\\
    &Q_R = Q_{Rs} \xbc= Q_{Rs} \frac{\I{R}}{\I{S}}
\end{align*}
Da cui seguono le relazioni:
\begin{align*}
    \frac{Q_{F}}{\I{F}} = \frac{Q_{Fs}}{\I{S}} = \tau_F\\
    \frac{Q_{R}}{\I{R}} = \frac{Q_{Rs}}{\I{S}} = \tau_R
\end{align*}
Sostituendo queste relazioni alle formule in regime statico otteniamo il modello:
\begin{align*}
    &\I{C} = \frac{Q_F}{\tau_F} - \frac{Q_R}{\alpha_R \tau_R}\\
    &\I{E} = \frac{Q_F}{\alpha_F \tau_F} - \frac{Q_R}{\tau_R}\\
    &\I{B} = \frac{Q_F}{\beta_F \tau_F} - \frac{Q_R}{\beta_R \tau_R}
\end{align*}
In altre parole, siccome tutte le correnti e cariche dipendono linearmente dagli stessi esponenziali, è possibile ricavare una relazione che lega le correnti alle cariche.

Queste equazioni descrivono ancora il modello statico del transistore, ma lega però il valore statico della carica immagazzinata alle due giunzioni. Se ci muoviamo in regime dinamico, non è più vero che la corrente sui condensatori è nulla, ma è esprimibile come $\frac{dQ_F}{dt}$ e $\frac{dQ_R}{dt}$, ottenendo le nuove relazioni:

\begin{align*}
    &\I{C} = \frac{Q_F}{\tau_F} - \frac{Q_R}{\alpha_R \tau_R} {\color{red} - \frac{dQ_R}{dt}}\\
    &\I{E} = \frac{Q_F}{\alpha_F \tau_F} - \frac{Q_R}{\tau_R} {\color{red} + \frac{dQ_F}{dt}}\\
    &\I{B} = \frac{Q_F}{\beta_F \tau_F} - \frac{Q_R}{\beta_R \tau_R} {\color{red} + \frac{dQ_F}{dt} + \frac{dQ_R}{dt} }
\end{align*}

Chiameremo questo modello: \textbf{modello a controllo di carica}, nel senso che le espressioni sono dipendenti dalla carica e non più dalla tensione.
\end{document}
