\documentclass[../elettronica]{subfiles}

\begin{document}
%Lezione 7
\newcommand\I[1]{\ensuremath{I_{\text{#1}}}}
\newcommand\V[1]{\ensuremath{V_{\text{#1}}}}
\def\vbe{$\V{BE}$}

\section{Transistore bipolare a giunzione BJT}
\begin{figure}[h]
    \centering
    \begin{circuitikz}[scale=1.5]
        \draw
        (0, 0) to[diode, v=$V_{BC}$, i<=$I_C$] (2, 0) node[above]{$C$}
        (0, 0) to[diode, v=$V_{BE}$, i=$I_E$] (-2, 0) node[above]{$E$}
        (0, 0) to[short, i<=$I_B$] (0, -1) node[right]{$B$}

        (0, 0) node[circ]{}
        ;
    \end{circuitikz}
    \caption{transistor bipolare}
    \label{circit:transistor_bjt}
\end{figure}
\subsection{Modello di Ebers e Moll}
Supponiamo che $V_{BE} > 0$ e $V_{BC} < 0$. Questo ci permetterà di dire che il diodo tra base
ed emettitore è polarizzato in regione diretta ed il diodo tra base e collettore è polarizzato
in inversa.
Questo comporta che $I_E > 0$ e $I_C \approx 0$ e, considerando la legge di Kirchoff $I_B + I_C = I_E$,
abbiamo $I_B \approx I_E$.

Quello che accade se la distanza tra i due diodi è ridotta (che indicheremo con $w$), l'interazione tra
le loro cariche porta fa cambiare loro comportamento radicalmente, portando a far valere le
relazioni $I_e \approx -I_c$ e $I_b \approx 0$.


Quando $w$ è sufficientemente piccolo, quello che accade è che nel momento in cui il diodo tra base ed emettitore è' polarizzato
in diretta, ed il diodo tra base e collettore è polarizzato in inversa, la corrente fluisce prevalentemente fra collettore
ed emettitore a fronte di una corrente di base molto più piccola rispetto alle altre due regioni.
% 9:40

\begin{wrapfigure}{r}{.4\textwidth}
    \centering
    \vspace{-2\baselineskip}
    \begin{circuitikz}[/tikz/circuitikz/bipoles/length=1cm]
        \draw
        (0, 0)
        to[short, i=$\I{BC}$] (0.7, 0)
        to[diode] (2, 0)
        to[short, i<=$\I{C}$] ++(0.5, 0) node[below]{$C$}
        (0, 0)
        to[short, i=$\I{BE}$] (-0.7, 0)
        to[diode] (-2, 0)
        to[short, i=$\I{E}$] ++(-0.5, 0) node[above]{$E$}
        (0, 0) to[short, i<=$\I{B}$] (0, -1) node[right]{$B$}

        (0, 0) node[circ]{}
        ;
        \draw[code-red] (2, 0)
        to[short, i=${\I{t}}$, color=code-red] (2, 1)
        to[american controlled current source, color=code-red] (-2, 1)
        -- (-2, 0)
        ;

        \draw[code-keyword] (-2, 1)
        -- (-2, 1.8)
        to[american controlled current source, color=code-keyword] ++(4, 0)
        -- (2, 1)
        ;
    \end{circuitikz}
\end{wrapfigure}
\vspace{10pt}
Possiamo tenere conto di questo comportamento attraverso un generatore aggiuntivo di corrente (riportato in rosso) che aggiunga
alla corrente prevista dal modello del diodo polarizzata in inversa, correntemente trascurabile, una nuova corrente
che dipende a sua volta esponenzialmente da \vbe. Chiameremo questa corrente $I_t$: corrente dell'\textbf{effetto transistore}.

\vspace{20pt}
\begin{tcolorbox}[title={Equazioni caratteristiche transistore}, width=\textwidth]
    \begin{align*}
        &\I{BE} = \diodecurrent[\I{BES}]{\V{BE}}\\
        &\I{BC} = \diodecurrent[\I{BCS}]{\V{BC}}\\
        &\I{t} = {\color{code-red} \diodecurrent[\I{S}]{\V{BE}}} - {\color{code-keyword} \diodecurrent[\I{S}]{\V{BC}}}
    \end{align*}
\end{tcolorbox}

\vspace{10pt}
\noindent È semplice calcolare l'espressione della corrente di emettitore $\I{E}$, di collettore $\I{C}$ e di base $\I{B}$,
attraverso le equazioni di Kirchoff ai tre nodi.
Bisogna considerare che abbiamo utilizzato solo una particolare condizione di polarizzazione.
Se studiassimo il caso opposto, ovviamente otterremmo risultati simmetrici: vedremo che con $\V{BE} < 0$ e $\V{BC} > 0$
avremo una componente aggiuntiva di corrente ad effetto transistore, diretta in direzione opposta, rappresentabile anch'essa
con un generatore di corrente pilotato.

%17:39
\begin{tcolorbox}[title=Equazioni caratteristiche transistore]
    \begin{align*}
        &\I{B} = \diodecurrent[\I{BES}]{\V{BE}} - \diodecurrent[\I{BC}]{\V{BC}}\\
        &\I{E} = \diodecurrent[(\I{S} + \I{BE})]{\V{BE}} - \diodecurrent[\I{S}]{\V{BC}}\\
        &\I{C} = \diodecurrent[\I{S}]{\V{BE}} - \diodecurrent[(\I{S} + \I{BCS})]{\V{BC}}\\
    \end{align*}
\end{tcolorbox}

\vspace{20pt}
\begin{wrapfigure}{r}{.2\textwidth}
    \vspace{-2\baselineskip}
    \begin{center}
        \begin{circuitikz}
            \draw (0, 0) node[npn](bjt){}
            (bjt.B) node[ocirc]{} node[left]{B}
            (bjt.E) node[ocirc]{} node[below]{E}
            (bjt.C) node[ocirc]{} node[above]{C}
            ;
        \end{circuitikz}
    \end{center}
    \caption{transistor npn}
    \label{component:transistor_npn}
\end{wrapfigure}

\noindent Il transistore in figura \ref{component:transistor_npn} prevede due giunzioni ed ha il nome di \textbf{transistore npn}.
Esiste anche il suo duale, \textbf{transistore pnp} al quale faremo solo un rapido cenno più avanti ma suo
comportamento è del tutto identico a quello che stiamo discutendo.

Per la legge di Kirchoff abbiamo se conosciamo due tra le differenze di potenziale ai lati del transistore, la terza è
univocamente determinata. Ragionamento del tutto analogo vale per le correnti.

Per determinare completamente il regime di funzionamento del transistore, occorre determinare le 6 grandezze: 3 correnti e 3
tensioni, attraverso 6 equazioni, due delle quali sono quelle di Kirchoff appena indicate.
Per trovare le altre quattro equazioni, utilizziamo lo stesso metodo che abbiamo applicato per determinare le
equazioni del diodo: mettiamo un morsetto a terra e, fornendo un potenziale su uno dei due morsetti rimanenti, misuriamo
il potenziale sull'ultimo morsetto.

Ovviamente è possibile connettere il transistor in 3 modi differenti: emettitore, base e collettore comune. Ma di queste
ultime due ce ne occuperemo più avanti.

\begin{figure}[h]
    \centering
    \begin{circuitikz}[scale=1.2]
        \draw (0, 0)
        to[american voltage source, v=$V_i$, i=$\I{B}$] (0, 2)
        -- (1, 2)
        -- (1, 1) node[npn, anchor=B](tr){}

        (tr.E) -- (tr.E |- 0, 0)
        (0, 0) -- (tr.E |- 0, 0)
        -- ++(1, 0) coordinate(x)

        (tr.C) -- (tr.C |- 0, 2)
        -- ++(1, 0)
        to[R, v^=$V_u$, i<=$\I{C}$] (x)
        ;
    \end{circuitikz}
    \caption{Connessione a Emettitore comune}
\end{figure}

%37:00
\noindent Già osservando il circuito, possiamo notare che due incognite sono eliminate dalla equazione della tensione in ingresso
$\V{i} = \V{BE}$, che possiamo considerare data, e dall'equazione $\V{u} = R\cdot \I{C}$.
Ricordando che la corrente in ingresso corrisponde alla corrente di base, e che la corrente in uscita corrisponde alla
corrente di collettore, per le ultime due equazioni residue possiamo utilizzare quelle fornite dalle caratteristiche del transistore
elencate precedentemente.

\vspace{10pt}
\begin{tcolorbox}
    \begin{align*}
        &\I{B}(\V{BE}, \V{CE}) = \diodecurrent[\I{BES}]{\V{BE}} - \diodecurrent[\I{BC}]{\V{BC}}\\
        &\I{C}(\V{BE}, \V{CE}) = \diodecurrent[\I{S}]{\V{BE}} - \diodecurrent[(\I{S} + \I{BCS})]{\V{BC}}\\
    \end{align*}
\end{tcolorbox}
\vspace{10pt}
Siccome queste relazioni di correnti sono due funzioni di due variabili ($\V{BE}$ e $\V{CE}$), è necessario
un grafico in tre dimensioni per poterle rappresentare graficamente, e ciò non sarebbe pratico.

Per questo motivo riconduciamo queste espressioni ad una rappresentazione più semplice, riconducendoci ad una famiglia di curve parametriche
ponendo $\V{CE}$ come variabile indipendente e tracciando le funzioni al variare di $\V{CE}$ (Figura \ref{graph:transistor_bjt_3d}).

\newpage
\begin{figure}[h]
\begin{minipage}{.47\textwidth}
    \centering
    \def\vt{1}
    \def\kuno{0.001}
    \def\kdue{20}
    \begin{tikzpicture}[declare function={
            ib(\vbe,\vce) = \kuno *(e^(\vbe/\vt)-1) + \kdue *(e^((\vbe - \vce)/\vt -1));
        }]

        \begin{axis}[restrict y to domain=-1:10, ymin=-1, xmin=-1, xlabel=$\V{BE}$, ylabel=$\I{B}$]
            \foreach \i in {4, 4.250, ..., 6}{
                \addplot {ib(x, \i)};
            }

            \draw[->, thick] (4.5, 0.5) -- (2, 2) node[above]{$\V{CE}$};
            \draw
                (0.8, 0.4) node[above]{OFF}
                (4.3, 0.9) node[right] {AD}
                (3.5, 4.5) node[left]{SAT};
        \end{axis}
    \end{tikzpicture}
    \caption{$\I{B}(\V{BE})$ al variare di $\V{CE}$}
    \label{graph:transistor_bjt_3d}
\end{minipage}
\begin{minipage}{.47\textwidth}
    \centering
    \def\vt{1}
    \def\kuno{1.250}
    \def\kdue{3}
    \begin{tikzpicture}[declare function={
            ib(\vbe,\vce) = \kuno *(e^(\vbe/\vt)-1) - \kdue *(e^((\vbe - \vce)/\vt -1));
        }]

        \begin{axis}[restrict y to domain=-1:10, ymin=-1, xmin=-1, xlabel=$\V{CE}$, ylabel=$\I{C}$]
            \foreach \i in {0.25, 0.50, ...,1.5}
            {
                \addplot {ib(\i, x)};
            }

            \draw[->, thick] (3, 0.2)--(3, 4.5)
                node[left]{$\V{BE}$};
            \draw
                (4.5, 4.5) node[above]{AD}
                (4.5, 0.3) node[above]{OFF}
                (1, 3.3) node[above]{SAT};
        \end{axis}
    \end{tikzpicture}
    \caption{$\I{C}(\V{CE})$ al variare di $\V{BE}$}
\end{minipage}
\end{figure}

\vspace{20pt}
\noindent Il transistor è composto da 2 diodi, ognuno polarizzabile in regione diretta o regione inversa. Possiamo riconoscere quindi
quattro regioni di funzionamento del transistore, dipendenti da $\V{BE}$ e $\V{BC}$.
\begin{itemize}
    \item BE on, BC off: Regione normale di funzionamento o regione di polarizzazione attiva diretta
    \item BE off, BC on: Regione di polarizzazione attiva inversa.
    \item BE off, BC off: Regione di interdizione, per brevità diremo che il transistore è spento
    \item BE on, BC on: Regione di saturazione.
\end{itemize}

\end{document}
