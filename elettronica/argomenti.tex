\documentclass{article}

\usepackage[a4paper]{geometry}

\begin{document}
\section*{Lista argomenti teoria elettronica}
\begin{enumerate}
    \item Drogaggio materiali: comportamento della zona drogata di tipo $n$ e della zona drogata tipo $p$. (Atomi accettori ed atomi donatori)
    \item Condizioni di equilibrio: tasso di generazione elettrone-lacuna e tasso di ricombinazione, relazione con concentrazione intrinseca
    \item Potenziale di barriera tra due zone drogate differentemente, relazione tra potenziale di barriera e campo elettrico
    \item Fenomeno di diffusione e trascinamento
    \item Definizione portatori maggioritari e minoritari
    \item Effetto transistore
    \item Esempio di quando si verifica la condizione di pinch-off
    \item Struttura del diodo
    \item Struttura del transistore MOS (Metallo Ossido Semiconduttore)
    \item Struttura del transistore bipolare
    \item Regioni svuotate
    \item Transistor JFET
    \item Tecnologia CMOS (posizione lacune ed elettroni nella regione di semiconduttre)
        Calcolo della carica nelle varie regioni, studio della condizione di pinch-off
    \item Circuiti ratio-less
    \item Margine di immunità ai disturbi
    \item Calcolo del ritardo
    \item Calcolo della potenza statica
    \item Energia richiesta per una singola operazione (tempo per potenza)
    \item Struttura ed utilizzi dell'amplificatore differenziale
\end{enumerate}

\section*{Formule da ricordare}
\begin{enumerate}
    \item
        Forza che il campo elettrico esercita sugli elettroni.
        \[
            F = -q E = ma
        \]

    \item Formula della densità di corrente $J = \frac{I}{S} = \frac{1}{S}\frac{dQ}{dt}$
    \item Relazione tra campo elettrico e potenziale $\Phi$
    \item Formula corrente del diodo
    \item Regime di funzionamento dei transistori MOS
\end{enumerate}
\end{document}
