\section{Diodi}
\begin{figure}[H]
\begin{circuitikz}
    \draw
        (0, -2) to[open, v>=$V_i$] (0, 0)
        to[diode, v=$V$, o-] (3, 0)
        to[R, v=$V_u$] (3, -2)
        to[short, -o] (0, -2);
\end{circuitikz}
\centering
\caption{\label{diodo_1} Circuito base con diodo}
\end{figure}

\bigbreak \( I = I_S \left(e^{\frac{V}{V_T}} -1\right) \)

Su intervallo limitato $V$ ha una regione sufficentemente piccola, da essere approssimabile con una costante

Analogamente su intervallo limitato $I$ \`e approssimabile con una costante

\textbf{Polarizzazione Diretta}: $V = V\gamma$

\textbf{Polarizzazione Inversa}: $I= 0$


\(
\begin{rcases}
    \begin{cases}
        V = V_\gamma\\
        I > 0
    \end{cases}\\
    \begin{cases}
        I = 0\\
        V < V_\gamma
    \end{cases}
\end{rcases}
\)
Modello a soglia


Svolgimento circuito in figura~\ref{diodo_1}:

\begin{itemize}
\item Diodo Spento

    \(
    \begin{rcases}
    I = 0\\
    V_u = RI
    \end{rcases}
    \)
    $V_u = 0$

    \begin{align*}
        &V < V_\gamma\\
        &V = V_i - \cancel{V_u}
    \end{align*}

\item Diodo Acceso

    \(
    \begin{rcases}
        V = V_\gamma\\
        V_u = V_i - V
    \end{rcases}
    \)
    $V_u = V_f - V_\gamma$
\end{itemize}

\begin{tikzpicture}
\draw [->](-1, 0) -- (5, 0);
\draw [->](0, -0.5) -- (0, 3);
\draw [thick, color=red](-1, 0) -- (1, 0)
    -- (5, 2);
\end{tikzpicture}

\subsection{Esercizio}

\begin{circuitikz}
    \draw (0, 0) to[R] (3, 0)
            to[diode] (3, -1)
            to[american voltage source, invert] (3, -2)
            -- (0, -2)
            to[open, v^>=$V_i$] (0, 0);

    \draw (3.5, 0) to[open, v^=$V_u$, anchor=west] (3.5, -2);
\end{circuitikz}

\begin{enumerate}
\item $V_u = E+C$
\item $I = \frac{V_i - V_u}{R}$
\end{enumerate}

\begin{itemize}
    \item Diodo OFF

        \( V < V_\gamma \)

        \(
        \text{\textcircled{2}}
        \begin{cases}
        I = 0\\
        V_i - V_u = 0
        \end{cases}
        \Rightarrow
        \begin{cases}
        I = 0\\
        V_u = V_i
        \end{cases}
        \)

    \item Diodo ON

    \(
    \begin{cases}
        V = V_\gamma\\
        V_u = E + V
    \end{cases}
    \Rightarrow
    \begin{cases}
        V= V_\gamma\\
        V_u = E + V_\gamma
    \end{cases}
    \)

    \( V_u + V_\gamma = E \)
\end{itemize}

\begin{tikzpicture}
\draw [->](-1, 0) -- (5, 0);
\draw [->](0, -0.5) -- (0, 3);
\draw [thick, color=red](-0.5, -1) -- (1, 2)
    -- (5, 2);
\end{tikzpicture}

\subsection{Esercizio}


\begin{circuitikz}
    \draw (0, 0) to[R] (3, 0)
            to[diode, invert] (3, -1)
            to[american voltage source] (3, -2)
            -- (0, -2)
            to[open, v^>=$V_i$] (0, 0);

    \draw (3.5, 0) to[open, v^=$V_u$, anchor=west] (3.5, -2);
\end{circuitikz}

\( I = -\frac{V_i - V_u}{R} \)

\( V_u = -E - V \)

\begin{itemize}
    \item Diodo Acceso

        \(
        \begin{cases}
            I = 0 \\
            V_u = V_i\\
            V_u > 0
        \end{cases}
        \Rightarrow
        \begin{cases}
            V_u > -E - V_\gamma\\
            V_i > - E - V_\gamma
        \end{cases}
        \)
    \item Diodo Spento

        \(
        \begin{cases}
            V = V_\gamma\\
            V_u = -E - V_\gamma
        \end{cases}
        \Rightarrow
        E = V_u - V_\gamma
        \)

        \begin{flalign*}
            &I > 0 &&\\\
            &\frac{V_u - V_i}{R} > 0\\
            &V_i < V_u = -E - V_\gamma
        \end{flalign*}
\end{itemize}

\begin{tikzpicture}
\draw [->](-1, 0) -- (5, 0);
\draw [->](0, -0.5) -- (0, 3);
\draw [thick, color=red](-2, -1) -- (-1, -1)
    -- (3, 3);
\end{tikzpicture}

\subsection{Esercizio}

\begin{circuitikz}
    \draw (0, 0) to[short, o-] (3, 0)
        to[american voltage source] (3, 1.5)
    to[diode, invert] (3, 3)
    to[R, -o, l_=$R$] (0, 3)
    to[open, v=$V_i$] (0, 0);

    \draw (3, 0) -- (5, 0)
    to[american voltage source, invert] (5, 1.5)
    to[diode] (5, 3)
    -- (3, 3);
    \draw(5.5, 3) to[open, v^=$V_u$] (5.5, 0);
\end{circuitikz}

\bigbreak
\(
    \begin{cases}
        \text{\textcircled{1}} \quad \frac{V_i - V_u}{R} = I\\
        \text{\textcircled{2}} \quad I + I_2 = I_1\\
        \text{\textcircled{3}} \quad V_u = E_1 + V_{D1}\\
        \text{\textcircled{4}} \quad V_u = - V_{D2} - E_2
    \end{cases}
\)

\begin{itemize}
    \item $D_1$ e $D_2$ \underline{OFF}

        \(
        \begin{cases}
            I_1 = 0\\
            I_2 = 0
        \end{cases} \quad \xrightarrow{\text{\textcircled{2}}}\quad
        I = 0 \quad\rightarrow\quad V_u = V_i
        \)
        \(
        V_{D1} < V_\gamma
        \)

        \textcircled{3}
        \(
        \begin{cases}
            V_{D1} = V_u - E_1\\
            V_u - E_1 < V_\gamma
        \end{cases}
        \Rightarrow
        \begin{cases}
            V_u < E_1 + V_\gamma\\
            V_i < E_1 + V_\gamma
        \end{cases}
        \)

        \( V_{D2} < V_\gamma \)

        \( V_{D2} = -V_U - E_2\)

        \( V_\gamma > - V_u - E_2 \)

        \( V_i > -V_\gamma - E_1\)

    \item $D_1$ Acceso e $D_2$ Spento


\end{itemize}

\begin{tikzpicture}
\draw [->](-1, 0) -- (5, 0);
\draw [->](0, -0.5) -- (0, 3);
\draw [thick, color=red](-2, -1) -- (-1, -1)
    -- (2, 2) -- (5, 2);
\end{tikzpicture}
