\section{Diodo}
\begin{center}
\begin{circuitikz}
    \draw(0, 0) to[diode, *-*] (2, 0);
\end{circuitikz}
\end{center}

Polarizzare una giunzione significa applicare una differenza di potenziale tra anodo e catodo

\begin{enumerate}
    \item Equilibrio
        \begin{align*}
            & V = 0\\
            & J_{Diff} = J_{Drift}
        \end{align*}
        Corrente totale nulla

    \item Polarizzazione in Diretta
        \begin{align*}
            & V = V_{AL} > 0 \\
            & J_{Diff} > J_{Drift}
        \end{align*}
        Corrente da n a p,  $I \approx I_S e^{\frac{V}{V_T}}$

    \item Polarizzazione Inversa
        \begin{align*}
            & V = V_{AL} < 0 \\
            & J_{Drift} > J_{Diff}
        \end{align*}
        Corrente da p a n, $ I \approx -I_s$
\end{enumerate}

